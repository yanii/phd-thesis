\documentclass[thesis]{subfiles}

\begin{document}
	\chapter{The Unreasonable Affect of Structure on Learning in Neural Networks}
	\label{motivation}
	
	It is well known that the design of a neural network architecture can have a large effect on the generalization of a learned model. And yet, despite this, network design itself remains poorly understood, and is considered somewhat of a black magic, with intuition and experience being the cited motivation behind most common architectures, rather than any theory. This, more than perhaps any other factor, has been a barrier to access for the field.
	
	Beyond hyper-parameters used for tuning the optimization method, such as learning rate, momentum and weight decay, the architecture of a network has a profound effect on the learning. Nowhere is this effect more pronounced than in the case of using neural networks with highly structured inputs, such as natural images. Although neural networks are usually posed as general learning machines, time and again it has been demonstrated that neural networks only truly stand out as a learning method when we encode our prior knowledge of the task in the architecture itself -- what we will denote \emph{structural priors}. 
	
	Neural Networks with structural priors still differ significantly from hand-tuned local features, as popularized in computer vision in the early 2000s, such as SIFT~\citep{Lowe2004}. As compared with neural networks, such local features are rigidly defined in terms of structure and weights, and the learning system is restricted to finding and cataloging the pre-determined features in images. Neural networks with structural priors on the other hand, while restricting the structure of the network somewhat, still allow the network to learn more fine-grained structure, and have no effect on the latitude given to learning weights.
	
	% This is why NN are not good at general learning, and we are now only good at expert systems
	
	The history of understanding the role of neural network architecture in learning is long, arguably going back to the Hebbian rule of learning~\citep{hebb1949organization}, and yet our understanding is still far from complete. In this section we will review a select number of the most important previous works relevant to understanding the role that structural priors play.
	
	\section{Network Architecture}
    A persistent question in training artificial neural networks has been in the design of the networks. Specifically the question of how many parameters should be learned, and in what way they should be connected,  so as to be suitable for good generalization from a given size dataset. Notable steps in the theoretical answers to this question include findings showing the limitations of single layer networks~\citep{minsky1988perceptrons}, information-theoretic measures of the representational capacity of a network~\citep{vapnik2015uniform}, the proof that single hidden-layer networks are universal approximators~\citep{hornik89a}, and the theoretical number of nodes required for generalization from a dataset of given size~\citep{baum1989size}. 
    
    Empirical results have, however, shown that the realities of training neural networks do not match what theory predicts. Deep networks of many hidden layers have been shown time and again to out-perform shallow networks~\citep{Krizhevsky2012,Simonyan2014verydeep,He2015,He2016}, perhaps due to our limited method of optimization~\citep{NIPS2014_5484}. Networks with many more parameters than training samples, that use early-stopping or are regularized strongly, generalize better in practice than networks with the theoretically sufficient capacity~\citep{caruana2001overfitting, Krizhevsky2012, HintonTalk2015}. Networks designed with a specialized connectivity structure closer reflecting the underlying solution have consistently generalized better than fully-connected networks with higher learning capacity~\citep{lecun1989backpropagation,He2016}. In fact these seemingly theoretically defying design strategies can claim to have been responsible for recent breakthroughs in generalization on previously difficult tasks such as image class recognition~\citep{Krizhevsky2012, HintonTalk2015}.
    
    \section{Model Capacity and Representational Power}
    
    \begin{figure}
        \centering
        \includegraphics[width=0.7\textwidth]{Figs/PDF/vcdim_r2lines}
        \caption{All possible labellings of 3 points in $\mathbb{R}^2$ can be separated by a oriented line (2D hyperplane). This is not possible for all labellings of 4 points in $\mathbb{R}^2$ however, and thus the VC dimension of oriented hyperplanes in $\mathbb{R}^2$ is 3. From \citet{burges1998tutorial}.}
        \label{fig:vcdim_r2line}
    \end{figure}
    The information theoretic notion of capacity, that is the expressive power of a classification algorithm, gives important insights to the learning ability of a classification algorithm. Analysis is typically based on the Vapnik–Chervonenkis (VC) dimension~\citep{vapnik2015uniform} of the class of functions used as discriminators, \eg hyperplanes in the case of neural networks. Intuitively, for a discriminative classifier, the VC dimension measures the largest number of points that can be classified without error. In such a case, the set of points is said to be \emph{shattered} by the classifier. A good overview of VC-dimension is given by \citet{burges1998tutorial}.
    
    More formally, a classification model $f(\theta)$, parameterized by $\theta$ is said to \emph{shatter} a set of data points $(x_0, x_1, \ldots, x_h)$ if for for all possible labelings of the points, the classification model can perfectly learn the points. The Vapnik-Chervonenkis (VC) dimension is the largest number of (any) points that can be shattered by such a classifier. For a classifier of VC-dimension $h$, it is sufficient that there exists one set of $h$ points which can be shattered. It is important to note that in general a classifier with a VC-dimension of $h$ will not necessarily shatter every possible set of $h$ points. 
    %Note that, if the VC dimension is h, then there exists at least one set of h points that can be shattered, but it in general it will not be true that every set of h points can be shattered
    
    For example, in Fig.\ \ref{fig:vcdim_r2line}, the function class of oriented hyperplanes, \ie lines in 2D, can separate all possible labellings of 3 points in $\mathbb{R}^2$ -- oriented hyperplanes in $\mathbb{R}^2$ shatter 3 points. However, for 4 points, this is no longer true. It can be proven (see~\citet{burges1998tutorial}) that in general for $\mathbb{R}^n$, the set of oriented hyperplanes shatters any set of $n+1$ points.

    The VC dimension can give a measure of theoretical learning capacity of a classifier, however it can also be somewhat counter-intuitive. While models with large numbers of parameters usually will have a higher VC dimension, there are examples of small single parameter models with infinite VC dimension for more specific sets of points. For example, if we have a set of evenly spaced points in 2D, a simple sinusoidal curve with the appropriate phase can shatter any labeling of an infinite number of such points. However, in general such a classifier would be poor at classifying more general sets of points.
    
    In the case of neural networks, early work showed that the capacity of neural networks is quite large~\citep{hornik89a,baum1989size}.\todo{expand here! put some math}
    
    This however did not match empirical results. Later work showed that rather than capacity being based on solely the number of weights, instead the number of large weights was a better measure~\citep{bartlett1997}
    
    \todo{Include references to \citet{caruana2001overfitting, baum1989size, rethinking2016}}
	
    \subsection{Model Size}
	\begin{chapquote}{John Denker \etal, \textit{Large Automatic Learning, Complex Systems,1987}}
		``A general tabula rasa network is a fine subject for the abstract, formal studies, but one should not try to use it to solve practical problems. \ldots One should pre-program the network with all available information about the structure of the problem, especially information about the symmetry and topology of the data.''
	\end{chapquote}
	
	\begin{figure}[tb]
		\centering
		\begin{subfigure}[t]{0.49\textwidth}
			\resizebox{\linewidth}{!}{%% Creator: Matplotlib, PGF backend
%%
%% To include the figure in your LaTeX document, write
%%   \input{<filename>.pgf}
%%
%% Make sure the required packages are loaded in your preamble
%%   \usepackage{pgf}
%%
%% Figures using additional raster images can only be included by \input if
%% they are in the same directory as the main LaTeX file. For loading figures
%% from other directories you can use the `import` package
%%   \usepackage{import}
%% and then include the figures with
%%   \import{<path to file>}{<filename>.pgf}
%%
%% Matplotlib used the following preamble
%%   \usepackage[utf8x]{inputenc}
%%   \usepackage[T1]{fontenc}
%%
\begingroup%
\makeatletter%
\begin{pgfpicture}%
\pgfpathrectangle{\pgfpointorigin}{\pgfqpoint{4.296389in}{2.655314in}}%
\pgfusepath{use as bounding box, clip}%
\begin{pgfscope}%
\pgfsetbuttcap%
\pgfsetmiterjoin%
\definecolor{currentfill}{rgb}{1.000000,1.000000,1.000000}%
\pgfsetfillcolor{currentfill}%
\pgfsetlinewidth{0.000000pt}%
\definecolor{currentstroke}{rgb}{1.000000,1.000000,1.000000}%
\pgfsetstrokecolor{currentstroke}%
\pgfsetdash{}{0pt}%
\pgfpathmoveto{\pgfqpoint{0.000000in}{0.000000in}}%
\pgfpathlineto{\pgfqpoint{4.296389in}{0.000000in}}%
\pgfpathlineto{\pgfqpoint{4.296389in}{2.655314in}}%
\pgfpathlineto{\pgfqpoint{0.000000in}{2.655314in}}%
\pgfpathclose%
\pgfusepath{fill}%
\end{pgfscope}%
\begin{pgfscope}%
\pgfsetbuttcap%
\pgfsetmiterjoin%
\definecolor{currentfill}{rgb}{1.000000,1.000000,1.000000}%
\pgfsetfillcolor{currentfill}%
\pgfsetlinewidth{0.000000pt}%
\definecolor{currentstroke}{rgb}{0.000000,0.000000,0.000000}%
\pgfsetstrokecolor{currentstroke}%
\pgfsetstrokeopacity{0.000000}%
\pgfsetdash{}{0pt}%
\pgfpathmoveto{\pgfqpoint{0.548317in}{0.386884in}}%
\pgfpathlineto{\pgfqpoint{4.124652in}{0.386884in}}%
\pgfpathlineto{\pgfqpoint{4.124652in}{2.488647in}}%
\pgfpathlineto{\pgfqpoint{0.548317in}{2.488647in}}%
\pgfpathclose%
\pgfusepath{fill}%
\end{pgfscope}%
\begin{pgfscope}%
\pgfsetbuttcap%
\pgfsetroundjoin%
\definecolor{currentfill}{rgb}{0.150000,0.150000,0.150000}%
\pgfsetfillcolor{currentfill}%
\pgfsetlinewidth{1.003750pt}%
\definecolor{currentstroke}{rgb}{0.150000,0.150000,0.150000}%
\pgfsetstrokecolor{currentstroke}%
\pgfsetdash{}{0pt}%
\pgfsys@defobject{currentmarker}{\pgfqpoint{0.000000in}{-0.083333in}}{\pgfqpoint{0.000000in}{0.000000in}}{%
\pgfpathmoveto{\pgfqpoint{0.000000in}{0.000000in}}%
\pgfpathlineto{\pgfqpoint{0.000000in}{-0.083333in}}%
\pgfusepath{stroke,fill}%
}%
\begin{pgfscope}%
\pgfsys@transformshift{0.548317in}{0.386884in}%
\pgfsys@useobject{currentmarker}{}%
\end{pgfscope}%
\end{pgfscope}%
\begin{pgfscope}%
\definecolor{textcolor}{rgb}{0.150000,0.150000,0.150000}%
\pgfsetstrokecolor{textcolor}%
\pgfsetfillcolor{textcolor}%
\pgftext[x=0.548317in,y=0.206329in,,top]{\color{textcolor}\sffamily\fontsize{10.000000}{12.000000}\selectfont \(\displaystyle 0.0\)}%
\end{pgfscope}%
\begin{pgfscope}%
\pgfsetbuttcap%
\pgfsetroundjoin%
\definecolor{currentfill}{rgb}{0.150000,0.150000,0.150000}%
\pgfsetfillcolor{currentfill}%
\pgfsetlinewidth{1.003750pt}%
\definecolor{currentstroke}{rgb}{0.150000,0.150000,0.150000}%
\pgfsetstrokecolor{currentstroke}%
\pgfsetdash{}{0pt}%
\pgfsys@defobject{currentmarker}{\pgfqpoint{0.000000in}{-0.083333in}}{\pgfqpoint{0.000000in}{0.000000in}}{%
\pgfpathmoveto{\pgfqpoint{0.000000in}{0.000000in}}%
\pgfpathlineto{\pgfqpoint{0.000000in}{-0.083333in}}%
\pgfusepath{stroke,fill}%
}%
\begin{pgfscope}%
\pgfsys@transformshift{1.263584in}{0.386884in}%
\pgfsys@useobject{currentmarker}{}%
\end{pgfscope}%
\end{pgfscope}%
\begin{pgfscope}%
\definecolor{textcolor}{rgb}{0.150000,0.150000,0.150000}%
\pgfsetstrokecolor{textcolor}%
\pgfsetfillcolor{textcolor}%
\pgftext[x=1.263584in,y=0.206329in,,top]{\color{textcolor}\sffamily\fontsize{10.000000}{12.000000}\selectfont \(\displaystyle 0.2\)}%
\end{pgfscope}%
\begin{pgfscope}%
\pgfsetbuttcap%
\pgfsetroundjoin%
\definecolor{currentfill}{rgb}{0.150000,0.150000,0.150000}%
\pgfsetfillcolor{currentfill}%
\pgfsetlinewidth{1.003750pt}%
\definecolor{currentstroke}{rgb}{0.150000,0.150000,0.150000}%
\pgfsetstrokecolor{currentstroke}%
\pgfsetdash{}{0pt}%
\pgfsys@defobject{currentmarker}{\pgfqpoint{0.000000in}{-0.083333in}}{\pgfqpoint{0.000000in}{0.000000in}}{%
\pgfpathmoveto{\pgfqpoint{0.000000in}{0.000000in}}%
\pgfpathlineto{\pgfqpoint{0.000000in}{-0.083333in}}%
\pgfusepath{stroke,fill}%
}%
\begin{pgfscope}%
\pgfsys@transformshift{1.978851in}{0.386884in}%
\pgfsys@useobject{currentmarker}{}%
\end{pgfscope}%
\end{pgfscope}%
\begin{pgfscope}%
\definecolor{textcolor}{rgb}{0.150000,0.150000,0.150000}%
\pgfsetstrokecolor{textcolor}%
\pgfsetfillcolor{textcolor}%
\pgftext[x=1.978851in,y=0.206329in,,top]{\color{textcolor}\sffamily\fontsize{10.000000}{12.000000}\selectfont \(\displaystyle 0.4\)}%
\end{pgfscope}%
\begin{pgfscope}%
\pgfsetbuttcap%
\pgfsetroundjoin%
\definecolor{currentfill}{rgb}{0.150000,0.150000,0.150000}%
\pgfsetfillcolor{currentfill}%
\pgfsetlinewidth{1.003750pt}%
\definecolor{currentstroke}{rgb}{0.150000,0.150000,0.150000}%
\pgfsetstrokecolor{currentstroke}%
\pgfsetdash{}{0pt}%
\pgfsys@defobject{currentmarker}{\pgfqpoint{0.000000in}{-0.083333in}}{\pgfqpoint{0.000000in}{0.000000in}}{%
\pgfpathmoveto{\pgfqpoint{0.000000in}{0.000000in}}%
\pgfpathlineto{\pgfqpoint{0.000000in}{-0.083333in}}%
\pgfusepath{stroke,fill}%
}%
\begin{pgfscope}%
\pgfsys@transformshift{2.694118in}{0.386884in}%
\pgfsys@useobject{currentmarker}{}%
\end{pgfscope}%
\end{pgfscope}%
\begin{pgfscope}%
\definecolor{textcolor}{rgb}{0.150000,0.150000,0.150000}%
\pgfsetstrokecolor{textcolor}%
\pgfsetfillcolor{textcolor}%
\pgftext[x=2.694118in,y=0.206329in,,top]{\color{textcolor}\sffamily\fontsize{10.000000}{12.000000}\selectfont \(\displaystyle 0.6\)}%
\end{pgfscope}%
\begin{pgfscope}%
\pgfsetbuttcap%
\pgfsetroundjoin%
\definecolor{currentfill}{rgb}{0.150000,0.150000,0.150000}%
\pgfsetfillcolor{currentfill}%
\pgfsetlinewidth{1.003750pt}%
\definecolor{currentstroke}{rgb}{0.150000,0.150000,0.150000}%
\pgfsetstrokecolor{currentstroke}%
\pgfsetdash{}{0pt}%
\pgfsys@defobject{currentmarker}{\pgfqpoint{0.000000in}{-0.083333in}}{\pgfqpoint{0.000000in}{0.000000in}}{%
\pgfpathmoveto{\pgfqpoint{0.000000in}{0.000000in}}%
\pgfpathlineto{\pgfqpoint{0.000000in}{-0.083333in}}%
\pgfusepath{stroke,fill}%
}%
\begin{pgfscope}%
\pgfsys@transformshift{3.409385in}{0.386884in}%
\pgfsys@useobject{currentmarker}{}%
\end{pgfscope}%
\end{pgfscope}%
\begin{pgfscope}%
\definecolor{textcolor}{rgb}{0.150000,0.150000,0.150000}%
\pgfsetstrokecolor{textcolor}%
\pgfsetfillcolor{textcolor}%
\pgftext[x=3.409385in,y=0.206329in,,top]{\color{textcolor}\sffamily\fontsize{10.000000}{12.000000}\selectfont \(\displaystyle 0.8\)}%
\end{pgfscope}%
\begin{pgfscope}%
\pgfsetbuttcap%
\pgfsetroundjoin%
\definecolor{currentfill}{rgb}{0.150000,0.150000,0.150000}%
\pgfsetfillcolor{currentfill}%
\pgfsetlinewidth{1.003750pt}%
\definecolor{currentstroke}{rgb}{0.150000,0.150000,0.150000}%
\pgfsetstrokecolor{currentstroke}%
\pgfsetdash{}{0pt}%
\pgfsys@defobject{currentmarker}{\pgfqpoint{0.000000in}{-0.083333in}}{\pgfqpoint{0.000000in}{0.000000in}}{%
\pgfpathmoveto{\pgfqpoint{0.000000in}{0.000000in}}%
\pgfpathlineto{\pgfqpoint{0.000000in}{-0.083333in}}%
\pgfusepath{stroke,fill}%
}%
\begin{pgfscope}%
\pgfsys@transformshift{4.124652in}{0.386884in}%
\pgfsys@useobject{currentmarker}{}%
\end{pgfscope}%
\end{pgfscope}%
\begin{pgfscope}%
\definecolor{textcolor}{rgb}{0.150000,0.150000,0.150000}%
\pgfsetstrokecolor{textcolor}%
\pgfsetfillcolor{textcolor}%
\pgftext[x=4.124652in,y=0.206329in,,top]{\color{textcolor}\sffamily\fontsize{10.000000}{12.000000}\selectfont \(\displaystyle 1.0\)}%
\end{pgfscope}%
\begin{pgfscope}%
\pgfsetbuttcap%
\pgfsetroundjoin%
\definecolor{currentfill}{rgb}{0.150000,0.150000,0.150000}%
\pgfsetfillcolor{currentfill}%
\pgfsetlinewidth{1.003750pt}%
\definecolor{currentstroke}{rgb}{0.150000,0.150000,0.150000}%
\pgfsetstrokecolor{currentstroke}%
\pgfsetdash{}{0pt}%
\pgfsys@defobject{currentmarker}{\pgfqpoint{-0.083333in}{0.000000in}}{\pgfqpoint{0.000000in}{0.000000in}}{%
\pgfpathmoveto{\pgfqpoint{0.000000in}{0.000000in}}%
\pgfpathlineto{\pgfqpoint{-0.083333in}{0.000000in}}%
\pgfusepath{stroke,fill}%
}%
\begin{pgfscope}%
\pgfsys@transformshift{0.548317in}{0.386884in}%
\pgfsys@useobject{currentmarker}{}%
\end{pgfscope}%
\end{pgfscope}%
\begin{pgfscope}%
\definecolor{textcolor}{rgb}{0.150000,0.150000,0.150000}%
\pgfsetstrokecolor{textcolor}%
\pgfsetfillcolor{textcolor}%
\pgftext[x=0.082267in,y=0.336742in,left,base]{\color{textcolor}\sffamily\fontsize{10.000000}{12.000000}\selectfont \(\displaystyle -1.0\)}%
\end{pgfscope}%
\begin{pgfscope}%
\pgfsetbuttcap%
\pgfsetroundjoin%
\definecolor{currentfill}{rgb}{0.150000,0.150000,0.150000}%
\pgfsetfillcolor{currentfill}%
\pgfsetlinewidth{1.003750pt}%
\definecolor{currentstroke}{rgb}{0.150000,0.150000,0.150000}%
\pgfsetstrokecolor{currentstroke}%
\pgfsetdash{}{0pt}%
\pgfsys@defobject{currentmarker}{\pgfqpoint{-0.083333in}{0.000000in}}{\pgfqpoint{0.000000in}{0.000000in}}{%
\pgfpathmoveto{\pgfqpoint{0.000000in}{0.000000in}}%
\pgfpathlineto{\pgfqpoint{-0.083333in}{0.000000in}}%
\pgfusepath{stroke,fill}%
}%
\begin{pgfscope}%
\pgfsys@transformshift{0.548317in}{0.912325in}%
\pgfsys@useobject{currentmarker}{}%
\end{pgfscope}%
\end{pgfscope}%
\begin{pgfscope}%
\definecolor{textcolor}{rgb}{0.150000,0.150000,0.150000}%
\pgfsetstrokecolor{textcolor}%
\pgfsetfillcolor{textcolor}%
\pgftext[x=0.082267in,y=0.862183in,left,base]{\color{textcolor}\sffamily\fontsize{10.000000}{12.000000}\selectfont \(\displaystyle -0.5\)}%
\end{pgfscope}%
\begin{pgfscope}%
\pgfsetbuttcap%
\pgfsetroundjoin%
\definecolor{currentfill}{rgb}{0.150000,0.150000,0.150000}%
\pgfsetfillcolor{currentfill}%
\pgfsetlinewidth{1.003750pt}%
\definecolor{currentstroke}{rgb}{0.150000,0.150000,0.150000}%
\pgfsetstrokecolor{currentstroke}%
\pgfsetdash{}{0pt}%
\pgfsys@defobject{currentmarker}{\pgfqpoint{-0.083333in}{0.000000in}}{\pgfqpoint{0.000000in}{0.000000in}}{%
\pgfpathmoveto{\pgfqpoint{0.000000in}{0.000000in}}%
\pgfpathlineto{\pgfqpoint{-0.083333in}{0.000000in}}%
\pgfusepath{stroke,fill}%
}%
\begin{pgfscope}%
\pgfsys@transformshift{0.548317in}{1.437766in}%
\pgfsys@useobject{currentmarker}{}%
\end{pgfscope}%
\end{pgfscope}%
\begin{pgfscope}%
\definecolor{textcolor}{rgb}{0.150000,0.150000,0.150000}%
\pgfsetstrokecolor{textcolor}%
\pgfsetfillcolor{textcolor}%
\pgftext[x=0.190292in,y=1.387624in,left,base]{\color{textcolor}\sffamily\fontsize{10.000000}{12.000000}\selectfont \(\displaystyle 0.0\)}%
\end{pgfscope}%
\begin{pgfscope}%
\pgfsetbuttcap%
\pgfsetroundjoin%
\definecolor{currentfill}{rgb}{0.150000,0.150000,0.150000}%
\pgfsetfillcolor{currentfill}%
\pgfsetlinewidth{1.003750pt}%
\definecolor{currentstroke}{rgb}{0.150000,0.150000,0.150000}%
\pgfsetstrokecolor{currentstroke}%
\pgfsetdash{}{0pt}%
\pgfsys@defobject{currentmarker}{\pgfqpoint{-0.083333in}{0.000000in}}{\pgfqpoint{0.000000in}{0.000000in}}{%
\pgfpathmoveto{\pgfqpoint{0.000000in}{0.000000in}}%
\pgfpathlineto{\pgfqpoint{-0.083333in}{0.000000in}}%
\pgfusepath{stroke,fill}%
}%
\begin{pgfscope}%
\pgfsys@transformshift{0.548317in}{1.963207in}%
\pgfsys@useobject{currentmarker}{}%
\end{pgfscope}%
\end{pgfscope}%
\begin{pgfscope}%
\definecolor{textcolor}{rgb}{0.150000,0.150000,0.150000}%
\pgfsetstrokecolor{textcolor}%
\pgfsetfillcolor{textcolor}%
\pgftext[x=0.190292in,y=1.913065in,left,base]{\color{textcolor}\sffamily\fontsize{10.000000}{12.000000}\selectfont \(\displaystyle 0.5\)}%
\end{pgfscope}%
\begin{pgfscope}%
\pgfsetbuttcap%
\pgfsetroundjoin%
\definecolor{currentfill}{rgb}{0.150000,0.150000,0.150000}%
\pgfsetfillcolor{currentfill}%
\pgfsetlinewidth{1.003750pt}%
\definecolor{currentstroke}{rgb}{0.150000,0.150000,0.150000}%
\pgfsetstrokecolor{currentstroke}%
\pgfsetdash{}{0pt}%
\pgfsys@defobject{currentmarker}{\pgfqpoint{-0.083333in}{0.000000in}}{\pgfqpoint{0.000000in}{0.000000in}}{%
\pgfpathmoveto{\pgfqpoint{0.000000in}{0.000000in}}%
\pgfpathlineto{\pgfqpoint{-0.083333in}{0.000000in}}%
\pgfusepath{stroke,fill}%
}%
\begin{pgfscope}%
\pgfsys@transformshift{0.548317in}{2.488647in}%
\pgfsys@useobject{currentmarker}{}%
\end{pgfscope}%
\end{pgfscope}%
\begin{pgfscope}%
\definecolor{textcolor}{rgb}{0.150000,0.150000,0.150000}%
\pgfsetstrokecolor{textcolor}%
\pgfsetfillcolor{textcolor}%
\pgftext[x=0.190292in,y=2.438505in,left,base]{\color{textcolor}\sffamily\fontsize{10.000000}{12.000000}\selectfont \(\displaystyle 1.0\)}%
\end{pgfscope}%
\begin{pgfscope}%
\pgfpathrectangle{\pgfqpoint{0.548317in}{0.386884in}}{\pgfqpoint{3.576335in}{2.101763in}} %
\pgfusepath{clip}%
\pgfsetbuttcap%
\pgfsetroundjoin%
\definecolor{currentfill}{rgb}{0.400000,0.760784,0.647059}%
\pgfsetfillcolor{currentfill}%
\pgfsetlinewidth{0.000000pt}%
\definecolor{currentstroke}{rgb}{0.400000,0.760784,0.647059}%
\pgfsetstrokecolor{currentstroke}%
\pgfsetdash{}{0pt}%
\pgfsys@defobject{currentmarker}{\pgfqpoint{-0.048611in}{-0.048611in}}{\pgfqpoint{0.048611in}{0.048611in}}{%
\pgfpathmoveto{\pgfqpoint{0.000000in}{-0.048611in}}%
\pgfpathcurveto{\pgfqpoint{0.012892in}{-0.048611in}}{\pgfqpoint{0.025257in}{-0.043489in}}{\pgfqpoint{0.034373in}{-0.034373in}}%
\pgfpathcurveto{\pgfqpoint{0.043489in}{-0.025257in}}{\pgfqpoint{0.048611in}{-0.012892in}}{\pgfqpoint{0.048611in}{0.000000in}}%
\pgfpathcurveto{\pgfqpoint{0.048611in}{0.012892in}}{\pgfqpoint{0.043489in}{0.025257in}}{\pgfqpoint{0.034373in}{0.034373in}}%
\pgfpathcurveto{\pgfqpoint{0.025257in}{0.043489in}}{\pgfqpoint{0.012892in}{0.048611in}}{\pgfqpoint{0.000000in}{0.048611in}}%
\pgfpathcurveto{\pgfqpoint{-0.012892in}{0.048611in}}{\pgfqpoint{-0.025257in}{0.043489in}}{\pgfqpoint{-0.034373in}{0.034373in}}%
\pgfpathcurveto{\pgfqpoint{-0.043489in}{0.025257in}}{\pgfqpoint{-0.048611in}{0.012892in}}{\pgfqpoint{-0.048611in}{0.000000in}}%
\pgfpathcurveto{\pgfqpoint{-0.048611in}{-0.012892in}}{\pgfqpoint{-0.043489in}{-0.025257in}}{\pgfqpoint{-0.034373in}{-0.034373in}}%
\pgfpathcurveto{\pgfqpoint{-0.025257in}{-0.043489in}}{\pgfqpoint{-0.012892in}{-0.048611in}}{\pgfqpoint{0.000000in}{-0.048611in}}%
\pgfpathclose%
\pgfusepath{fill}%
}%
\begin{pgfscope}%
\pgfsys@transformshift{2.260811in}{0.754316in}%
\pgfsys@useobject{currentmarker}{}%
\end{pgfscope}%
\begin{pgfscope}%
\pgfsys@transformshift{2.611975in}{0.899184in}%
\pgfsys@useobject{currentmarker}{}%
\end{pgfscope}%
\begin{pgfscope}%
\pgfsys@transformshift{3.012052in}{1.144101in}%
\pgfsys@useobject{currentmarker}{}%
\end{pgfscope}%
\end{pgfscope}%
\begin{pgfscope}%
\pgfpathrectangle{\pgfqpoint{0.548317in}{0.386884in}}{\pgfqpoint{3.576335in}{2.101763in}} %
\pgfusepath{clip}%
\pgfsetbuttcap%
\pgfsetroundjoin%
\pgfsetlinewidth{1.756562pt}%
\definecolor{currentstroke}{rgb}{0.988235,0.552941,0.384314}%
\pgfsetstrokecolor{currentstroke}%
\pgfsetdash{{5.600000pt}{2.400000pt}}{0.000000pt}%
\pgfpathmoveto{\pgfqpoint{0.534428in}{0.382588in}}%
\pgfpathlineto{\pgfqpoint{0.584442in}{0.397288in}}%
\pgfpathlineto{\pgfqpoint{0.656691in}{0.416887in}}%
\pgfpathlineto{\pgfqpoint{0.728940in}{0.435004in}}%
\pgfpathlineto{\pgfqpoint{0.801189in}{0.451796in}}%
\pgfpathlineto{\pgfqpoint{0.873438in}{0.467418in}}%
\pgfpathlineto{\pgfqpoint{0.945688in}{0.482026in}}%
\pgfpathlineto{\pgfqpoint{1.017937in}{0.495776in}}%
\pgfpathlineto{\pgfqpoint{1.090186in}{0.508825in}}%
\pgfpathlineto{\pgfqpoint{1.162435in}{0.521328in}}%
\pgfpathlineto{\pgfqpoint{1.234684in}{0.533440in}}%
\pgfpathlineto{\pgfqpoint{1.306933in}{0.545319in}}%
\pgfpathlineto{\pgfqpoint{1.379183in}{0.557120in}}%
\pgfpathlineto{\pgfqpoint{1.451432in}{0.568999in}}%
\pgfpathlineto{\pgfqpoint{1.523681in}{0.581112in}}%
\pgfpathlineto{\pgfqpoint{1.595930in}{0.593615in}}%
\pgfpathlineto{\pgfqpoint{1.668179in}{0.606663in}}%
\pgfpathlineto{\pgfqpoint{1.740429in}{0.620414in}}%
\pgfpathlineto{\pgfqpoint{1.812678in}{0.635022in}}%
\pgfpathlineto{\pgfqpoint{1.884927in}{0.650644in}}%
\pgfpathlineto{\pgfqpoint{1.957176in}{0.667435in}}%
\pgfpathlineto{\pgfqpoint{2.029425in}{0.685553in}}%
\pgfpathlineto{\pgfqpoint{2.101675in}{0.705151in}}%
\pgfpathlineto{\pgfqpoint{2.173924in}{0.726388in}}%
\pgfpathlineto{\pgfqpoint{2.246173in}{0.749418in}}%
\pgfpathlineto{\pgfqpoint{2.318422in}{0.774397in}}%
\pgfpathlineto{\pgfqpoint{2.390671in}{0.801482in}}%
\pgfpathlineto{\pgfqpoint{2.462921in}{0.830828in}}%
\pgfpathlineto{\pgfqpoint{2.535170in}{0.862592in}}%
\pgfpathlineto{\pgfqpoint{2.607419in}{0.896929in}}%
\pgfpathlineto{\pgfqpoint{2.679668in}{0.933995in}}%
\pgfpathlineto{\pgfqpoint{2.751917in}{0.973947in}}%
\pgfpathlineto{\pgfqpoint{2.824166in}{1.016940in}}%
\pgfpathlineto{\pgfqpoint{2.896416in}{1.063129in}}%
\pgfpathlineto{\pgfqpoint{2.968665in}{1.112672in}}%
\pgfpathlineto{\pgfqpoint{3.040914in}{1.165725in}}%
\pgfpathlineto{\pgfqpoint{3.113163in}{1.222442in}}%
\pgfpathlineto{\pgfqpoint{3.185412in}{1.282980in}}%
\pgfpathlineto{\pgfqpoint{3.257662in}{1.347495in}}%
\pgfpathlineto{\pgfqpoint{3.329911in}{1.416143in}}%
\pgfpathlineto{\pgfqpoint{3.402160in}{1.489080in}}%
\pgfpathlineto{\pgfqpoint{3.474409in}{1.566461in}}%
\pgfpathlineto{\pgfqpoint{3.546658in}{1.648444in}}%
\pgfpathlineto{\pgfqpoint{3.618908in}{1.735183in}}%
\pgfpathlineto{\pgfqpoint{3.691157in}{1.826835in}}%
\pgfpathlineto{\pgfqpoint{3.763406in}{1.923556in}}%
\pgfpathlineto{\pgfqpoint{3.835655in}{2.025501in}}%
\pgfpathlineto{\pgfqpoint{3.907904in}{2.132827in}}%
\pgfpathlineto{\pgfqpoint{3.980154in}{2.245689in}}%
\pgfpathlineto{\pgfqpoint{4.052403in}{2.364244in}}%
\pgfpathlineto{\pgfqpoint{4.124652in}{2.488647in}}%
\pgfusepath{stroke}%
\end{pgfscope}%
\begin{pgfscope}%
\pgfpathrectangle{\pgfqpoint{0.548317in}{0.386884in}}{\pgfqpoint{3.576335in}{2.101763in}} %
\pgfusepath{clip}%
\pgfsetroundcap%
\pgfsetroundjoin%
\pgfsetlinewidth{1.756562pt}%
\definecolor{currentstroke}{rgb}{0.552941,0.627451,0.796078}%
\pgfsetstrokecolor{currentstroke}%
\pgfsetdash{}{0pt}%
\pgfpathmoveto{\pgfqpoint{0.534428in}{0.664235in}}%
\pgfpathlineto{\pgfqpoint{0.584442in}{0.658600in}}%
\pgfpathlineto{\pgfqpoint{0.656691in}{0.650850in}}%
\pgfpathlineto{\pgfqpoint{0.728940in}{0.643574in}}%
\pgfpathlineto{\pgfqpoint{0.801189in}{0.636856in}}%
\pgfpathlineto{\pgfqpoint{0.873438in}{0.630780in}}%
\pgfpathlineto{\pgfqpoint{0.945688in}{0.625432in}}%
\pgfpathlineto{\pgfqpoint{1.017937in}{0.620894in}}%
\pgfpathlineto{\pgfqpoint{1.090186in}{0.617252in}}%
\pgfpathlineto{\pgfqpoint{1.162435in}{0.614590in}}%
\pgfpathlineto{\pgfqpoint{1.234684in}{0.612992in}}%
\pgfpathlineto{\pgfqpoint{1.306933in}{0.612542in}}%
\pgfpathlineto{\pgfqpoint{1.379183in}{0.613325in}}%
\pgfpathlineto{\pgfqpoint{1.451432in}{0.615424in}}%
\pgfpathlineto{\pgfqpoint{1.523681in}{0.618925in}}%
\pgfpathlineto{\pgfqpoint{1.595930in}{0.623912in}}%
\pgfpathlineto{\pgfqpoint{1.668179in}{0.630468in}}%
\pgfpathlineto{\pgfqpoint{1.740429in}{0.638678in}}%
\pgfpathlineto{\pgfqpoint{1.812678in}{0.648627in}}%
\pgfpathlineto{\pgfqpoint{1.884927in}{0.660399in}}%
\pgfpathlineto{\pgfqpoint{1.957176in}{0.674077in}}%
\pgfpathlineto{\pgfqpoint{2.029425in}{0.689747in}}%
\pgfpathlineto{\pgfqpoint{2.101675in}{0.707493in}}%
\pgfpathlineto{\pgfqpoint{2.173924in}{0.727398in}}%
\pgfpathlineto{\pgfqpoint{2.246173in}{0.749548in}}%
\pgfpathlineto{\pgfqpoint{2.318422in}{0.774026in}}%
\pgfpathlineto{\pgfqpoint{2.390671in}{0.800917in}}%
\pgfpathlineto{\pgfqpoint{2.462921in}{0.830305in}}%
\pgfpathlineto{\pgfqpoint{2.535170in}{0.862274in}}%
\pgfpathlineto{\pgfqpoint{2.607419in}{0.896909in}}%
\pgfpathlineto{\pgfqpoint{2.679668in}{0.934294in}}%
\pgfpathlineto{\pgfqpoint{2.751917in}{0.974513in}}%
\pgfpathlineto{\pgfqpoint{2.824166in}{1.017651in}}%
\pgfpathlineto{\pgfqpoint{2.896416in}{1.063792in}}%
\pgfpathlineto{\pgfqpoint{2.968665in}{1.113019in}}%
\pgfpathlineto{\pgfqpoint{3.040914in}{1.165419in}}%
\pgfpathlineto{\pgfqpoint{3.113163in}{1.221074in}}%
\pgfpathlineto{\pgfqpoint{3.185412in}{1.280069in}}%
\pgfpathlineto{\pgfqpoint{3.257662in}{1.342488in}}%
\pgfpathlineto{\pgfqpoint{3.329911in}{1.408416in}}%
\pgfpathlineto{\pgfqpoint{3.402160in}{1.477937in}}%
\pgfpathlineto{\pgfqpoint{3.474409in}{1.551136in}}%
\pgfpathlineto{\pgfqpoint{3.546658in}{1.628095in}}%
\pgfpathlineto{\pgfqpoint{3.618908in}{1.708901in}}%
\pgfpathlineto{\pgfqpoint{3.691157in}{1.793637in}}%
\pgfpathlineto{\pgfqpoint{3.763406in}{1.882387in}}%
\pgfpathlineto{\pgfqpoint{3.835655in}{1.975236in}}%
\pgfpathlineto{\pgfqpoint{3.907904in}{2.072268in}}%
\pgfpathlineto{\pgfqpoint{3.980154in}{2.173567in}}%
\pgfpathlineto{\pgfqpoint{4.052403in}{2.279218in}}%
\pgfpathlineto{\pgfqpoint{4.124652in}{2.389305in}}%
\pgfusepath{stroke}%
\end{pgfscope}%
\begin{pgfscope}%
\pgfsetrectcap%
\pgfsetmiterjoin%
\pgfsetlinewidth{1.254687pt}%
\definecolor{currentstroke}{rgb}{0.150000,0.150000,0.150000}%
\pgfsetstrokecolor{currentstroke}%
\pgfsetdash{}{0pt}%
\pgfpathmoveto{\pgfqpoint{0.548317in}{0.386884in}}%
\pgfpathlineto{\pgfqpoint{0.548317in}{2.488647in}}%
\pgfusepath{stroke}%
\end{pgfscope}%
\begin{pgfscope}%
\pgfsetrectcap%
\pgfsetmiterjoin%
\pgfsetlinewidth{1.254687pt}%
\definecolor{currentstroke}{rgb}{0.150000,0.150000,0.150000}%
\pgfsetstrokecolor{currentstroke}%
\pgfsetdash{}{0pt}%
\pgfpathmoveto{\pgfqpoint{0.548317in}{0.386884in}}%
\pgfpathlineto{\pgfqpoint{4.124652in}{0.386884in}}%
\pgfusepath{stroke}%
\end{pgfscope}%
\begin{pgfscope}%
\pgfsetbuttcap%
\pgfsetroundjoin%
\definecolor{currentfill}{rgb}{0.400000,0.760784,0.647059}%
\pgfsetfillcolor{currentfill}%
\pgfsetlinewidth{0.000000pt}%
\definecolor{currentstroke}{rgb}{0.400000,0.760784,0.647059}%
\pgfsetstrokecolor{currentstroke}%
\pgfsetdash{}{0pt}%
\pgfsys@defobject{currentmarker}{\pgfqpoint{-0.048611in}{-0.048611in}}{\pgfqpoint{0.048611in}{0.048611in}}{%
\pgfpathmoveto{\pgfqpoint{0.000000in}{-0.048611in}}%
\pgfpathcurveto{\pgfqpoint{0.012892in}{-0.048611in}}{\pgfqpoint{0.025257in}{-0.043489in}}{\pgfqpoint{0.034373in}{-0.034373in}}%
\pgfpathcurveto{\pgfqpoint{0.043489in}{-0.025257in}}{\pgfqpoint{0.048611in}{-0.012892in}}{\pgfqpoint{0.048611in}{0.000000in}}%
\pgfpathcurveto{\pgfqpoint{0.048611in}{0.012892in}}{\pgfqpoint{0.043489in}{0.025257in}}{\pgfqpoint{0.034373in}{0.034373in}}%
\pgfpathcurveto{\pgfqpoint{0.025257in}{0.043489in}}{\pgfqpoint{0.012892in}{0.048611in}}{\pgfqpoint{0.000000in}{0.048611in}}%
\pgfpathcurveto{\pgfqpoint{-0.012892in}{0.048611in}}{\pgfqpoint{-0.025257in}{0.043489in}}{\pgfqpoint{-0.034373in}{0.034373in}}%
\pgfpathcurveto{\pgfqpoint{-0.043489in}{0.025257in}}{\pgfqpoint{-0.048611in}{0.012892in}}{\pgfqpoint{-0.048611in}{0.000000in}}%
\pgfpathcurveto{\pgfqpoint{-0.048611in}{-0.012892in}}{\pgfqpoint{-0.043489in}{-0.025257in}}{\pgfqpoint{-0.034373in}{-0.034373in}}%
\pgfpathcurveto{\pgfqpoint{-0.025257in}{-0.043489in}}{\pgfqpoint{-0.012892in}{-0.048611in}}{\pgfqpoint{0.000000in}{-0.048611in}}%
\pgfpathclose%
\pgfusepath{fill}%
}%
\begin{pgfscope}%
\pgfsys@transformshift{0.812206in}{2.311974in}%
\pgfsys@useobject{currentmarker}{}%
\end{pgfscope}%
\end{pgfscope}%
\begin{pgfscope}%
\definecolor{textcolor}{rgb}{0.150000,0.150000,0.150000}%
\pgfsetstrokecolor{textcolor}%
\pgfsetfillcolor{textcolor}%
\pgftext[x=1.062206in,y=2.263363in,left,base]{\color{textcolor}\sffamily\fontsize{10.000000}{12.000000}\selectfont samples}%
\end{pgfscope}%
\begin{pgfscope}%
\pgfsetbuttcap%
\pgfsetroundjoin%
\pgfsetlinewidth{1.756562pt}%
\definecolor{currentstroke}{rgb}{0.988235,0.552941,0.384314}%
\pgfsetstrokecolor{currentstroke}%
\pgfsetdash{{5.600000pt}{2.400000pt}}{0.000000pt}%
\pgfpathmoveto{\pgfqpoint{0.673317in}{2.115246in}}%
\pgfpathlineto{\pgfqpoint{0.951095in}{2.115246in}}%
\pgfusepath{stroke}%
\end{pgfscope}%
\begin{pgfscope}%
\definecolor{textcolor}{rgb}{0.150000,0.150000,0.150000}%
\pgfsetstrokecolor{textcolor}%
\pgfsetfillcolor{textcolor}%
\pgftext[x=1.062206in,y=2.066635in,left,base]{\color{textcolor}\sffamily\fontsize{10.000000}{12.000000}\selectfont function}%
\end{pgfscope}%
\begin{pgfscope}%
\pgfsetroundcap%
\pgfsetroundjoin%
\pgfsetlinewidth{1.756562pt}%
\definecolor{currentstroke}{rgb}{0.552941,0.627451,0.796078}%
\pgfsetstrokecolor{currentstroke}%
\pgfsetdash{}{0pt}%
\pgfpathmoveto{\pgfqpoint{0.673317in}{1.918518in}}%
\pgfpathlineto{\pgfqpoint{0.951095in}{1.918518in}}%
\pgfusepath{stroke}%
\end{pgfscope}%
\begin{pgfscope}%
\definecolor{textcolor}{rgb}{0.150000,0.150000,0.150000}%
\pgfsetstrokecolor{textcolor}%
\pgfsetfillcolor{textcolor}%
\pgftext[x=1.062206in,y=1.869907in,left,base]{\color{textcolor}\sffamily\fontsize{10.000000}{12.000000}\selectfont 3rd order fit}%
\end{pgfscope}%
\end{pgfpicture}%
\makeatother%
\endgroup%
}
			\caption{\engordnumber{3}-order polynomial fitting 3 points}
			\label{fig:polyfit3rd}
		\end{subfigure}
		~
		\begin{subfigure}[t]{0.49\textwidth}
			\resizebox{\linewidth}{!}{\input{Figs/matplotlib/polyfit20th.pgf}}
			\caption{\engordnumber{20}-order polynomial fitting 3 points}
			\label{fig:polyfit20th}
		\end{subfigure}\\	
		\begin{subfigure}[t]{0.49\textwidth}
			\resizebox{\linewidth}{!}{\input{Figs/matplotlib/polyfit3rdlots.pgf}}
			\caption{\engordnumber{3}-order polynomial fitting 10 points}
			\label{fig:polyfit3rdlots}
		\end{subfigure}
		~
		\begin{subfigure}[t]{0.49\textwidth}
			\resizebox{\linewidth}{!}{%% Creator: Matplotlib, PGF backend
%%
%% To include the figure in your LaTeX document, write
%%   \input{<filename>.pgf}
%%
%% Make sure the required packages are loaded in your preamble
%%   \usepackage{pgf}
%%
%% Figures using additional raster images can only be included by \input if
%% they are in the same directory as the main LaTeX file. For loading figures
%% from other directories you can use the `import` package
%%   \usepackage{import}
%% and then include the figures with
%%   \import{<path to file>}{<filename>.pgf}
%%
%% Matplotlib used the following preamble
%%   \usepackage[utf8x]{inputenc}
%%   \usepackage[T1]{fontenc}
%%
\begingroup%
\makeatletter%
\begin{pgfpicture}%
\pgfpathrectangle{\pgfpointorigin}{\pgfqpoint{4.296389in}{2.655314in}}%
\pgfusepath{use as bounding box, clip}%
\begin{pgfscope}%
\pgfsetbuttcap%
\pgfsetmiterjoin%
\definecolor{currentfill}{rgb}{1.000000,1.000000,1.000000}%
\pgfsetfillcolor{currentfill}%
\pgfsetlinewidth{0.000000pt}%
\definecolor{currentstroke}{rgb}{1.000000,1.000000,1.000000}%
\pgfsetstrokecolor{currentstroke}%
\pgfsetdash{}{0pt}%
\pgfpathmoveto{\pgfqpoint{0.000000in}{0.000000in}}%
\pgfpathlineto{\pgfqpoint{4.296389in}{0.000000in}}%
\pgfpathlineto{\pgfqpoint{4.296389in}{2.655314in}}%
\pgfpathlineto{\pgfqpoint{0.000000in}{2.655314in}}%
\pgfpathclose%
\pgfusepath{fill}%
\end{pgfscope}%
\begin{pgfscope}%
\pgfsetbuttcap%
\pgfsetmiterjoin%
\definecolor{currentfill}{rgb}{1.000000,1.000000,1.000000}%
\pgfsetfillcolor{currentfill}%
\pgfsetlinewidth{0.000000pt}%
\definecolor{currentstroke}{rgb}{0.000000,0.000000,0.000000}%
\pgfsetstrokecolor{currentstroke}%
\pgfsetstrokeopacity{0.000000}%
\pgfsetdash{}{0pt}%
\pgfpathmoveto{\pgfqpoint{0.548317in}{0.386884in}}%
\pgfpathlineto{\pgfqpoint{4.124652in}{0.386884in}}%
\pgfpathlineto{\pgfqpoint{4.124652in}{2.488647in}}%
\pgfpathlineto{\pgfqpoint{0.548317in}{2.488647in}}%
\pgfpathclose%
\pgfusepath{fill}%
\end{pgfscope}%
\begin{pgfscope}%
\pgfsetbuttcap%
\pgfsetroundjoin%
\definecolor{currentfill}{rgb}{0.150000,0.150000,0.150000}%
\pgfsetfillcolor{currentfill}%
\pgfsetlinewidth{1.003750pt}%
\definecolor{currentstroke}{rgb}{0.150000,0.150000,0.150000}%
\pgfsetstrokecolor{currentstroke}%
\pgfsetdash{}{0pt}%
\pgfsys@defobject{currentmarker}{\pgfqpoint{0.000000in}{-0.083333in}}{\pgfqpoint{0.000000in}{0.000000in}}{%
\pgfpathmoveto{\pgfqpoint{0.000000in}{0.000000in}}%
\pgfpathlineto{\pgfqpoint{0.000000in}{-0.083333in}}%
\pgfusepath{stroke,fill}%
}%
\begin{pgfscope}%
\pgfsys@transformshift{0.548317in}{0.386884in}%
\pgfsys@useobject{currentmarker}{}%
\end{pgfscope}%
\end{pgfscope}%
\begin{pgfscope}%
\definecolor{textcolor}{rgb}{0.150000,0.150000,0.150000}%
\pgfsetstrokecolor{textcolor}%
\pgfsetfillcolor{textcolor}%
\pgftext[x=0.548317in,y=0.206329in,,top]{\color{textcolor}\sffamily\fontsize{10.000000}{12.000000}\selectfont \(\displaystyle 0.0\)}%
\end{pgfscope}%
\begin{pgfscope}%
\pgfsetbuttcap%
\pgfsetroundjoin%
\definecolor{currentfill}{rgb}{0.150000,0.150000,0.150000}%
\pgfsetfillcolor{currentfill}%
\pgfsetlinewidth{1.003750pt}%
\definecolor{currentstroke}{rgb}{0.150000,0.150000,0.150000}%
\pgfsetstrokecolor{currentstroke}%
\pgfsetdash{}{0pt}%
\pgfsys@defobject{currentmarker}{\pgfqpoint{0.000000in}{-0.083333in}}{\pgfqpoint{0.000000in}{0.000000in}}{%
\pgfpathmoveto{\pgfqpoint{0.000000in}{0.000000in}}%
\pgfpathlineto{\pgfqpoint{0.000000in}{-0.083333in}}%
\pgfusepath{stroke,fill}%
}%
\begin{pgfscope}%
\pgfsys@transformshift{1.263584in}{0.386884in}%
\pgfsys@useobject{currentmarker}{}%
\end{pgfscope}%
\end{pgfscope}%
\begin{pgfscope}%
\definecolor{textcolor}{rgb}{0.150000,0.150000,0.150000}%
\pgfsetstrokecolor{textcolor}%
\pgfsetfillcolor{textcolor}%
\pgftext[x=1.263584in,y=0.206329in,,top]{\color{textcolor}\sffamily\fontsize{10.000000}{12.000000}\selectfont \(\displaystyle 0.2\)}%
\end{pgfscope}%
\begin{pgfscope}%
\pgfsetbuttcap%
\pgfsetroundjoin%
\definecolor{currentfill}{rgb}{0.150000,0.150000,0.150000}%
\pgfsetfillcolor{currentfill}%
\pgfsetlinewidth{1.003750pt}%
\definecolor{currentstroke}{rgb}{0.150000,0.150000,0.150000}%
\pgfsetstrokecolor{currentstroke}%
\pgfsetdash{}{0pt}%
\pgfsys@defobject{currentmarker}{\pgfqpoint{0.000000in}{-0.083333in}}{\pgfqpoint{0.000000in}{0.000000in}}{%
\pgfpathmoveto{\pgfqpoint{0.000000in}{0.000000in}}%
\pgfpathlineto{\pgfqpoint{0.000000in}{-0.083333in}}%
\pgfusepath{stroke,fill}%
}%
\begin{pgfscope}%
\pgfsys@transformshift{1.978851in}{0.386884in}%
\pgfsys@useobject{currentmarker}{}%
\end{pgfscope}%
\end{pgfscope}%
\begin{pgfscope}%
\definecolor{textcolor}{rgb}{0.150000,0.150000,0.150000}%
\pgfsetstrokecolor{textcolor}%
\pgfsetfillcolor{textcolor}%
\pgftext[x=1.978851in,y=0.206329in,,top]{\color{textcolor}\sffamily\fontsize{10.000000}{12.000000}\selectfont \(\displaystyle 0.4\)}%
\end{pgfscope}%
\begin{pgfscope}%
\pgfsetbuttcap%
\pgfsetroundjoin%
\definecolor{currentfill}{rgb}{0.150000,0.150000,0.150000}%
\pgfsetfillcolor{currentfill}%
\pgfsetlinewidth{1.003750pt}%
\definecolor{currentstroke}{rgb}{0.150000,0.150000,0.150000}%
\pgfsetstrokecolor{currentstroke}%
\pgfsetdash{}{0pt}%
\pgfsys@defobject{currentmarker}{\pgfqpoint{0.000000in}{-0.083333in}}{\pgfqpoint{0.000000in}{0.000000in}}{%
\pgfpathmoveto{\pgfqpoint{0.000000in}{0.000000in}}%
\pgfpathlineto{\pgfqpoint{0.000000in}{-0.083333in}}%
\pgfusepath{stroke,fill}%
}%
\begin{pgfscope}%
\pgfsys@transformshift{2.694118in}{0.386884in}%
\pgfsys@useobject{currentmarker}{}%
\end{pgfscope}%
\end{pgfscope}%
\begin{pgfscope}%
\definecolor{textcolor}{rgb}{0.150000,0.150000,0.150000}%
\pgfsetstrokecolor{textcolor}%
\pgfsetfillcolor{textcolor}%
\pgftext[x=2.694118in,y=0.206329in,,top]{\color{textcolor}\sffamily\fontsize{10.000000}{12.000000}\selectfont \(\displaystyle 0.6\)}%
\end{pgfscope}%
\begin{pgfscope}%
\pgfsetbuttcap%
\pgfsetroundjoin%
\definecolor{currentfill}{rgb}{0.150000,0.150000,0.150000}%
\pgfsetfillcolor{currentfill}%
\pgfsetlinewidth{1.003750pt}%
\definecolor{currentstroke}{rgb}{0.150000,0.150000,0.150000}%
\pgfsetstrokecolor{currentstroke}%
\pgfsetdash{}{0pt}%
\pgfsys@defobject{currentmarker}{\pgfqpoint{0.000000in}{-0.083333in}}{\pgfqpoint{0.000000in}{0.000000in}}{%
\pgfpathmoveto{\pgfqpoint{0.000000in}{0.000000in}}%
\pgfpathlineto{\pgfqpoint{0.000000in}{-0.083333in}}%
\pgfusepath{stroke,fill}%
}%
\begin{pgfscope}%
\pgfsys@transformshift{3.409385in}{0.386884in}%
\pgfsys@useobject{currentmarker}{}%
\end{pgfscope}%
\end{pgfscope}%
\begin{pgfscope}%
\definecolor{textcolor}{rgb}{0.150000,0.150000,0.150000}%
\pgfsetstrokecolor{textcolor}%
\pgfsetfillcolor{textcolor}%
\pgftext[x=3.409385in,y=0.206329in,,top]{\color{textcolor}\sffamily\fontsize{10.000000}{12.000000}\selectfont \(\displaystyle 0.8\)}%
\end{pgfscope}%
\begin{pgfscope}%
\pgfsetbuttcap%
\pgfsetroundjoin%
\definecolor{currentfill}{rgb}{0.150000,0.150000,0.150000}%
\pgfsetfillcolor{currentfill}%
\pgfsetlinewidth{1.003750pt}%
\definecolor{currentstroke}{rgb}{0.150000,0.150000,0.150000}%
\pgfsetstrokecolor{currentstroke}%
\pgfsetdash{}{0pt}%
\pgfsys@defobject{currentmarker}{\pgfqpoint{0.000000in}{-0.083333in}}{\pgfqpoint{0.000000in}{0.000000in}}{%
\pgfpathmoveto{\pgfqpoint{0.000000in}{0.000000in}}%
\pgfpathlineto{\pgfqpoint{0.000000in}{-0.083333in}}%
\pgfusepath{stroke,fill}%
}%
\begin{pgfscope}%
\pgfsys@transformshift{4.124652in}{0.386884in}%
\pgfsys@useobject{currentmarker}{}%
\end{pgfscope}%
\end{pgfscope}%
\begin{pgfscope}%
\definecolor{textcolor}{rgb}{0.150000,0.150000,0.150000}%
\pgfsetstrokecolor{textcolor}%
\pgfsetfillcolor{textcolor}%
\pgftext[x=4.124652in,y=0.206329in,,top]{\color{textcolor}\sffamily\fontsize{10.000000}{12.000000}\selectfont \(\displaystyle 1.0\)}%
\end{pgfscope}%
\begin{pgfscope}%
\pgfsetbuttcap%
\pgfsetroundjoin%
\definecolor{currentfill}{rgb}{0.150000,0.150000,0.150000}%
\pgfsetfillcolor{currentfill}%
\pgfsetlinewidth{1.003750pt}%
\definecolor{currentstroke}{rgb}{0.150000,0.150000,0.150000}%
\pgfsetstrokecolor{currentstroke}%
\pgfsetdash{}{0pt}%
\pgfsys@defobject{currentmarker}{\pgfqpoint{-0.083333in}{0.000000in}}{\pgfqpoint{0.000000in}{0.000000in}}{%
\pgfpathmoveto{\pgfqpoint{0.000000in}{0.000000in}}%
\pgfpathlineto{\pgfqpoint{-0.083333in}{0.000000in}}%
\pgfusepath{stroke,fill}%
}%
\begin{pgfscope}%
\pgfsys@transformshift{0.548317in}{0.386884in}%
\pgfsys@useobject{currentmarker}{}%
\end{pgfscope}%
\end{pgfscope}%
\begin{pgfscope}%
\definecolor{textcolor}{rgb}{0.150000,0.150000,0.150000}%
\pgfsetstrokecolor{textcolor}%
\pgfsetfillcolor{textcolor}%
\pgftext[x=0.082267in,y=0.336742in,left,base]{\color{textcolor}\sffamily\fontsize{10.000000}{12.000000}\selectfont \(\displaystyle -1.0\)}%
\end{pgfscope}%
\begin{pgfscope}%
\pgfsetbuttcap%
\pgfsetroundjoin%
\definecolor{currentfill}{rgb}{0.150000,0.150000,0.150000}%
\pgfsetfillcolor{currentfill}%
\pgfsetlinewidth{1.003750pt}%
\definecolor{currentstroke}{rgb}{0.150000,0.150000,0.150000}%
\pgfsetstrokecolor{currentstroke}%
\pgfsetdash{}{0pt}%
\pgfsys@defobject{currentmarker}{\pgfqpoint{-0.083333in}{0.000000in}}{\pgfqpoint{0.000000in}{0.000000in}}{%
\pgfpathmoveto{\pgfqpoint{0.000000in}{0.000000in}}%
\pgfpathlineto{\pgfqpoint{-0.083333in}{0.000000in}}%
\pgfusepath{stroke,fill}%
}%
\begin{pgfscope}%
\pgfsys@transformshift{0.548317in}{0.912325in}%
\pgfsys@useobject{currentmarker}{}%
\end{pgfscope}%
\end{pgfscope}%
\begin{pgfscope}%
\definecolor{textcolor}{rgb}{0.150000,0.150000,0.150000}%
\pgfsetstrokecolor{textcolor}%
\pgfsetfillcolor{textcolor}%
\pgftext[x=0.082267in,y=0.862183in,left,base]{\color{textcolor}\sffamily\fontsize{10.000000}{12.000000}\selectfont \(\displaystyle -0.5\)}%
\end{pgfscope}%
\begin{pgfscope}%
\pgfsetbuttcap%
\pgfsetroundjoin%
\definecolor{currentfill}{rgb}{0.150000,0.150000,0.150000}%
\pgfsetfillcolor{currentfill}%
\pgfsetlinewidth{1.003750pt}%
\definecolor{currentstroke}{rgb}{0.150000,0.150000,0.150000}%
\pgfsetstrokecolor{currentstroke}%
\pgfsetdash{}{0pt}%
\pgfsys@defobject{currentmarker}{\pgfqpoint{-0.083333in}{0.000000in}}{\pgfqpoint{0.000000in}{0.000000in}}{%
\pgfpathmoveto{\pgfqpoint{0.000000in}{0.000000in}}%
\pgfpathlineto{\pgfqpoint{-0.083333in}{0.000000in}}%
\pgfusepath{stroke,fill}%
}%
\begin{pgfscope}%
\pgfsys@transformshift{0.548317in}{1.437766in}%
\pgfsys@useobject{currentmarker}{}%
\end{pgfscope}%
\end{pgfscope}%
\begin{pgfscope}%
\definecolor{textcolor}{rgb}{0.150000,0.150000,0.150000}%
\pgfsetstrokecolor{textcolor}%
\pgfsetfillcolor{textcolor}%
\pgftext[x=0.190292in,y=1.387624in,left,base]{\color{textcolor}\sffamily\fontsize{10.000000}{12.000000}\selectfont \(\displaystyle 0.0\)}%
\end{pgfscope}%
\begin{pgfscope}%
\pgfsetbuttcap%
\pgfsetroundjoin%
\definecolor{currentfill}{rgb}{0.150000,0.150000,0.150000}%
\pgfsetfillcolor{currentfill}%
\pgfsetlinewidth{1.003750pt}%
\definecolor{currentstroke}{rgb}{0.150000,0.150000,0.150000}%
\pgfsetstrokecolor{currentstroke}%
\pgfsetdash{}{0pt}%
\pgfsys@defobject{currentmarker}{\pgfqpoint{-0.083333in}{0.000000in}}{\pgfqpoint{0.000000in}{0.000000in}}{%
\pgfpathmoveto{\pgfqpoint{0.000000in}{0.000000in}}%
\pgfpathlineto{\pgfqpoint{-0.083333in}{0.000000in}}%
\pgfusepath{stroke,fill}%
}%
\begin{pgfscope}%
\pgfsys@transformshift{0.548317in}{1.963207in}%
\pgfsys@useobject{currentmarker}{}%
\end{pgfscope}%
\end{pgfscope}%
\begin{pgfscope}%
\definecolor{textcolor}{rgb}{0.150000,0.150000,0.150000}%
\pgfsetstrokecolor{textcolor}%
\pgfsetfillcolor{textcolor}%
\pgftext[x=0.190292in,y=1.913065in,left,base]{\color{textcolor}\sffamily\fontsize{10.000000}{12.000000}\selectfont \(\displaystyle 0.5\)}%
\end{pgfscope}%
\begin{pgfscope}%
\pgfsetbuttcap%
\pgfsetroundjoin%
\definecolor{currentfill}{rgb}{0.150000,0.150000,0.150000}%
\pgfsetfillcolor{currentfill}%
\pgfsetlinewidth{1.003750pt}%
\definecolor{currentstroke}{rgb}{0.150000,0.150000,0.150000}%
\pgfsetstrokecolor{currentstroke}%
\pgfsetdash{}{0pt}%
\pgfsys@defobject{currentmarker}{\pgfqpoint{-0.083333in}{0.000000in}}{\pgfqpoint{0.000000in}{0.000000in}}{%
\pgfpathmoveto{\pgfqpoint{0.000000in}{0.000000in}}%
\pgfpathlineto{\pgfqpoint{-0.083333in}{0.000000in}}%
\pgfusepath{stroke,fill}%
}%
\begin{pgfscope}%
\pgfsys@transformshift{0.548317in}{2.488647in}%
\pgfsys@useobject{currentmarker}{}%
\end{pgfscope}%
\end{pgfscope}%
\begin{pgfscope}%
\definecolor{textcolor}{rgb}{0.150000,0.150000,0.150000}%
\pgfsetstrokecolor{textcolor}%
\pgfsetfillcolor{textcolor}%
\pgftext[x=0.190292in,y=2.438505in,left,base]{\color{textcolor}\sffamily\fontsize{10.000000}{12.000000}\selectfont \(\displaystyle 1.0\)}%
\end{pgfscope}%
\begin{pgfscope}%
\pgfpathrectangle{\pgfqpoint{0.548317in}{0.386884in}}{\pgfqpoint{3.576335in}{2.101763in}} %
\pgfusepath{clip}%
\pgfsetbuttcap%
\pgfsetroundjoin%
\definecolor{currentfill}{rgb}{0.400000,0.760784,0.647059}%
\pgfsetfillcolor{currentfill}%
\pgfsetlinewidth{0.000000pt}%
\definecolor{currentstroke}{rgb}{0.400000,0.760784,0.647059}%
\pgfsetstrokecolor{currentstroke}%
\pgfsetdash{}{0pt}%
\pgfsys@defobject{currentmarker}{\pgfqpoint{-0.048611in}{-0.048611in}}{\pgfqpoint{0.048611in}{0.048611in}}{%
\pgfpathmoveto{\pgfqpoint{0.000000in}{-0.048611in}}%
\pgfpathcurveto{\pgfqpoint{0.012892in}{-0.048611in}}{\pgfqpoint{0.025257in}{-0.043489in}}{\pgfqpoint{0.034373in}{-0.034373in}}%
\pgfpathcurveto{\pgfqpoint{0.043489in}{-0.025257in}}{\pgfqpoint{0.048611in}{-0.012892in}}{\pgfqpoint{0.048611in}{0.000000in}}%
\pgfpathcurveto{\pgfqpoint{0.048611in}{0.012892in}}{\pgfqpoint{0.043489in}{0.025257in}}{\pgfqpoint{0.034373in}{0.034373in}}%
\pgfpathcurveto{\pgfqpoint{0.025257in}{0.043489in}}{\pgfqpoint{0.012892in}{0.048611in}}{\pgfqpoint{0.000000in}{0.048611in}}%
\pgfpathcurveto{\pgfqpoint{-0.012892in}{0.048611in}}{\pgfqpoint{-0.025257in}{0.043489in}}{\pgfqpoint{-0.034373in}{0.034373in}}%
\pgfpathcurveto{\pgfqpoint{-0.043489in}{0.025257in}}{\pgfqpoint{-0.048611in}{0.012892in}}{\pgfqpoint{-0.048611in}{0.000000in}}%
\pgfpathcurveto{\pgfqpoint{-0.048611in}{-0.012892in}}{\pgfqpoint{-0.043489in}{-0.025257in}}{\pgfqpoint{-0.034373in}{-0.034373in}}%
\pgfpathcurveto{\pgfqpoint{-0.025257in}{-0.043489in}}{\pgfqpoint{-0.012892in}{-0.048611in}}{\pgfqpoint{0.000000in}{-0.048611in}}%
\pgfpathclose%
\pgfusepath{fill}%
}%
\begin{pgfscope}%
\pgfsys@transformshift{2.070523in}{0.696509in}%
\pgfsys@useobject{currentmarker}{}%
\end{pgfscope}%
\begin{pgfscope}%
\pgfsys@transformshift{3.072668in}{1.190191in}%
\pgfsys@useobject{currentmarker}{}%
\end{pgfscope}%
\begin{pgfscope}%
\pgfsys@transformshift{1.924913in}{0.659782in}%
\pgfsys@useobject{currentmarker}{}%
\end{pgfscope}%
\begin{pgfscope}%
\pgfsys@transformshift{2.346466in}{0.784652in}%
\pgfsys@useobject{currentmarker}{}%
\end{pgfscope}%
\begin{pgfscope}%
\pgfsys@transformshift{1.785447in}{0.629405in}%
\pgfsys@useobject{currentmarker}{}%
\end{pgfscope}%
\begin{pgfscope}%
\pgfsys@transformshift{2.976879in}{1.118524in}%
\pgfsys@useobject{currentmarker}{}%
\end{pgfscope}%
\begin{pgfscope}%
\pgfsys@transformshift{1.535300in}{0.583092in}%
\pgfsys@useobject{currentmarker}{}%
\end{pgfscope}%
\begin{pgfscope}%
\pgfsys@transformshift{2.445080in}{0.823363in}%
\pgfsys@useobject{currentmarker}{}%
\end{pgfscope}%
\begin{pgfscope}%
\pgfsys@transformshift{1.459375in}{0.570316in}%
\pgfsys@useobject{currentmarker}{}%
\end{pgfscope}%
\begin{pgfscope}%
\pgfsys@transformshift{2.736514in}{0.965179in}%
\pgfsys@useobject{currentmarker}{}%
\end{pgfscope}%
\end{pgfscope}%
\begin{pgfscope}%
\pgfpathrectangle{\pgfqpoint{0.548317in}{0.386884in}}{\pgfqpoint{3.576335in}{2.101763in}} %
\pgfusepath{clip}%
\pgfsetbuttcap%
\pgfsetroundjoin%
\pgfsetlinewidth{1.756562pt}%
\definecolor{currentstroke}{rgb}{0.988235,0.552941,0.384314}%
\pgfsetstrokecolor{currentstroke}%
\pgfsetdash{{5.600000pt}{2.400000pt}}{0.000000pt}%
\pgfpathmoveto{\pgfqpoint{0.548317in}{0.386884in}}%
\pgfpathlineto{\pgfqpoint{0.584442in}{0.397288in}}%
\pgfpathlineto{\pgfqpoint{0.620566in}{0.407283in}}%
\pgfpathlineto{\pgfqpoint{0.656691in}{0.416887in}}%
\pgfpathlineto{\pgfqpoint{0.692815in}{0.426121in}}%
\pgfpathlineto{\pgfqpoint{0.728940in}{0.435004in}}%
\pgfpathlineto{\pgfqpoint{0.765065in}{0.443556in}}%
\pgfpathlineto{\pgfqpoint{0.801189in}{0.451796in}}%
\pgfpathlineto{\pgfqpoint{0.837314in}{0.459743in}}%
\pgfpathlineto{\pgfqpoint{0.873438in}{0.467418in}}%
\pgfpathlineto{\pgfqpoint{0.909563in}{0.474839in}}%
\pgfpathlineto{\pgfqpoint{0.945688in}{0.482026in}}%
\pgfpathlineto{\pgfqpoint{0.981812in}{0.488999in}}%
\pgfpathlineto{\pgfqpoint{1.017937in}{0.495776in}}%
\pgfpathlineto{\pgfqpoint{1.054061in}{0.502379in}}%
\pgfpathlineto{\pgfqpoint{1.090186in}{0.508825in}}%
\pgfpathlineto{\pgfqpoint{1.126311in}{0.515135in}}%
\pgfpathlineto{\pgfqpoint{1.162435in}{0.521328in}}%
\pgfpathlineto{\pgfqpoint{1.198560in}{0.527423in}}%
\pgfpathlineto{\pgfqpoint{1.234684in}{0.533440in}}%
\pgfpathlineto{\pgfqpoint{1.270809in}{0.539399in}}%
\pgfpathlineto{\pgfqpoint{1.306933in}{0.545319in}}%
\pgfpathlineto{\pgfqpoint{1.343058in}{0.551220in}}%
\pgfpathlineto{\pgfqpoint{1.379183in}{0.557120in}}%
\pgfpathlineto{\pgfqpoint{1.415307in}{0.563040in}}%
\pgfpathlineto{\pgfqpoint{1.451432in}{0.568999in}}%
\pgfpathlineto{\pgfqpoint{1.487556in}{0.575017in}}%
\pgfpathlineto{\pgfqpoint{1.523681in}{0.581112in}}%
\pgfpathlineto{\pgfqpoint{1.559806in}{0.587305in}}%
\pgfpathlineto{\pgfqpoint{1.595930in}{0.593615in}}%
\pgfpathlineto{\pgfqpoint{1.632055in}{0.600061in}}%
\pgfpathlineto{\pgfqpoint{1.668179in}{0.606663in}}%
\pgfpathlineto{\pgfqpoint{1.704304in}{0.613441in}}%
\pgfpathlineto{\pgfqpoint{1.740429in}{0.620414in}}%
\pgfpathlineto{\pgfqpoint{1.776553in}{0.627601in}}%
\pgfpathlineto{\pgfqpoint{1.812678in}{0.635022in}}%
\pgfpathlineto{\pgfqpoint{1.848802in}{0.642696in}}%
\pgfpathlineto{\pgfqpoint{1.884927in}{0.650644in}}%
\pgfpathlineto{\pgfqpoint{1.921052in}{0.658884in}}%
\pgfpathlineto{\pgfqpoint{1.957176in}{0.667435in}}%
\pgfpathlineto{\pgfqpoint{1.993301in}{0.676318in}}%
\pgfpathlineto{\pgfqpoint{2.029425in}{0.685553in}}%
\pgfpathlineto{\pgfqpoint{2.065550in}{0.695157in}}%
\pgfpathlineto{\pgfqpoint{2.101675in}{0.705151in}}%
\pgfpathlineto{\pgfqpoint{2.137799in}{0.715555in}}%
\pgfpathlineto{\pgfqpoint{2.173924in}{0.726388in}}%
\pgfpathlineto{\pgfqpoint{2.210048in}{0.737669in}}%
\pgfpathlineto{\pgfqpoint{2.246173in}{0.749418in}}%
\pgfpathlineto{\pgfqpoint{2.282298in}{0.761654in}}%
\pgfpathlineto{\pgfqpoint{2.318422in}{0.774397in}}%
\pgfpathlineto{\pgfqpoint{2.354547in}{0.787667in}}%
\pgfpathlineto{\pgfqpoint{2.390671in}{0.801482in}}%
\pgfpathlineto{\pgfqpoint{2.426796in}{0.815863in}}%
\pgfpathlineto{\pgfqpoint{2.462921in}{0.830828in}}%
\pgfpathlineto{\pgfqpoint{2.499045in}{0.846398in}}%
\pgfpathlineto{\pgfqpoint{2.535170in}{0.862592in}}%
\pgfpathlineto{\pgfqpoint{2.571294in}{0.879429in}}%
\pgfpathlineto{\pgfqpoint{2.607419in}{0.896929in}}%
\pgfpathlineto{\pgfqpoint{2.643543in}{0.915111in}}%
\pgfpathlineto{\pgfqpoint{2.679668in}{0.933995in}}%
\pgfpathlineto{\pgfqpoint{2.715793in}{0.953601in}}%
\pgfpathlineto{\pgfqpoint{2.751917in}{0.973947in}}%
\pgfpathlineto{\pgfqpoint{2.788042in}{0.995053in}}%
\pgfpathlineto{\pgfqpoint{2.824166in}{1.016940in}}%
\pgfpathlineto{\pgfqpoint{2.860291in}{1.039625in}}%
\pgfpathlineto{\pgfqpoint{2.896416in}{1.063129in}}%
\pgfpathlineto{\pgfqpoint{2.932540in}{1.087472in}}%
\pgfpathlineto{\pgfqpoint{2.968665in}{1.112672in}}%
\pgfpathlineto{\pgfqpoint{3.004789in}{1.138750in}}%
\pgfpathlineto{\pgfqpoint{3.040914in}{1.165725in}}%
\pgfpathlineto{\pgfqpoint{3.077039in}{1.193615in}}%
\pgfpathlineto{\pgfqpoint{3.113163in}{1.222442in}}%
\pgfpathlineto{\pgfqpoint{3.149288in}{1.252223in}}%
\pgfpathlineto{\pgfqpoint{3.185412in}{1.282980in}}%
\pgfpathlineto{\pgfqpoint{3.221537in}{1.314730in}}%
\pgfpathlineto{\pgfqpoint{3.257662in}{1.347495in}}%
\pgfpathlineto{\pgfqpoint{3.293786in}{1.381293in}}%
\pgfpathlineto{\pgfqpoint{3.329911in}{1.416143in}}%
\pgfpathlineto{\pgfqpoint{3.366035in}{1.452065in}}%
\pgfpathlineto{\pgfqpoint{3.402160in}{1.489080in}}%
\pgfpathlineto{\pgfqpoint{3.438285in}{1.527205in}}%
\pgfpathlineto{\pgfqpoint{3.474409in}{1.566461in}}%
\pgfpathlineto{\pgfqpoint{3.510534in}{1.606868in}}%
\pgfpathlineto{\pgfqpoint{3.546658in}{1.648444in}}%
\pgfpathlineto{\pgfqpoint{3.582783in}{1.691209in}}%
\pgfpathlineto{\pgfqpoint{3.618908in}{1.735183in}}%
\pgfpathlineto{\pgfqpoint{3.655032in}{1.780385in}}%
\pgfpathlineto{\pgfqpoint{3.691157in}{1.826835in}}%
\pgfpathlineto{\pgfqpoint{3.727281in}{1.874552in}}%
\pgfpathlineto{\pgfqpoint{3.763406in}{1.923556in}}%
\pgfpathlineto{\pgfqpoint{3.799531in}{1.973866in}}%
\pgfpathlineto{\pgfqpoint{3.835655in}{2.025501in}}%
\pgfpathlineto{\pgfqpoint{3.871780in}{2.078482in}}%
\pgfpathlineto{\pgfqpoint{3.907904in}{2.132827in}}%
\pgfpathlineto{\pgfqpoint{3.944029in}{2.188556in}}%
\pgfpathlineto{\pgfqpoint{3.980154in}{2.245689in}}%
\pgfpathlineto{\pgfqpoint{4.016278in}{2.304245in}}%
\pgfpathlineto{\pgfqpoint{4.052403in}{2.364244in}}%
\pgfpathlineto{\pgfqpoint{4.088527in}{2.425705in}}%
\pgfpathlineto{\pgfqpoint{4.124652in}{2.488647in}}%
\pgfusepath{stroke}%
\end{pgfscope}%
\begin{pgfscope}%
\pgfpathrectangle{\pgfqpoint{0.548317in}{0.386884in}}{\pgfqpoint{3.576335in}{2.101763in}} %
\pgfusepath{clip}%
\pgfsetroundcap%
\pgfsetroundjoin%
\pgfsetlinewidth{1.756562pt}%
\definecolor{currentstroke}{rgb}{0.552941,0.627451,0.796078}%
\pgfsetstrokecolor{currentstroke}%
\pgfsetdash{}{0pt}%
\pgfpathmoveto{\pgfqpoint{0.548317in}{0.391060in}}%
\pgfpathlineto{\pgfqpoint{0.584442in}{0.400779in}}%
\pgfpathlineto{\pgfqpoint{0.620566in}{0.410181in}}%
\pgfpathlineto{\pgfqpoint{0.656691in}{0.419277in}}%
\pgfpathlineto{\pgfqpoint{0.692815in}{0.428077in}}%
\pgfpathlineto{\pgfqpoint{0.728940in}{0.436592in}}%
\pgfpathlineto{\pgfqpoint{0.765065in}{0.444834in}}%
\pgfpathlineto{\pgfqpoint{0.801189in}{0.452814in}}%
\pgfpathlineto{\pgfqpoint{0.837314in}{0.460547in}}%
\pgfpathlineto{\pgfqpoint{0.873438in}{0.468044in}}%
\pgfpathlineto{\pgfqpoint{0.909563in}{0.475322in}}%
\pgfpathlineto{\pgfqpoint{0.945688in}{0.482393in}}%
\pgfpathlineto{\pgfqpoint{0.981812in}{0.489274in}}%
\pgfpathlineto{\pgfqpoint{1.017937in}{0.495979in}}%
\pgfpathlineto{\pgfqpoint{1.054061in}{0.502525in}}%
\pgfpathlineto{\pgfqpoint{1.090186in}{0.508928in}}%
\pgfpathlineto{\pgfqpoint{1.126311in}{0.515206in}}%
\pgfpathlineto{\pgfqpoint{1.162435in}{0.521376in}}%
\pgfpathlineto{\pgfqpoint{1.198560in}{0.527454in}}%
\pgfpathlineto{\pgfqpoint{1.234684in}{0.533460in}}%
\pgfpathlineto{\pgfqpoint{1.270809in}{0.539411in}}%
\pgfpathlineto{\pgfqpoint{1.306933in}{0.545326in}}%
\pgfpathlineto{\pgfqpoint{1.343058in}{0.551223in}}%
\pgfpathlineto{\pgfqpoint{1.379183in}{0.557122in}}%
\pgfpathlineto{\pgfqpoint{1.415307in}{0.563041in}}%
\pgfpathlineto{\pgfqpoint{1.451432in}{0.568999in}}%
\pgfpathlineto{\pgfqpoint{1.487556in}{0.575016in}}%
\pgfpathlineto{\pgfqpoint{1.523681in}{0.581112in}}%
\pgfpathlineto{\pgfqpoint{1.559806in}{0.587305in}}%
\pgfpathlineto{\pgfqpoint{1.595930in}{0.593615in}}%
\pgfpathlineto{\pgfqpoint{1.632055in}{0.600061in}}%
\pgfpathlineto{\pgfqpoint{1.668179in}{0.606663in}}%
\pgfpathlineto{\pgfqpoint{1.704304in}{0.613441in}}%
\pgfpathlineto{\pgfqpoint{1.740429in}{0.620414in}}%
\pgfpathlineto{\pgfqpoint{1.776553in}{0.627601in}}%
\pgfpathlineto{\pgfqpoint{1.812678in}{0.635022in}}%
\pgfpathlineto{\pgfqpoint{1.848802in}{0.642696in}}%
\pgfpathlineto{\pgfqpoint{1.884927in}{0.650644in}}%
\pgfpathlineto{\pgfqpoint{1.921052in}{0.658884in}}%
\pgfpathlineto{\pgfqpoint{1.957176in}{0.667435in}}%
\pgfpathlineto{\pgfqpoint{1.993301in}{0.676318in}}%
\pgfpathlineto{\pgfqpoint{2.029425in}{0.685553in}}%
\pgfpathlineto{\pgfqpoint{2.065550in}{0.695157in}}%
\pgfpathlineto{\pgfqpoint{2.101675in}{0.705151in}}%
\pgfpathlineto{\pgfqpoint{2.137799in}{0.715555in}}%
\pgfpathlineto{\pgfqpoint{2.173924in}{0.726388in}}%
\pgfpathlineto{\pgfqpoint{2.210048in}{0.737669in}}%
\pgfpathlineto{\pgfqpoint{2.246173in}{0.749418in}}%
\pgfpathlineto{\pgfqpoint{2.282298in}{0.761654in}}%
\pgfpathlineto{\pgfqpoint{2.318422in}{0.774397in}}%
\pgfpathlineto{\pgfqpoint{2.354547in}{0.787667in}}%
\pgfpathlineto{\pgfqpoint{2.390671in}{0.801482in}}%
\pgfpathlineto{\pgfqpoint{2.426796in}{0.815863in}}%
\pgfpathlineto{\pgfqpoint{2.462921in}{0.830828in}}%
\pgfpathlineto{\pgfqpoint{2.499045in}{0.846398in}}%
\pgfpathlineto{\pgfqpoint{2.535170in}{0.862592in}}%
\pgfpathlineto{\pgfqpoint{2.571294in}{0.879429in}}%
\pgfpathlineto{\pgfqpoint{2.607419in}{0.896928in}}%
\pgfpathlineto{\pgfqpoint{2.643543in}{0.915111in}}%
\pgfpathlineto{\pgfqpoint{2.679668in}{0.933995in}}%
\pgfpathlineto{\pgfqpoint{2.715793in}{0.953600in}}%
\pgfpathlineto{\pgfqpoint{2.751917in}{0.973947in}}%
\pgfpathlineto{\pgfqpoint{2.788042in}{0.995055in}}%
\pgfpathlineto{\pgfqpoint{2.824166in}{1.016942in}}%
\pgfpathlineto{\pgfqpoint{2.860291in}{1.039628in}}%
\pgfpathlineto{\pgfqpoint{2.896416in}{1.063133in}}%
\pgfpathlineto{\pgfqpoint{2.932540in}{1.087475in}}%
\pgfpathlineto{\pgfqpoint{2.968665in}{1.112673in}}%
\pgfpathlineto{\pgfqpoint{3.004789in}{1.138748in}}%
\pgfpathlineto{\pgfqpoint{3.040914in}{1.165721in}}%
\pgfpathlineto{\pgfqpoint{3.077039in}{1.193616in}}%
\pgfpathlineto{\pgfqpoint{3.113163in}{1.222465in}}%
\pgfpathlineto{\pgfqpoint{3.149288in}{1.252307in}}%
\pgfpathlineto{\pgfqpoint{3.185412in}{1.283195in}}%
\pgfpathlineto{\pgfqpoint{3.221537in}{1.315210in}}%
\pgfpathlineto{\pgfqpoint{3.257662in}{1.348463in}}%
\pgfpathlineto{\pgfqpoint{3.293786in}{1.383121in}}%
\pgfpathlineto{\pgfqpoint{3.329911in}{1.419427in}}%
\pgfpathlineto{\pgfqpoint{3.366035in}{1.457730in}}%
\pgfpathlineto{\pgfqpoint{3.402160in}{1.498534in}}%
\pgfpathlineto{\pgfqpoint{3.438285in}{1.542553in}}%
\pgfpathlineto{\pgfqpoint{3.474409in}{1.590795in}}%
\pgfpathlineto{\pgfqpoint{3.510534in}{1.644658in}}%
\pgfpathlineto{\pgfqpoint{3.546658in}{1.706077in}}%
\pgfpathlineto{\pgfqpoint{3.582783in}{1.777690in}}%
\pgfpathlineto{\pgfqpoint{3.618908in}{1.863075in}}%
\pgfpathlineto{\pgfqpoint{3.655032in}{1.967039in}}%
\pgfpathlineto{\pgfqpoint{3.691157in}{2.095992in}}%
\pgfpathlineto{\pgfqpoint{3.727281in}{2.258420in}}%
\pgfpathlineto{\pgfqpoint{3.763406in}{2.465484in}}%
\pgfpathlineto{\pgfqpoint{3.768433in}{2.502536in}}%
\pgfusepath{stroke}%
\end{pgfscope}%
\begin{pgfscope}%
\pgfsetrectcap%
\pgfsetmiterjoin%
\pgfsetlinewidth{1.254687pt}%
\definecolor{currentstroke}{rgb}{0.150000,0.150000,0.150000}%
\pgfsetstrokecolor{currentstroke}%
\pgfsetdash{}{0pt}%
\pgfpathmoveto{\pgfqpoint{0.548317in}{0.386884in}}%
\pgfpathlineto{\pgfqpoint{0.548317in}{2.488647in}}%
\pgfusepath{stroke}%
\end{pgfscope}%
\begin{pgfscope}%
\pgfsetrectcap%
\pgfsetmiterjoin%
\pgfsetlinewidth{1.254687pt}%
\definecolor{currentstroke}{rgb}{0.150000,0.150000,0.150000}%
\pgfsetstrokecolor{currentstroke}%
\pgfsetdash{}{0pt}%
\pgfpathmoveto{\pgfqpoint{0.548317in}{0.386884in}}%
\pgfpathlineto{\pgfqpoint{4.124652in}{0.386884in}}%
\pgfusepath{stroke}%
\end{pgfscope}%
\begin{pgfscope}%
\pgfsetbuttcap%
\pgfsetroundjoin%
\definecolor{currentfill}{rgb}{0.400000,0.760784,0.647059}%
\pgfsetfillcolor{currentfill}%
\pgfsetlinewidth{0.000000pt}%
\definecolor{currentstroke}{rgb}{0.400000,0.760784,0.647059}%
\pgfsetstrokecolor{currentstroke}%
\pgfsetdash{}{0pt}%
\pgfsys@defobject{currentmarker}{\pgfqpoint{-0.048611in}{-0.048611in}}{\pgfqpoint{0.048611in}{0.048611in}}{%
\pgfpathmoveto{\pgfqpoint{0.000000in}{-0.048611in}}%
\pgfpathcurveto{\pgfqpoint{0.012892in}{-0.048611in}}{\pgfqpoint{0.025257in}{-0.043489in}}{\pgfqpoint{0.034373in}{-0.034373in}}%
\pgfpathcurveto{\pgfqpoint{0.043489in}{-0.025257in}}{\pgfqpoint{0.048611in}{-0.012892in}}{\pgfqpoint{0.048611in}{0.000000in}}%
\pgfpathcurveto{\pgfqpoint{0.048611in}{0.012892in}}{\pgfqpoint{0.043489in}{0.025257in}}{\pgfqpoint{0.034373in}{0.034373in}}%
\pgfpathcurveto{\pgfqpoint{0.025257in}{0.043489in}}{\pgfqpoint{0.012892in}{0.048611in}}{\pgfqpoint{0.000000in}{0.048611in}}%
\pgfpathcurveto{\pgfqpoint{-0.012892in}{0.048611in}}{\pgfqpoint{-0.025257in}{0.043489in}}{\pgfqpoint{-0.034373in}{0.034373in}}%
\pgfpathcurveto{\pgfqpoint{-0.043489in}{0.025257in}}{\pgfqpoint{-0.048611in}{0.012892in}}{\pgfqpoint{-0.048611in}{0.000000in}}%
\pgfpathcurveto{\pgfqpoint{-0.048611in}{-0.012892in}}{\pgfqpoint{-0.043489in}{-0.025257in}}{\pgfqpoint{-0.034373in}{-0.034373in}}%
\pgfpathcurveto{\pgfqpoint{-0.025257in}{-0.043489in}}{\pgfqpoint{-0.012892in}{-0.048611in}}{\pgfqpoint{0.000000in}{-0.048611in}}%
\pgfpathclose%
\pgfusepath{fill}%
}%
\begin{pgfscope}%
\pgfsys@transformshift{0.812206in}{2.311974in}%
\pgfsys@useobject{currentmarker}{}%
\end{pgfscope}%
\end{pgfscope}%
\begin{pgfscope}%
\definecolor{textcolor}{rgb}{0.150000,0.150000,0.150000}%
\pgfsetstrokecolor{textcolor}%
\pgfsetfillcolor{textcolor}%
\pgftext[x=1.062206in,y=2.263363in,left,base]{\color{textcolor}\sffamily\fontsize{10.000000}{12.000000}\selectfont samples}%
\end{pgfscope}%
\begin{pgfscope}%
\pgfsetbuttcap%
\pgfsetroundjoin%
\pgfsetlinewidth{1.756562pt}%
\definecolor{currentstroke}{rgb}{0.988235,0.552941,0.384314}%
\pgfsetstrokecolor{currentstroke}%
\pgfsetdash{{5.600000pt}{2.400000pt}}{0.000000pt}%
\pgfpathmoveto{\pgfqpoint{0.673317in}{2.115246in}}%
\pgfpathlineto{\pgfqpoint{0.951095in}{2.115246in}}%
\pgfusepath{stroke}%
\end{pgfscope}%
\begin{pgfscope}%
\definecolor{textcolor}{rgb}{0.150000,0.150000,0.150000}%
\pgfsetstrokecolor{textcolor}%
\pgfsetfillcolor{textcolor}%
\pgftext[x=1.062206in,y=2.066635in,left,base]{\color{textcolor}\sffamily\fontsize{10.000000}{12.000000}\selectfont function}%
\end{pgfscope}%
\begin{pgfscope}%
\pgfsetroundcap%
\pgfsetroundjoin%
\pgfsetlinewidth{1.756562pt}%
\definecolor{currentstroke}{rgb}{0.552941,0.627451,0.796078}%
\pgfsetstrokecolor{currentstroke}%
\pgfsetdash{}{0pt}%
\pgfpathmoveto{\pgfqpoint{0.673317in}{1.918518in}}%
\pgfpathlineto{\pgfqpoint{0.951095in}{1.918518in}}%
\pgfusepath{stroke}%
\end{pgfscope}%
\begin{pgfscope}%
\definecolor{textcolor}{rgb}{0.150000,0.150000,0.150000}%
\pgfsetstrokecolor{textcolor}%
\pgfsetfillcolor{textcolor}%
\pgftext[x=1.062206in,y=1.869907in,left,base]{\color{textcolor}\sffamily\fontsize{10.000000}{12.000000}\selectfont 20th order fit}%
\end{pgfscope}%
\end{pgfpicture}%
\makeatother%
\endgroup%
}
			\caption{\engordnumber{20}-order polynomial fitting 10 points}
			\label{fig:polyfit20thlots}
		\end{subfigure}
		\caption[Polynomial Data Fitting]{Polynomial fits of samples from a 3rd order function. Polynomials of high order, like neural networks of many parameters, easily overfit a small number of samples as compared to polynomials of a more suitable order for the sampled function. While generalization is helped by more data, the higher order polynomial still tends to overfit.}
		\label{fig:polyfits}
	\end{figure}

	\citet{denker1987large} explored the relationship of network architecture to generalization. The work was particularly motivating in the later design of convolutional neural networks~\citep{lecun1989generalization, lecun1989backpropagation}. The authors make the intuitive analogy between the affect of the size of a neural network on its generalization, and a simple least-squares polynomial fit. Fig.~\ref{fig:polyfits} shows various polynomial fits to samples from a \engordnumber{3}-order polynomial function. When using a \engordnumber{3}-order polynomial to fit even a small number of samples (Fig.~\ref{fig:polyfit3rd}), the fit extrapolates, \ie is closer to the desired function outside the range of training samples, better than when we use a \engordnumber{20}-order polynomial to fit the same data (Fig.~\ref{fig:polyfit20th}). While the number of samples can help the fit of the higher order function, even with a large number of samples the \engordnumber{20}-order polynomial fit  (Fig.~\ref{fig:polyfit20thlots}) will not extrapolate as well as the polynomial with a more appropriate lower number of parameters (Fig.~\ref{fig:polyfit3rdlots}). Similarly, a neural network with a large number of parameters may not generalize as well as a neural network with fewer more salient parameters.
    
    \citet{caruana2001overfitting} further explored the analogy by training neural networks to fit polynomials, showing that overfitting in neural networks does not seem to be as serious a problem as in polynomials. The greatly over-parameterized neural networks still found relatively good fits. The authors suggest that neural networks trained with backpropogration may be biased towards ``smoother approximations''.
    
    \section{Deep Networks}
	% deep boltzmann networks
	\citep{Krizhevsky2012}
	\citep{Simonyan2014verydeep}
	\citep{He2015}
	\citep{He2016}
    
	
	%% NOTE: Polynomial fits are a special case of linear regression where we've used a polynomial basis
	%% Use example of sin wave? We can fit it with a polynomial, but better to reparameterize into a basis
	
	\subsection{Generalization and Parameters in Neural Networks}
	\begin{figure}[tb]
		\centering
		\large
        \renewcommand{\ttdefault}{pcr}
		\begin{subfigure}[t]{0.45\textwidth}
			\begin{center}
			\texttt{00\textbf{11111}00}\\
			\texttt{00\textbf{111}0000}\\
			\texttt{0000\textbf{1}0000}
			\end{center}
			\caption{Binary sequences with one clump}
			\label{fig:oneclump}
		\end{subfigure}
		~
		\begin{subfigure}[t]{0.45\textwidth}
			\begin{center}
			\texttt{00\textbf{11}0\textbf{11}00}\\
			\texttt{00\textbf{1}0\textbf{1}0\textbf{1}00}\\
			\texttt{00\textbf{111}0\textbf{1}00}
			\end{center}
			\caption{Binary sequences with two or more clumps}
			\label{fig:polyfit20th}
		\end{subfigure}
		
        \renewcommand{\ttdefault}{lmodern}
		\caption[Two-or-more Clump Predicate]{The two-or-more clumps predicate asks for the network to classify (padded) binary input sequences as having one or two or more contiguous strings of ones.}
		\label{fig:tomclumps}
	\end{figure}
	
	\citet{denker1987large,giles1987learning} explore the relationship between network architecture and generalization by evaluating networks for solving the \emph{two-or-more clumps} predicate. The two-or-more clumps predicate asks for the network to classify binary input sequences as having one or two or more contiguous strings of ones, some examples of which are shown in Fig.~\ref{fig:tomclumps}. 
	
	The authors illustrate some surprising properties of the generalization of fully-connected neural networks learned with backpropogation. First, a human-preferred `geometric' solution is manually hard-coded into the weights of a fully-connected network. While this weight configuration is a valid solution, and is intuitive to humans, the authors show that is is not a solution that the network would ever settle upon when trained with backpropogation. By using the geometric solution as an initialization, and training the network further, the authors show that the error-surface around the region is not stable,
	
	\begin{figure}[tb]
		\centering
		\begin{subfigure}[t]{0.49\textwidth}
			\resizebox{\linewidth}{!}{%% Creator: Matplotlib, PGF backend
%%
%% To include the figure in your LaTeX document, write
%%   \input{<filename>.pgf}
%%
%% Make sure the required packages are loaded in your preamble
%%   \usepackage{pgf}
%%
%% Figures using additional raster images can only be included by \input if
%% they are in the same directory as the main LaTeX file. For loading figures
%% from other directories you can use the `import` package
%%   \usepackage{import}
%% and then include the figures with
%%   \import{<path to file>}{<filename>.pgf}
%%
%% Matplotlib used the following preamble
%%   \usepackage[utf8x]{inputenc}
%%   \usepackage[T1]{fontenc}
%%
\begingroup%
\makeatletter%
\begin{pgfpicture}%
\pgfpathrectangle{\pgfpointorigin}{\pgfqpoint{4.296389in}{2.655314in}}%
\pgfusepath{use as bounding box, clip}%
\begin{pgfscope}%
\pgfsetbuttcap%
\pgfsetmiterjoin%
\definecolor{currentfill}{rgb}{1.000000,1.000000,1.000000}%
\pgfsetfillcolor{currentfill}%
\pgfsetlinewidth{0.000000pt}%
\definecolor{currentstroke}{rgb}{1.000000,1.000000,1.000000}%
\pgfsetstrokecolor{currentstroke}%
\pgfsetdash{}{0pt}%
\pgfpathmoveto{\pgfqpoint{0.000000in}{0.000000in}}%
\pgfpathlineto{\pgfqpoint{4.296389in}{0.000000in}}%
\pgfpathlineto{\pgfqpoint{4.296389in}{2.655314in}}%
\pgfpathlineto{\pgfqpoint{0.000000in}{2.655314in}}%
\pgfpathclose%
\pgfusepath{fill}%
\end{pgfscope}%
\begin{pgfscope}%
\pgfsetbuttcap%
\pgfsetmiterjoin%
\definecolor{currentfill}{rgb}{1.000000,1.000000,1.000000}%
\pgfsetfillcolor{currentfill}%
\pgfsetlinewidth{0.000000pt}%
\definecolor{currentstroke}{rgb}{0.000000,0.000000,0.000000}%
\pgfsetstrokecolor{currentstroke}%
\pgfsetstrokeopacity{0.000000}%
\pgfsetdash{}{0pt}%
\pgfpathmoveto{\pgfqpoint{0.734747in}{0.594647in}}%
\pgfpathlineto{\pgfqpoint{4.129722in}{0.594647in}}%
\pgfpathlineto{\pgfqpoint{4.129722in}{2.488647in}}%
\pgfpathlineto{\pgfqpoint{0.734747in}{2.488647in}}%
\pgfpathclose%
\pgfusepath{fill}%
\end{pgfscope}%
\begin{pgfscope}%
\pgfsetbuttcap%
\pgfsetroundjoin%
\definecolor{currentfill}{rgb}{0.150000,0.150000,0.150000}%
\pgfsetfillcolor{currentfill}%
\pgfsetlinewidth{1.003750pt}%
\definecolor{currentstroke}{rgb}{0.150000,0.150000,0.150000}%
\pgfsetstrokecolor{currentstroke}%
\pgfsetdash{}{0pt}%
\pgfsys@defobject{currentmarker}{\pgfqpoint{0.000000in}{-0.083333in}}{\pgfqpoint{0.000000in}{0.000000in}}{%
\pgfpathmoveto{\pgfqpoint{0.000000in}{0.000000in}}%
\pgfpathlineto{\pgfqpoint{0.000000in}{-0.083333in}}%
\pgfusepath{stroke,fill}%
}%
\begin{pgfscope}%
\pgfsys@transformshift{0.888721in}{0.594647in}%
\pgfsys@useobject{currentmarker}{}%
\end{pgfscope}%
\end{pgfscope}%
\begin{pgfscope}%
\definecolor{textcolor}{rgb}{0.150000,0.150000,0.150000}%
\pgfsetstrokecolor{textcolor}%
\pgfsetfillcolor{textcolor}%
\pgftext[x=0.888721in,y=0.414091in,,top]{\color{textcolor}\sffamily\fontsize{10.000000}{12.000000}\selectfont \(\displaystyle 0\)}%
\end{pgfscope}%
\begin{pgfscope}%
\pgfsetbuttcap%
\pgfsetroundjoin%
\definecolor{currentfill}{rgb}{0.150000,0.150000,0.150000}%
\pgfsetfillcolor{currentfill}%
\pgfsetlinewidth{1.003750pt}%
\definecolor{currentstroke}{rgb}{0.150000,0.150000,0.150000}%
\pgfsetstrokecolor{currentstroke}%
\pgfsetdash{}{0pt}%
\pgfsys@defobject{currentmarker}{\pgfqpoint{0.000000in}{-0.083333in}}{\pgfqpoint{0.000000in}{0.000000in}}{%
\pgfpathmoveto{\pgfqpoint{0.000000in}{0.000000in}}%
\pgfpathlineto{\pgfqpoint{0.000000in}{-0.083333in}}%
\pgfusepath{stroke,fill}%
}%
\begin{pgfscope}%
\pgfsys@transformshift{1.574651in}{0.594647in}%
\pgfsys@useobject{currentmarker}{}%
\end{pgfscope}%
\end{pgfscope}%
\begin{pgfscope}%
\definecolor{textcolor}{rgb}{0.150000,0.150000,0.150000}%
\pgfsetstrokecolor{textcolor}%
\pgfsetfillcolor{textcolor}%
\pgftext[x=1.574651in,y=0.414091in,,top]{\color{textcolor}\sffamily\fontsize{10.000000}{12.000000}\selectfont \(\displaystyle 2000\)}%
\end{pgfscope}%
\begin{pgfscope}%
\pgfsetbuttcap%
\pgfsetroundjoin%
\definecolor{currentfill}{rgb}{0.150000,0.150000,0.150000}%
\pgfsetfillcolor{currentfill}%
\pgfsetlinewidth{1.003750pt}%
\definecolor{currentstroke}{rgb}{0.150000,0.150000,0.150000}%
\pgfsetstrokecolor{currentstroke}%
\pgfsetdash{}{0pt}%
\pgfsys@defobject{currentmarker}{\pgfqpoint{0.000000in}{-0.083333in}}{\pgfqpoint{0.000000in}{0.000000in}}{%
\pgfpathmoveto{\pgfqpoint{0.000000in}{0.000000in}}%
\pgfpathlineto{\pgfqpoint{0.000000in}{-0.083333in}}%
\pgfusepath{stroke,fill}%
}%
\begin{pgfscope}%
\pgfsys@transformshift{2.260581in}{0.594647in}%
\pgfsys@useobject{currentmarker}{}%
\end{pgfscope}%
\end{pgfscope}%
\begin{pgfscope}%
\definecolor{textcolor}{rgb}{0.150000,0.150000,0.150000}%
\pgfsetstrokecolor{textcolor}%
\pgfsetfillcolor{textcolor}%
\pgftext[x=2.260581in,y=0.414091in,,top]{\color{textcolor}\sffamily\fontsize{10.000000}{12.000000}\selectfont \(\displaystyle 4000\)}%
\end{pgfscope}%
\begin{pgfscope}%
\pgfsetbuttcap%
\pgfsetroundjoin%
\definecolor{currentfill}{rgb}{0.150000,0.150000,0.150000}%
\pgfsetfillcolor{currentfill}%
\pgfsetlinewidth{1.003750pt}%
\definecolor{currentstroke}{rgb}{0.150000,0.150000,0.150000}%
\pgfsetstrokecolor{currentstroke}%
\pgfsetdash{}{0pt}%
\pgfsys@defobject{currentmarker}{\pgfqpoint{0.000000in}{-0.083333in}}{\pgfqpoint{0.000000in}{0.000000in}}{%
\pgfpathmoveto{\pgfqpoint{0.000000in}{0.000000in}}%
\pgfpathlineto{\pgfqpoint{0.000000in}{-0.083333in}}%
\pgfusepath{stroke,fill}%
}%
\begin{pgfscope}%
\pgfsys@transformshift{2.946510in}{0.594647in}%
\pgfsys@useobject{currentmarker}{}%
\end{pgfscope}%
\end{pgfscope}%
\begin{pgfscope}%
\definecolor{textcolor}{rgb}{0.150000,0.150000,0.150000}%
\pgfsetstrokecolor{textcolor}%
\pgfsetfillcolor{textcolor}%
\pgftext[x=2.946510in,y=0.414091in,,top]{\color{textcolor}\sffamily\fontsize{10.000000}{12.000000}\selectfont \(\displaystyle 6000\)}%
\end{pgfscope}%
\begin{pgfscope}%
\pgfsetbuttcap%
\pgfsetroundjoin%
\definecolor{currentfill}{rgb}{0.150000,0.150000,0.150000}%
\pgfsetfillcolor{currentfill}%
\pgfsetlinewidth{1.003750pt}%
\definecolor{currentstroke}{rgb}{0.150000,0.150000,0.150000}%
\pgfsetstrokecolor{currentstroke}%
\pgfsetdash{}{0pt}%
\pgfsys@defobject{currentmarker}{\pgfqpoint{0.000000in}{-0.083333in}}{\pgfqpoint{0.000000in}{0.000000in}}{%
\pgfpathmoveto{\pgfqpoint{0.000000in}{0.000000in}}%
\pgfpathlineto{\pgfqpoint{0.000000in}{-0.083333in}}%
\pgfusepath{stroke,fill}%
}%
\begin{pgfscope}%
\pgfsys@transformshift{3.632440in}{0.594647in}%
\pgfsys@useobject{currentmarker}{}%
\end{pgfscope}%
\end{pgfscope}%
\begin{pgfscope}%
\definecolor{textcolor}{rgb}{0.150000,0.150000,0.150000}%
\pgfsetstrokecolor{textcolor}%
\pgfsetfillcolor{textcolor}%
\pgftext[x=3.632440in,y=0.414091in,,top]{\color{textcolor}\sffamily\fontsize{10.000000}{12.000000}\selectfont \(\displaystyle 8000\)}%
\end{pgfscope}%
\begin{pgfscope}%
\definecolor{textcolor}{rgb}{0.150000,0.150000,0.150000}%
\pgfsetstrokecolor{textcolor}%
\pgfsetfillcolor{textcolor}%
\pgftext[x=2.432235in,y=0.231252in,,top]{\color{textcolor}\sffamily\fontsize{11.000000}{13.200000}\selectfont Train Data (\# samples)}%
\end{pgfscope}%
\begin{pgfscope}%
\pgfsetbuttcap%
\pgfsetroundjoin%
\definecolor{currentfill}{rgb}{0.150000,0.150000,0.150000}%
\pgfsetfillcolor{currentfill}%
\pgfsetlinewidth{1.003750pt}%
\definecolor{currentstroke}{rgb}{0.150000,0.150000,0.150000}%
\pgfsetstrokecolor{currentstroke}%
\pgfsetdash{}{0pt}%
\pgfsys@defobject{currentmarker}{\pgfqpoint{-0.083333in}{0.000000in}}{\pgfqpoint{0.000000in}{0.000000in}}{%
\pgfpathmoveto{\pgfqpoint{0.000000in}{0.000000in}}%
\pgfpathlineto{\pgfqpoint{-0.083333in}{0.000000in}}%
\pgfusepath{stroke,fill}%
}%
\begin{pgfscope}%
\pgfsys@transformshift{0.734747in}{0.921577in}%
\pgfsys@useobject{currentmarker}{}%
\end{pgfscope}%
\end{pgfscope}%
\begin{pgfscope}%
\definecolor{textcolor}{rgb}{0.150000,0.150000,0.150000}%
\pgfsetstrokecolor{textcolor}%
\pgfsetfillcolor{textcolor}%
\pgftext[x=0.266189in,y=0.871435in,left,base]{\color{textcolor}\sffamily\fontsize{10.000000}{12.000000}\selectfont \(\displaystyle 10^{-3}\)}%
\end{pgfscope}%
\begin{pgfscope}%
\pgfsetbuttcap%
\pgfsetroundjoin%
\definecolor{currentfill}{rgb}{0.150000,0.150000,0.150000}%
\pgfsetfillcolor{currentfill}%
\pgfsetlinewidth{1.003750pt}%
\definecolor{currentstroke}{rgb}{0.150000,0.150000,0.150000}%
\pgfsetstrokecolor{currentstroke}%
\pgfsetdash{}{0pt}%
\pgfsys@defobject{currentmarker}{\pgfqpoint{-0.083333in}{0.000000in}}{\pgfqpoint{0.000000in}{0.000000in}}{%
\pgfpathmoveto{\pgfqpoint{0.000000in}{0.000000in}}%
\pgfpathlineto{\pgfqpoint{-0.083333in}{0.000000in}}%
\pgfusepath{stroke,fill}%
}%
\begin{pgfscope}%
\pgfsys@transformshift{0.734747in}{1.347574in}%
\pgfsys@useobject{currentmarker}{}%
\end{pgfscope}%
\end{pgfscope}%
\begin{pgfscope}%
\definecolor{textcolor}{rgb}{0.150000,0.150000,0.150000}%
\pgfsetstrokecolor{textcolor}%
\pgfsetfillcolor{textcolor}%
\pgftext[x=0.266189in,y=1.297432in,left,base]{\color{textcolor}\sffamily\fontsize{10.000000}{12.000000}\selectfont \(\displaystyle 10^{-2}\)}%
\end{pgfscope}%
\begin{pgfscope}%
\pgfsetbuttcap%
\pgfsetroundjoin%
\definecolor{currentfill}{rgb}{0.150000,0.150000,0.150000}%
\pgfsetfillcolor{currentfill}%
\pgfsetlinewidth{1.003750pt}%
\definecolor{currentstroke}{rgb}{0.150000,0.150000,0.150000}%
\pgfsetstrokecolor{currentstroke}%
\pgfsetdash{}{0pt}%
\pgfsys@defobject{currentmarker}{\pgfqpoint{-0.083333in}{0.000000in}}{\pgfqpoint{0.000000in}{0.000000in}}{%
\pgfpathmoveto{\pgfqpoint{0.000000in}{0.000000in}}%
\pgfpathlineto{\pgfqpoint{-0.083333in}{0.000000in}}%
\pgfusepath{stroke,fill}%
}%
\begin{pgfscope}%
\pgfsys@transformshift{0.734747in}{1.773571in}%
\pgfsys@useobject{currentmarker}{}%
\end{pgfscope}%
\end{pgfscope}%
\begin{pgfscope}%
\definecolor{textcolor}{rgb}{0.150000,0.150000,0.150000}%
\pgfsetstrokecolor{textcolor}%
\pgfsetfillcolor{textcolor}%
\pgftext[x=0.266189in,y=1.723429in,left,base]{\color{textcolor}\sffamily\fontsize{10.000000}{12.000000}\selectfont \(\displaystyle 10^{-1}\)}%
\end{pgfscope}%
\begin{pgfscope}%
\pgfsetbuttcap%
\pgfsetroundjoin%
\definecolor{currentfill}{rgb}{0.150000,0.150000,0.150000}%
\pgfsetfillcolor{currentfill}%
\pgfsetlinewidth{1.003750pt}%
\definecolor{currentstroke}{rgb}{0.150000,0.150000,0.150000}%
\pgfsetstrokecolor{currentstroke}%
\pgfsetdash{}{0pt}%
\pgfsys@defobject{currentmarker}{\pgfqpoint{-0.083333in}{0.000000in}}{\pgfqpoint{0.000000in}{0.000000in}}{%
\pgfpathmoveto{\pgfqpoint{0.000000in}{0.000000in}}%
\pgfpathlineto{\pgfqpoint{-0.083333in}{0.000000in}}%
\pgfusepath{stroke,fill}%
}%
\begin{pgfscope}%
\pgfsys@transformshift{0.734747in}{2.199568in}%
\pgfsys@useobject{currentmarker}{}%
\end{pgfscope}%
\end{pgfscope}%
\begin{pgfscope}%
\definecolor{textcolor}{rgb}{0.150000,0.150000,0.150000}%
\pgfsetstrokecolor{textcolor}%
\pgfsetfillcolor{textcolor}%
\pgftext[x=0.352995in,y=2.149426in,left,base]{\color{textcolor}\sffamily\fontsize{10.000000}{12.000000}\selectfont \(\displaystyle 10^{0}\)}%
\end{pgfscope}%
\begin{pgfscope}%
\pgfsetbuttcap%
\pgfsetroundjoin%
\definecolor{currentfill}{rgb}{0.150000,0.150000,0.150000}%
\pgfsetfillcolor{currentfill}%
\pgfsetlinewidth{0.501875pt}%
\definecolor{currentstroke}{rgb}{0.150000,0.150000,0.150000}%
\pgfsetstrokecolor{currentstroke}%
\pgfsetdash{}{0pt}%
\pgfsys@defobject{currentmarker}{\pgfqpoint{-0.041667in}{0.000000in}}{\pgfqpoint{0.000000in}{0.000000in}}{%
\pgfpathmoveto{\pgfqpoint{0.000000in}{0.000000in}}%
\pgfpathlineto{\pgfqpoint{-0.041667in}{0.000000in}}%
\pgfusepath{stroke,fill}%
}%
\begin{pgfscope}%
\pgfsys@transformshift{0.734747in}{0.623818in}%
\pgfsys@useobject{currentmarker}{}%
\end{pgfscope}%
\end{pgfscope}%
\begin{pgfscope}%
\pgfsetbuttcap%
\pgfsetroundjoin%
\definecolor{currentfill}{rgb}{0.150000,0.150000,0.150000}%
\pgfsetfillcolor{currentfill}%
\pgfsetlinewidth{0.501875pt}%
\definecolor{currentstroke}{rgb}{0.150000,0.150000,0.150000}%
\pgfsetstrokecolor{currentstroke}%
\pgfsetdash{}{0pt}%
\pgfsys@defobject{currentmarker}{\pgfqpoint{-0.041667in}{0.000000in}}{\pgfqpoint{0.000000in}{0.000000in}}{%
\pgfpathmoveto{\pgfqpoint{0.000000in}{0.000000in}}%
\pgfpathlineto{\pgfqpoint{-0.041667in}{0.000000in}}%
\pgfusepath{stroke,fill}%
}%
\begin{pgfscope}%
\pgfsys@transformshift{0.734747in}{0.698832in}%
\pgfsys@useobject{currentmarker}{}%
\end{pgfscope}%
\end{pgfscope}%
\begin{pgfscope}%
\pgfsetbuttcap%
\pgfsetroundjoin%
\definecolor{currentfill}{rgb}{0.150000,0.150000,0.150000}%
\pgfsetfillcolor{currentfill}%
\pgfsetlinewidth{0.501875pt}%
\definecolor{currentstroke}{rgb}{0.150000,0.150000,0.150000}%
\pgfsetstrokecolor{currentstroke}%
\pgfsetdash{}{0pt}%
\pgfsys@defobject{currentmarker}{\pgfqpoint{-0.041667in}{0.000000in}}{\pgfqpoint{0.000000in}{0.000000in}}{%
\pgfpathmoveto{\pgfqpoint{0.000000in}{0.000000in}}%
\pgfpathlineto{\pgfqpoint{-0.041667in}{0.000000in}}%
\pgfusepath{stroke,fill}%
}%
\begin{pgfscope}%
\pgfsys@transformshift{0.734747in}{0.752056in}%
\pgfsys@useobject{currentmarker}{}%
\end{pgfscope}%
\end{pgfscope}%
\begin{pgfscope}%
\pgfsetbuttcap%
\pgfsetroundjoin%
\definecolor{currentfill}{rgb}{0.150000,0.150000,0.150000}%
\pgfsetfillcolor{currentfill}%
\pgfsetlinewidth{0.501875pt}%
\definecolor{currentstroke}{rgb}{0.150000,0.150000,0.150000}%
\pgfsetstrokecolor{currentstroke}%
\pgfsetdash{}{0pt}%
\pgfsys@defobject{currentmarker}{\pgfqpoint{-0.041667in}{0.000000in}}{\pgfqpoint{0.000000in}{0.000000in}}{%
\pgfpathmoveto{\pgfqpoint{0.000000in}{0.000000in}}%
\pgfpathlineto{\pgfqpoint{-0.041667in}{0.000000in}}%
\pgfusepath{stroke,fill}%
}%
\begin{pgfscope}%
\pgfsys@transformshift{0.734747in}{0.793339in}%
\pgfsys@useobject{currentmarker}{}%
\end{pgfscope}%
\end{pgfscope}%
\begin{pgfscope}%
\pgfsetbuttcap%
\pgfsetroundjoin%
\definecolor{currentfill}{rgb}{0.150000,0.150000,0.150000}%
\pgfsetfillcolor{currentfill}%
\pgfsetlinewidth{0.501875pt}%
\definecolor{currentstroke}{rgb}{0.150000,0.150000,0.150000}%
\pgfsetstrokecolor{currentstroke}%
\pgfsetdash{}{0pt}%
\pgfsys@defobject{currentmarker}{\pgfqpoint{-0.041667in}{0.000000in}}{\pgfqpoint{0.000000in}{0.000000in}}{%
\pgfpathmoveto{\pgfqpoint{0.000000in}{0.000000in}}%
\pgfpathlineto{\pgfqpoint{-0.041667in}{0.000000in}}%
\pgfusepath{stroke,fill}%
}%
\begin{pgfscope}%
\pgfsys@transformshift{0.734747in}{0.827070in}%
\pgfsys@useobject{currentmarker}{}%
\end{pgfscope}%
\end{pgfscope}%
\begin{pgfscope}%
\pgfsetbuttcap%
\pgfsetroundjoin%
\definecolor{currentfill}{rgb}{0.150000,0.150000,0.150000}%
\pgfsetfillcolor{currentfill}%
\pgfsetlinewidth{0.501875pt}%
\definecolor{currentstroke}{rgb}{0.150000,0.150000,0.150000}%
\pgfsetstrokecolor{currentstroke}%
\pgfsetdash{}{0pt}%
\pgfsys@defobject{currentmarker}{\pgfqpoint{-0.041667in}{0.000000in}}{\pgfqpoint{0.000000in}{0.000000in}}{%
\pgfpathmoveto{\pgfqpoint{0.000000in}{0.000000in}}%
\pgfpathlineto{\pgfqpoint{-0.041667in}{0.000000in}}%
\pgfusepath{stroke,fill}%
}%
\begin{pgfscope}%
\pgfsys@transformshift{0.734747in}{0.855589in}%
\pgfsys@useobject{currentmarker}{}%
\end{pgfscope}%
\end{pgfscope}%
\begin{pgfscope}%
\pgfsetbuttcap%
\pgfsetroundjoin%
\definecolor{currentfill}{rgb}{0.150000,0.150000,0.150000}%
\pgfsetfillcolor{currentfill}%
\pgfsetlinewidth{0.501875pt}%
\definecolor{currentstroke}{rgb}{0.150000,0.150000,0.150000}%
\pgfsetstrokecolor{currentstroke}%
\pgfsetdash{}{0pt}%
\pgfsys@defobject{currentmarker}{\pgfqpoint{-0.041667in}{0.000000in}}{\pgfqpoint{0.000000in}{0.000000in}}{%
\pgfpathmoveto{\pgfqpoint{0.000000in}{0.000000in}}%
\pgfpathlineto{\pgfqpoint{-0.041667in}{0.000000in}}%
\pgfusepath{stroke,fill}%
}%
\begin{pgfscope}%
\pgfsys@transformshift{0.734747in}{0.880294in}%
\pgfsys@useobject{currentmarker}{}%
\end{pgfscope}%
\end{pgfscope}%
\begin{pgfscope}%
\pgfsetbuttcap%
\pgfsetroundjoin%
\definecolor{currentfill}{rgb}{0.150000,0.150000,0.150000}%
\pgfsetfillcolor{currentfill}%
\pgfsetlinewidth{0.501875pt}%
\definecolor{currentstroke}{rgb}{0.150000,0.150000,0.150000}%
\pgfsetstrokecolor{currentstroke}%
\pgfsetdash{}{0pt}%
\pgfsys@defobject{currentmarker}{\pgfqpoint{-0.041667in}{0.000000in}}{\pgfqpoint{0.000000in}{0.000000in}}{%
\pgfpathmoveto{\pgfqpoint{0.000000in}{0.000000in}}%
\pgfpathlineto{\pgfqpoint{-0.041667in}{0.000000in}}%
\pgfusepath{stroke,fill}%
}%
\begin{pgfscope}%
\pgfsys@transformshift{0.734747in}{0.902084in}%
\pgfsys@useobject{currentmarker}{}%
\end{pgfscope}%
\end{pgfscope}%
\begin{pgfscope}%
\pgfsetbuttcap%
\pgfsetroundjoin%
\definecolor{currentfill}{rgb}{0.150000,0.150000,0.150000}%
\pgfsetfillcolor{currentfill}%
\pgfsetlinewidth{0.501875pt}%
\definecolor{currentstroke}{rgb}{0.150000,0.150000,0.150000}%
\pgfsetstrokecolor{currentstroke}%
\pgfsetdash{}{0pt}%
\pgfsys@defobject{currentmarker}{\pgfqpoint{-0.041667in}{0.000000in}}{\pgfqpoint{0.000000in}{0.000000in}}{%
\pgfpathmoveto{\pgfqpoint{0.000000in}{0.000000in}}%
\pgfpathlineto{\pgfqpoint{-0.041667in}{0.000000in}}%
\pgfusepath{stroke,fill}%
}%
\begin{pgfscope}%
\pgfsys@transformshift{0.734747in}{1.049815in}%
\pgfsys@useobject{currentmarker}{}%
\end{pgfscope}%
\end{pgfscope}%
\begin{pgfscope}%
\pgfsetbuttcap%
\pgfsetroundjoin%
\definecolor{currentfill}{rgb}{0.150000,0.150000,0.150000}%
\pgfsetfillcolor{currentfill}%
\pgfsetlinewidth{0.501875pt}%
\definecolor{currentstroke}{rgb}{0.150000,0.150000,0.150000}%
\pgfsetstrokecolor{currentstroke}%
\pgfsetdash{}{0pt}%
\pgfsys@defobject{currentmarker}{\pgfqpoint{-0.041667in}{0.000000in}}{\pgfqpoint{0.000000in}{0.000000in}}{%
\pgfpathmoveto{\pgfqpoint{0.000000in}{0.000000in}}%
\pgfpathlineto{\pgfqpoint{-0.041667in}{0.000000in}}%
\pgfusepath{stroke,fill}%
}%
\begin{pgfscope}%
\pgfsys@transformshift{0.734747in}{1.124829in}%
\pgfsys@useobject{currentmarker}{}%
\end{pgfscope}%
\end{pgfscope}%
\begin{pgfscope}%
\pgfsetbuttcap%
\pgfsetroundjoin%
\definecolor{currentfill}{rgb}{0.150000,0.150000,0.150000}%
\pgfsetfillcolor{currentfill}%
\pgfsetlinewidth{0.501875pt}%
\definecolor{currentstroke}{rgb}{0.150000,0.150000,0.150000}%
\pgfsetstrokecolor{currentstroke}%
\pgfsetdash{}{0pt}%
\pgfsys@defobject{currentmarker}{\pgfqpoint{-0.041667in}{0.000000in}}{\pgfqpoint{0.000000in}{0.000000in}}{%
\pgfpathmoveto{\pgfqpoint{0.000000in}{0.000000in}}%
\pgfpathlineto{\pgfqpoint{-0.041667in}{0.000000in}}%
\pgfusepath{stroke,fill}%
}%
\begin{pgfscope}%
\pgfsys@transformshift{0.734747in}{1.178053in}%
\pgfsys@useobject{currentmarker}{}%
\end{pgfscope}%
\end{pgfscope}%
\begin{pgfscope}%
\pgfsetbuttcap%
\pgfsetroundjoin%
\definecolor{currentfill}{rgb}{0.150000,0.150000,0.150000}%
\pgfsetfillcolor{currentfill}%
\pgfsetlinewidth{0.501875pt}%
\definecolor{currentstroke}{rgb}{0.150000,0.150000,0.150000}%
\pgfsetstrokecolor{currentstroke}%
\pgfsetdash{}{0pt}%
\pgfsys@defobject{currentmarker}{\pgfqpoint{-0.041667in}{0.000000in}}{\pgfqpoint{0.000000in}{0.000000in}}{%
\pgfpathmoveto{\pgfqpoint{0.000000in}{0.000000in}}%
\pgfpathlineto{\pgfqpoint{-0.041667in}{0.000000in}}%
\pgfusepath{stroke,fill}%
}%
\begin{pgfscope}%
\pgfsys@transformshift{0.734747in}{1.219336in}%
\pgfsys@useobject{currentmarker}{}%
\end{pgfscope}%
\end{pgfscope}%
\begin{pgfscope}%
\pgfsetbuttcap%
\pgfsetroundjoin%
\definecolor{currentfill}{rgb}{0.150000,0.150000,0.150000}%
\pgfsetfillcolor{currentfill}%
\pgfsetlinewidth{0.501875pt}%
\definecolor{currentstroke}{rgb}{0.150000,0.150000,0.150000}%
\pgfsetstrokecolor{currentstroke}%
\pgfsetdash{}{0pt}%
\pgfsys@defobject{currentmarker}{\pgfqpoint{-0.041667in}{0.000000in}}{\pgfqpoint{0.000000in}{0.000000in}}{%
\pgfpathmoveto{\pgfqpoint{0.000000in}{0.000000in}}%
\pgfpathlineto{\pgfqpoint{-0.041667in}{0.000000in}}%
\pgfusepath{stroke,fill}%
}%
\begin{pgfscope}%
\pgfsys@transformshift{0.734747in}{1.253067in}%
\pgfsys@useobject{currentmarker}{}%
\end{pgfscope}%
\end{pgfscope}%
\begin{pgfscope}%
\pgfsetbuttcap%
\pgfsetroundjoin%
\definecolor{currentfill}{rgb}{0.150000,0.150000,0.150000}%
\pgfsetfillcolor{currentfill}%
\pgfsetlinewidth{0.501875pt}%
\definecolor{currentstroke}{rgb}{0.150000,0.150000,0.150000}%
\pgfsetstrokecolor{currentstroke}%
\pgfsetdash{}{0pt}%
\pgfsys@defobject{currentmarker}{\pgfqpoint{-0.041667in}{0.000000in}}{\pgfqpoint{0.000000in}{0.000000in}}{%
\pgfpathmoveto{\pgfqpoint{0.000000in}{0.000000in}}%
\pgfpathlineto{\pgfqpoint{-0.041667in}{0.000000in}}%
\pgfusepath{stroke,fill}%
}%
\begin{pgfscope}%
\pgfsys@transformshift{0.734747in}{1.281586in}%
\pgfsys@useobject{currentmarker}{}%
\end{pgfscope}%
\end{pgfscope}%
\begin{pgfscope}%
\pgfsetbuttcap%
\pgfsetroundjoin%
\definecolor{currentfill}{rgb}{0.150000,0.150000,0.150000}%
\pgfsetfillcolor{currentfill}%
\pgfsetlinewidth{0.501875pt}%
\definecolor{currentstroke}{rgb}{0.150000,0.150000,0.150000}%
\pgfsetstrokecolor{currentstroke}%
\pgfsetdash{}{0pt}%
\pgfsys@defobject{currentmarker}{\pgfqpoint{-0.041667in}{0.000000in}}{\pgfqpoint{0.000000in}{0.000000in}}{%
\pgfpathmoveto{\pgfqpoint{0.000000in}{0.000000in}}%
\pgfpathlineto{\pgfqpoint{-0.041667in}{0.000000in}}%
\pgfusepath{stroke,fill}%
}%
\begin{pgfscope}%
\pgfsys@transformshift{0.734747in}{1.306290in}%
\pgfsys@useobject{currentmarker}{}%
\end{pgfscope}%
\end{pgfscope}%
\begin{pgfscope}%
\pgfsetbuttcap%
\pgfsetroundjoin%
\definecolor{currentfill}{rgb}{0.150000,0.150000,0.150000}%
\pgfsetfillcolor{currentfill}%
\pgfsetlinewidth{0.501875pt}%
\definecolor{currentstroke}{rgb}{0.150000,0.150000,0.150000}%
\pgfsetstrokecolor{currentstroke}%
\pgfsetdash{}{0pt}%
\pgfsys@defobject{currentmarker}{\pgfqpoint{-0.041667in}{0.000000in}}{\pgfqpoint{0.000000in}{0.000000in}}{%
\pgfpathmoveto{\pgfqpoint{0.000000in}{0.000000in}}%
\pgfpathlineto{\pgfqpoint{-0.041667in}{0.000000in}}%
\pgfusepath{stroke,fill}%
}%
\begin{pgfscope}%
\pgfsys@transformshift{0.734747in}{1.328081in}%
\pgfsys@useobject{currentmarker}{}%
\end{pgfscope}%
\end{pgfscope}%
\begin{pgfscope}%
\pgfsetbuttcap%
\pgfsetroundjoin%
\definecolor{currentfill}{rgb}{0.150000,0.150000,0.150000}%
\pgfsetfillcolor{currentfill}%
\pgfsetlinewidth{0.501875pt}%
\definecolor{currentstroke}{rgb}{0.150000,0.150000,0.150000}%
\pgfsetstrokecolor{currentstroke}%
\pgfsetdash{}{0pt}%
\pgfsys@defobject{currentmarker}{\pgfqpoint{-0.041667in}{0.000000in}}{\pgfqpoint{0.000000in}{0.000000in}}{%
\pgfpathmoveto{\pgfqpoint{0.000000in}{0.000000in}}%
\pgfpathlineto{\pgfqpoint{-0.041667in}{0.000000in}}%
\pgfusepath{stroke,fill}%
}%
\begin{pgfscope}%
\pgfsys@transformshift{0.734747in}{1.475812in}%
\pgfsys@useobject{currentmarker}{}%
\end{pgfscope}%
\end{pgfscope}%
\begin{pgfscope}%
\pgfsetbuttcap%
\pgfsetroundjoin%
\definecolor{currentfill}{rgb}{0.150000,0.150000,0.150000}%
\pgfsetfillcolor{currentfill}%
\pgfsetlinewidth{0.501875pt}%
\definecolor{currentstroke}{rgb}{0.150000,0.150000,0.150000}%
\pgfsetstrokecolor{currentstroke}%
\pgfsetdash{}{0pt}%
\pgfsys@defobject{currentmarker}{\pgfqpoint{-0.041667in}{0.000000in}}{\pgfqpoint{0.000000in}{0.000000in}}{%
\pgfpathmoveto{\pgfqpoint{0.000000in}{0.000000in}}%
\pgfpathlineto{\pgfqpoint{-0.041667in}{0.000000in}}%
\pgfusepath{stroke,fill}%
}%
\begin{pgfscope}%
\pgfsys@transformshift{0.734747in}{1.550826in}%
\pgfsys@useobject{currentmarker}{}%
\end{pgfscope}%
\end{pgfscope}%
\begin{pgfscope}%
\pgfsetbuttcap%
\pgfsetroundjoin%
\definecolor{currentfill}{rgb}{0.150000,0.150000,0.150000}%
\pgfsetfillcolor{currentfill}%
\pgfsetlinewidth{0.501875pt}%
\definecolor{currentstroke}{rgb}{0.150000,0.150000,0.150000}%
\pgfsetstrokecolor{currentstroke}%
\pgfsetdash{}{0pt}%
\pgfsys@defobject{currentmarker}{\pgfqpoint{-0.041667in}{0.000000in}}{\pgfqpoint{0.000000in}{0.000000in}}{%
\pgfpathmoveto{\pgfqpoint{0.000000in}{0.000000in}}%
\pgfpathlineto{\pgfqpoint{-0.041667in}{0.000000in}}%
\pgfusepath{stroke,fill}%
}%
\begin{pgfscope}%
\pgfsys@transformshift{0.734747in}{1.604050in}%
\pgfsys@useobject{currentmarker}{}%
\end{pgfscope}%
\end{pgfscope}%
\begin{pgfscope}%
\pgfsetbuttcap%
\pgfsetroundjoin%
\definecolor{currentfill}{rgb}{0.150000,0.150000,0.150000}%
\pgfsetfillcolor{currentfill}%
\pgfsetlinewidth{0.501875pt}%
\definecolor{currentstroke}{rgb}{0.150000,0.150000,0.150000}%
\pgfsetstrokecolor{currentstroke}%
\pgfsetdash{}{0pt}%
\pgfsys@defobject{currentmarker}{\pgfqpoint{-0.041667in}{0.000000in}}{\pgfqpoint{0.000000in}{0.000000in}}{%
\pgfpathmoveto{\pgfqpoint{0.000000in}{0.000000in}}%
\pgfpathlineto{\pgfqpoint{-0.041667in}{0.000000in}}%
\pgfusepath{stroke,fill}%
}%
\begin{pgfscope}%
\pgfsys@transformshift{0.734747in}{1.645333in}%
\pgfsys@useobject{currentmarker}{}%
\end{pgfscope}%
\end{pgfscope}%
\begin{pgfscope}%
\pgfsetbuttcap%
\pgfsetroundjoin%
\definecolor{currentfill}{rgb}{0.150000,0.150000,0.150000}%
\pgfsetfillcolor{currentfill}%
\pgfsetlinewidth{0.501875pt}%
\definecolor{currentstroke}{rgb}{0.150000,0.150000,0.150000}%
\pgfsetstrokecolor{currentstroke}%
\pgfsetdash{}{0pt}%
\pgfsys@defobject{currentmarker}{\pgfqpoint{-0.041667in}{0.000000in}}{\pgfqpoint{0.000000in}{0.000000in}}{%
\pgfpathmoveto{\pgfqpoint{0.000000in}{0.000000in}}%
\pgfpathlineto{\pgfqpoint{-0.041667in}{0.000000in}}%
\pgfusepath{stroke,fill}%
}%
\begin{pgfscope}%
\pgfsys@transformshift{0.734747in}{1.679064in}%
\pgfsys@useobject{currentmarker}{}%
\end{pgfscope}%
\end{pgfscope}%
\begin{pgfscope}%
\pgfsetbuttcap%
\pgfsetroundjoin%
\definecolor{currentfill}{rgb}{0.150000,0.150000,0.150000}%
\pgfsetfillcolor{currentfill}%
\pgfsetlinewidth{0.501875pt}%
\definecolor{currentstroke}{rgb}{0.150000,0.150000,0.150000}%
\pgfsetstrokecolor{currentstroke}%
\pgfsetdash{}{0pt}%
\pgfsys@defobject{currentmarker}{\pgfqpoint{-0.041667in}{0.000000in}}{\pgfqpoint{0.000000in}{0.000000in}}{%
\pgfpathmoveto{\pgfqpoint{0.000000in}{0.000000in}}%
\pgfpathlineto{\pgfqpoint{-0.041667in}{0.000000in}}%
\pgfusepath{stroke,fill}%
}%
\begin{pgfscope}%
\pgfsys@transformshift{0.734747in}{1.707583in}%
\pgfsys@useobject{currentmarker}{}%
\end{pgfscope}%
\end{pgfscope}%
\begin{pgfscope}%
\pgfsetbuttcap%
\pgfsetroundjoin%
\definecolor{currentfill}{rgb}{0.150000,0.150000,0.150000}%
\pgfsetfillcolor{currentfill}%
\pgfsetlinewidth{0.501875pt}%
\definecolor{currentstroke}{rgb}{0.150000,0.150000,0.150000}%
\pgfsetstrokecolor{currentstroke}%
\pgfsetdash{}{0pt}%
\pgfsys@defobject{currentmarker}{\pgfqpoint{-0.041667in}{0.000000in}}{\pgfqpoint{0.000000in}{0.000000in}}{%
\pgfpathmoveto{\pgfqpoint{0.000000in}{0.000000in}}%
\pgfpathlineto{\pgfqpoint{-0.041667in}{0.000000in}}%
\pgfusepath{stroke,fill}%
}%
\begin{pgfscope}%
\pgfsys@transformshift{0.734747in}{1.732287in}%
\pgfsys@useobject{currentmarker}{}%
\end{pgfscope}%
\end{pgfscope}%
\begin{pgfscope}%
\pgfsetbuttcap%
\pgfsetroundjoin%
\definecolor{currentfill}{rgb}{0.150000,0.150000,0.150000}%
\pgfsetfillcolor{currentfill}%
\pgfsetlinewidth{0.501875pt}%
\definecolor{currentstroke}{rgb}{0.150000,0.150000,0.150000}%
\pgfsetstrokecolor{currentstroke}%
\pgfsetdash{}{0pt}%
\pgfsys@defobject{currentmarker}{\pgfqpoint{-0.041667in}{0.000000in}}{\pgfqpoint{0.000000in}{0.000000in}}{%
\pgfpathmoveto{\pgfqpoint{0.000000in}{0.000000in}}%
\pgfpathlineto{\pgfqpoint{-0.041667in}{0.000000in}}%
\pgfusepath{stroke,fill}%
}%
\begin{pgfscope}%
\pgfsys@transformshift{0.734747in}{1.754078in}%
\pgfsys@useobject{currentmarker}{}%
\end{pgfscope}%
\end{pgfscope}%
\begin{pgfscope}%
\pgfsetbuttcap%
\pgfsetroundjoin%
\definecolor{currentfill}{rgb}{0.150000,0.150000,0.150000}%
\pgfsetfillcolor{currentfill}%
\pgfsetlinewidth{0.501875pt}%
\definecolor{currentstroke}{rgb}{0.150000,0.150000,0.150000}%
\pgfsetstrokecolor{currentstroke}%
\pgfsetdash{}{0pt}%
\pgfsys@defobject{currentmarker}{\pgfqpoint{-0.041667in}{0.000000in}}{\pgfqpoint{0.000000in}{0.000000in}}{%
\pgfpathmoveto{\pgfqpoint{0.000000in}{0.000000in}}%
\pgfpathlineto{\pgfqpoint{-0.041667in}{0.000000in}}%
\pgfusepath{stroke,fill}%
}%
\begin{pgfscope}%
\pgfsys@transformshift{0.734747in}{1.901809in}%
\pgfsys@useobject{currentmarker}{}%
\end{pgfscope}%
\end{pgfscope}%
\begin{pgfscope}%
\pgfsetbuttcap%
\pgfsetroundjoin%
\definecolor{currentfill}{rgb}{0.150000,0.150000,0.150000}%
\pgfsetfillcolor{currentfill}%
\pgfsetlinewidth{0.501875pt}%
\definecolor{currentstroke}{rgb}{0.150000,0.150000,0.150000}%
\pgfsetstrokecolor{currentstroke}%
\pgfsetdash{}{0pt}%
\pgfsys@defobject{currentmarker}{\pgfqpoint{-0.041667in}{0.000000in}}{\pgfqpoint{0.000000in}{0.000000in}}{%
\pgfpathmoveto{\pgfqpoint{0.000000in}{0.000000in}}%
\pgfpathlineto{\pgfqpoint{-0.041667in}{0.000000in}}%
\pgfusepath{stroke,fill}%
}%
\begin{pgfscope}%
\pgfsys@transformshift{0.734747in}{1.976823in}%
\pgfsys@useobject{currentmarker}{}%
\end{pgfscope}%
\end{pgfscope}%
\begin{pgfscope}%
\pgfsetbuttcap%
\pgfsetroundjoin%
\definecolor{currentfill}{rgb}{0.150000,0.150000,0.150000}%
\pgfsetfillcolor{currentfill}%
\pgfsetlinewidth{0.501875pt}%
\definecolor{currentstroke}{rgb}{0.150000,0.150000,0.150000}%
\pgfsetstrokecolor{currentstroke}%
\pgfsetdash{}{0pt}%
\pgfsys@defobject{currentmarker}{\pgfqpoint{-0.041667in}{0.000000in}}{\pgfqpoint{0.000000in}{0.000000in}}{%
\pgfpathmoveto{\pgfqpoint{0.000000in}{0.000000in}}%
\pgfpathlineto{\pgfqpoint{-0.041667in}{0.000000in}}%
\pgfusepath{stroke,fill}%
}%
\begin{pgfscope}%
\pgfsys@transformshift{0.734747in}{2.030047in}%
\pgfsys@useobject{currentmarker}{}%
\end{pgfscope}%
\end{pgfscope}%
\begin{pgfscope}%
\pgfsetbuttcap%
\pgfsetroundjoin%
\definecolor{currentfill}{rgb}{0.150000,0.150000,0.150000}%
\pgfsetfillcolor{currentfill}%
\pgfsetlinewidth{0.501875pt}%
\definecolor{currentstroke}{rgb}{0.150000,0.150000,0.150000}%
\pgfsetstrokecolor{currentstroke}%
\pgfsetdash{}{0pt}%
\pgfsys@defobject{currentmarker}{\pgfqpoint{-0.041667in}{0.000000in}}{\pgfqpoint{0.000000in}{0.000000in}}{%
\pgfpathmoveto{\pgfqpoint{0.000000in}{0.000000in}}%
\pgfpathlineto{\pgfqpoint{-0.041667in}{0.000000in}}%
\pgfusepath{stroke,fill}%
}%
\begin{pgfscope}%
\pgfsys@transformshift{0.734747in}{2.071330in}%
\pgfsys@useobject{currentmarker}{}%
\end{pgfscope}%
\end{pgfscope}%
\begin{pgfscope}%
\pgfsetbuttcap%
\pgfsetroundjoin%
\definecolor{currentfill}{rgb}{0.150000,0.150000,0.150000}%
\pgfsetfillcolor{currentfill}%
\pgfsetlinewidth{0.501875pt}%
\definecolor{currentstroke}{rgb}{0.150000,0.150000,0.150000}%
\pgfsetstrokecolor{currentstroke}%
\pgfsetdash{}{0pt}%
\pgfsys@defobject{currentmarker}{\pgfqpoint{-0.041667in}{0.000000in}}{\pgfqpoint{0.000000in}{0.000000in}}{%
\pgfpathmoveto{\pgfqpoint{0.000000in}{0.000000in}}%
\pgfpathlineto{\pgfqpoint{-0.041667in}{0.000000in}}%
\pgfusepath{stroke,fill}%
}%
\begin{pgfscope}%
\pgfsys@transformshift{0.734747in}{2.105061in}%
\pgfsys@useobject{currentmarker}{}%
\end{pgfscope}%
\end{pgfscope}%
\begin{pgfscope}%
\pgfsetbuttcap%
\pgfsetroundjoin%
\definecolor{currentfill}{rgb}{0.150000,0.150000,0.150000}%
\pgfsetfillcolor{currentfill}%
\pgfsetlinewidth{0.501875pt}%
\definecolor{currentstroke}{rgb}{0.150000,0.150000,0.150000}%
\pgfsetstrokecolor{currentstroke}%
\pgfsetdash{}{0pt}%
\pgfsys@defobject{currentmarker}{\pgfqpoint{-0.041667in}{0.000000in}}{\pgfqpoint{0.000000in}{0.000000in}}{%
\pgfpathmoveto{\pgfqpoint{0.000000in}{0.000000in}}%
\pgfpathlineto{\pgfqpoint{-0.041667in}{0.000000in}}%
\pgfusepath{stroke,fill}%
}%
\begin{pgfscope}%
\pgfsys@transformshift{0.734747in}{2.133580in}%
\pgfsys@useobject{currentmarker}{}%
\end{pgfscope}%
\end{pgfscope}%
\begin{pgfscope}%
\pgfsetbuttcap%
\pgfsetroundjoin%
\definecolor{currentfill}{rgb}{0.150000,0.150000,0.150000}%
\pgfsetfillcolor{currentfill}%
\pgfsetlinewidth{0.501875pt}%
\definecolor{currentstroke}{rgb}{0.150000,0.150000,0.150000}%
\pgfsetstrokecolor{currentstroke}%
\pgfsetdash{}{0pt}%
\pgfsys@defobject{currentmarker}{\pgfqpoint{-0.041667in}{0.000000in}}{\pgfqpoint{0.000000in}{0.000000in}}{%
\pgfpathmoveto{\pgfqpoint{0.000000in}{0.000000in}}%
\pgfpathlineto{\pgfqpoint{-0.041667in}{0.000000in}}%
\pgfusepath{stroke,fill}%
}%
\begin{pgfscope}%
\pgfsys@transformshift{0.734747in}{2.158284in}%
\pgfsys@useobject{currentmarker}{}%
\end{pgfscope}%
\end{pgfscope}%
\begin{pgfscope}%
\pgfsetbuttcap%
\pgfsetroundjoin%
\definecolor{currentfill}{rgb}{0.150000,0.150000,0.150000}%
\pgfsetfillcolor{currentfill}%
\pgfsetlinewidth{0.501875pt}%
\definecolor{currentstroke}{rgb}{0.150000,0.150000,0.150000}%
\pgfsetstrokecolor{currentstroke}%
\pgfsetdash{}{0pt}%
\pgfsys@defobject{currentmarker}{\pgfqpoint{-0.041667in}{0.000000in}}{\pgfqpoint{0.000000in}{0.000000in}}{%
\pgfpathmoveto{\pgfqpoint{0.000000in}{0.000000in}}%
\pgfpathlineto{\pgfqpoint{-0.041667in}{0.000000in}}%
\pgfusepath{stroke,fill}%
}%
\begin{pgfscope}%
\pgfsys@transformshift{0.734747in}{2.180075in}%
\pgfsys@useobject{currentmarker}{}%
\end{pgfscope}%
\end{pgfscope}%
\begin{pgfscope}%
\pgfsetbuttcap%
\pgfsetroundjoin%
\definecolor{currentfill}{rgb}{0.150000,0.150000,0.150000}%
\pgfsetfillcolor{currentfill}%
\pgfsetlinewidth{0.501875pt}%
\definecolor{currentstroke}{rgb}{0.150000,0.150000,0.150000}%
\pgfsetstrokecolor{currentstroke}%
\pgfsetdash{}{0pt}%
\pgfsys@defobject{currentmarker}{\pgfqpoint{-0.041667in}{0.000000in}}{\pgfqpoint{0.000000in}{0.000000in}}{%
\pgfpathmoveto{\pgfqpoint{0.000000in}{0.000000in}}%
\pgfpathlineto{\pgfqpoint{-0.041667in}{0.000000in}}%
\pgfusepath{stroke,fill}%
}%
\begin{pgfscope}%
\pgfsys@transformshift{0.734747in}{2.327806in}%
\pgfsys@useobject{currentmarker}{}%
\end{pgfscope}%
\end{pgfscope}%
\begin{pgfscope}%
\pgfsetbuttcap%
\pgfsetroundjoin%
\definecolor{currentfill}{rgb}{0.150000,0.150000,0.150000}%
\pgfsetfillcolor{currentfill}%
\pgfsetlinewidth{0.501875pt}%
\definecolor{currentstroke}{rgb}{0.150000,0.150000,0.150000}%
\pgfsetstrokecolor{currentstroke}%
\pgfsetdash{}{0pt}%
\pgfsys@defobject{currentmarker}{\pgfqpoint{-0.041667in}{0.000000in}}{\pgfqpoint{0.000000in}{0.000000in}}{%
\pgfpathmoveto{\pgfqpoint{0.000000in}{0.000000in}}%
\pgfpathlineto{\pgfqpoint{-0.041667in}{0.000000in}}%
\pgfusepath{stroke,fill}%
}%
\begin{pgfscope}%
\pgfsys@transformshift{0.734747in}{2.402820in}%
\pgfsys@useobject{currentmarker}{}%
\end{pgfscope}%
\end{pgfscope}%
\begin{pgfscope}%
\pgfsetbuttcap%
\pgfsetroundjoin%
\definecolor{currentfill}{rgb}{0.150000,0.150000,0.150000}%
\pgfsetfillcolor{currentfill}%
\pgfsetlinewidth{0.501875pt}%
\definecolor{currentstroke}{rgb}{0.150000,0.150000,0.150000}%
\pgfsetstrokecolor{currentstroke}%
\pgfsetdash{}{0pt}%
\pgfsys@defobject{currentmarker}{\pgfqpoint{-0.041667in}{0.000000in}}{\pgfqpoint{0.000000in}{0.000000in}}{%
\pgfpathmoveto{\pgfqpoint{0.000000in}{0.000000in}}%
\pgfpathlineto{\pgfqpoint{-0.041667in}{0.000000in}}%
\pgfusepath{stroke,fill}%
}%
\begin{pgfscope}%
\pgfsys@transformshift{0.734747in}{2.456043in}%
\pgfsys@useobject{currentmarker}{}%
\end{pgfscope}%
\end{pgfscope}%
\begin{pgfscope}%
\definecolor{textcolor}{rgb}{0.150000,0.150000,0.150000}%
\pgfsetstrokecolor{textcolor}%
\pgfsetfillcolor{textcolor}%
\pgftext[x=0.210634in,y=1.541647in,,bottom,rotate=90.000000]{\color{textcolor}\sffamily\fontsize{11.000000}{13.200000}\selectfont Log Train Loss}%
\end{pgfscope}%
\begin{pgfscope}%
\pgfpathrectangle{\pgfqpoint{0.734747in}{0.594647in}}{\pgfqpoint{3.394974in}{1.894001in}} %
\pgfusepath{clip}%
\pgfsetroundcap%
\pgfsetroundjoin%
\pgfsetlinewidth{1.756562pt}%
\definecolor{currentstroke}{rgb}{0.400000,0.760784,0.647059}%
\pgfsetstrokecolor{currentstroke}%
\pgfsetdash{}{0pt}%
\pgfpathmoveto{\pgfqpoint{0.889064in}{2.402557in}}%
\pgfpathlineto{\pgfqpoint{0.889407in}{2.274634in}}%
\pgfpathlineto{\pgfqpoint{0.889750in}{2.199899in}}%
\pgfpathlineto{\pgfqpoint{0.890093in}{1.200613in}}%
\pgfpathlineto{\pgfqpoint{0.890436in}{2.105785in}}%
\pgfpathlineto{\pgfqpoint{0.890779in}{2.199875in}}%
\pgfpathlineto{\pgfqpoint{0.891122in}{1.206753in}}%
\pgfpathlineto{\pgfqpoint{0.891465in}{1.184498in}}%
\pgfpathlineto{\pgfqpoint{0.891808in}{2.252745in}}%
\pgfpathlineto{\pgfqpoint{0.892151in}{2.105809in}}%
\pgfpathlineto{\pgfqpoint{0.895581in}{1.213547in}}%
\pgfpathlineto{\pgfqpoint{0.899010in}{1.220814in}}%
\pgfpathlineto{\pgfqpoint{0.902440in}{1.239454in}}%
\pgfpathlineto{\pgfqpoint{0.905870in}{1.272764in}}%
\pgfpathlineto{\pgfqpoint{0.909299in}{1.284489in}}%
\pgfpathlineto{\pgfqpoint{0.912729in}{1.310865in}}%
\pgfpathlineto{\pgfqpoint{0.916159in}{1.630578in}}%
\pgfpathlineto{\pgfqpoint{0.919588in}{1.423821in}}%
\pgfpathlineto{\pgfqpoint{0.923018in}{1.650245in}}%
\pgfpathlineto{\pgfqpoint{0.957314in}{1.696824in}}%
\pgfpathlineto{\pgfqpoint{0.991611in}{1.632118in}}%
\pgfpathlineto{\pgfqpoint{1.025907in}{1.626665in}}%
\pgfpathlineto{\pgfqpoint{1.060204in}{1.737602in}}%
\pgfpathlineto{\pgfqpoint{1.094500in}{1.662937in}}%
\pgfpathlineto{\pgfqpoint{1.128797in}{1.634766in}}%
\pgfpathlineto{\pgfqpoint{1.163093in}{1.616537in}}%
\pgfpathlineto{\pgfqpoint{1.197390in}{1.716777in}}%
\pgfpathlineto{\pgfqpoint{1.231686in}{1.727944in}}%
\pgfpathlineto{\pgfqpoint{1.265983in}{1.780027in}}%
\pgfpathlineto{\pgfqpoint{1.300279in}{1.574733in}}%
\pgfpathlineto{\pgfqpoint{1.334576in}{1.576094in}}%
\pgfpathlineto{\pgfqpoint{1.368872in}{1.660930in}}%
\pgfpathlineto{\pgfqpoint{1.403169in}{1.695029in}}%
\pgfpathlineto{\pgfqpoint{1.437465in}{1.728616in}}%
\pgfpathlineto{\pgfqpoint{1.471762in}{1.769804in}}%
\pgfpathlineto{\pgfqpoint{1.506058in}{1.662223in}}%
\pgfpathlineto{\pgfqpoint{1.540355in}{1.653902in}}%
\pgfpathlineto{\pgfqpoint{1.574651in}{1.709819in}}%
\pgfpathlineto{\pgfqpoint{1.608948in}{1.708825in}}%
\pgfpathlineto{\pgfqpoint{1.643244in}{1.711136in}}%
\pgfpathlineto{\pgfqpoint{1.677541in}{1.633428in}}%
\pgfpathlineto{\pgfqpoint{1.711837in}{1.690583in}}%
\pgfpathlineto{\pgfqpoint{1.746134in}{1.658783in}}%
\pgfpathlineto{\pgfqpoint{1.780430in}{1.592869in}}%
\pgfpathlineto{\pgfqpoint{1.814726in}{1.652715in}}%
\pgfpathlineto{\pgfqpoint{1.849023in}{1.640886in}}%
\pgfpathlineto{\pgfqpoint{1.883319in}{1.560693in}}%
\pgfpathlineto{\pgfqpoint{1.917616in}{1.636207in}}%
\pgfpathlineto{\pgfqpoint{1.951912in}{1.781445in}}%
\pgfpathlineto{\pgfqpoint{1.986209in}{1.583424in}}%
\pgfpathlineto{\pgfqpoint{2.020505in}{1.659776in}}%
\pgfpathlineto{\pgfqpoint{2.054802in}{1.586026in}}%
\pgfpathlineto{\pgfqpoint{2.089098in}{1.649850in}}%
\pgfpathlineto{\pgfqpoint{2.123395in}{1.629982in}}%
\pgfpathlineto{\pgfqpoint{2.157691in}{1.609478in}}%
\pgfpathlineto{\pgfqpoint{2.191988in}{1.649966in}}%
\pgfpathlineto{\pgfqpoint{2.226284in}{1.615754in}}%
\pgfpathlineto{\pgfqpoint{2.260581in}{1.744754in}}%
\pgfpathlineto{\pgfqpoint{2.294877in}{1.569490in}}%
\pgfpathlineto{\pgfqpoint{2.329174in}{1.660824in}}%
\pgfpathlineto{\pgfqpoint{2.363470in}{1.658671in}}%
\pgfpathlineto{\pgfqpoint{2.397767in}{1.761698in}}%
\pgfpathlineto{\pgfqpoint{2.432063in}{1.675552in}}%
\pgfpathlineto{\pgfqpoint{2.466360in}{1.596388in}}%
\pgfpathlineto{\pgfqpoint{2.500656in}{1.702694in}}%
\pgfpathlineto{\pgfqpoint{2.534953in}{1.791912in}}%
\pgfpathlineto{\pgfqpoint{2.569249in}{1.690306in}}%
\pgfpathlineto{\pgfqpoint{2.603546in}{1.677261in}}%
\pgfpathlineto{\pgfqpoint{2.637842in}{1.732196in}}%
\pgfpathlineto{\pgfqpoint{2.672139in}{1.667627in}}%
\pgfpathlineto{\pgfqpoint{2.706435in}{1.639758in}}%
\pgfpathlineto{\pgfqpoint{2.740732in}{1.618562in}}%
\pgfpathlineto{\pgfqpoint{2.775028in}{1.630496in}}%
\pgfpathlineto{\pgfqpoint{2.809324in}{1.610957in}}%
\pgfpathlineto{\pgfqpoint{2.843621in}{1.687798in}}%
\pgfpathlineto{\pgfqpoint{2.877917in}{1.683211in}}%
\pgfpathlineto{\pgfqpoint{2.912214in}{1.612355in}}%
\pgfpathlineto{\pgfqpoint{2.946510in}{1.687171in}}%
\pgfpathlineto{\pgfqpoint{2.980807in}{1.672424in}}%
\pgfpathlineto{\pgfqpoint{3.015103in}{1.647757in}}%
\pgfpathlineto{\pgfqpoint{3.049400in}{1.633098in}}%
\pgfpathlineto{\pgfqpoint{3.083696in}{1.652488in}}%
\pgfpathlineto{\pgfqpoint{3.117993in}{1.581278in}}%
\pgfpathlineto{\pgfqpoint{3.152289in}{1.618070in}}%
\pgfpathlineto{\pgfqpoint{3.186586in}{1.651315in}}%
\pgfpathlineto{\pgfqpoint{3.220882in}{1.675118in}}%
\pgfpathlineto{\pgfqpoint{3.255179in}{1.673550in}}%
\pgfpathlineto{\pgfqpoint{3.289475in}{1.633669in}}%
\pgfpathlineto{\pgfqpoint{3.323772in}{1.642273in}}%
\pgfpathlineto{\pgfqpoint{3.358068in}{1.561994in}}%
\pgfpathlineto{\pgfqpoint{3.392365in}{1.662220in}}%
\pgfpathlineto{\pgfqpoint{3.426661in}{1.642274in}}%
\pgfpathlineto{\pgfqpoint{3.460958in}{1.592545in}}%
\pgfpathlineto{\pgfqpoint{3.495254in}{1.596872in}}%
\pgfpathlineto{\pgfqpoint{3.529551in}{1.661302in}}%
\pgfpathlineto{\pgfqpoint{3.563847in}{1.639512in}}%
\pgfpathlineto{\pgfqpoint{3.598144in}{1.641053in}}%
\pgfpathlineto{\pgfqpoint{3.632440in}{1.656932in}}%
\pgfpathlineto{\pgfqpoint{3.666737in}{1.596052in}}%
\pgfpathlineto{\pgfqpoint{3.701033in}{1.657405in}}%
\pgfpathlineto{\pgfqpoint{3.735329in}{1.597961in}}%
\pgfpathlineto{\pgfqpoint{3.769626in}{1.650038in}}%
\pgfpathlineto{\pgfqpoint{3.803922in}{1.647404in}}%
\pgfpathlineto{\pgfqpoint{3.838219in}{1.680509in}}%
\pgfpathlineto{\pgfqpoint{3.872515in}{1.639986in}}%
\pgfpathlineto{\pgfqpoint{3.906812in}{1.654573in}}%
\pgfpathlineto{\pgfqpoint{3.941108in}{1.683041in}}%
\pgfpathlineto{\pgfqpoint{3.975405in}{1.629540in}}%
\pgfusepath{stroke}%
\end{pgfscope}%
\begin{pgfscope}%
\pgfpathrectangle{\pgfqpoint{0.734747in}{0.594647in}}{\pgfqpoint{3.394974in}{1.894001in}} %
\pgfusepath{clip}%
\pgfsetroundcap%
\pgfsetroundjoin%
\pgfsetlinewidth{1.756562pt}%
\definecolor{currentstroke}{rgb}{0.988235,0.552941,0.384314}%
\pgfsetstrokecolor{currentstroke}%
\pgfsetdash{}{0pt}%
\pgfpathmoveto{\pgfqpoint{0.889064in}{2.402557in}}%
\pgfpathlineto{\pgfqpoint{0.889407in}{2.402557in}}%
\pgfpathlineto{\pgfqpoint{0.889750in}{1.106131in}}%
\pgfpathlineto{\pgfqpoint{0.890093in}{0.974124in}}%
\pgfpathlineto{\pgfqpoint{0.890436in}{1.163861in}}%
\pgfpathlineto{\pgfqpoint{0.890779in}{2.368945in}}%
\pgfpathlineto{\pgfqpoint{0.891122in}{1.281657in}}%
\pgfpathlineto{\pgfqpoint{0.891465in}{1.151568in}}%
\pgfpathlineto{\pgfqpoint{0.891808in}{2.327685in}}%
\pgfpathlineto{\pgfqpoint{0.892151in}{1.978584in}}%
\pgfpathlineto{\pgfqpoint{0.895581in}{1.208465in}}%
\pgfpathlineto{\pgfqpoint{0.899010in}{1.325730in}}%
\pgfpathlineto{\pgfqpoint{0.902440in}{1.414583in}}%
\pgfpathlineto{\pgfqpoint{0.905870in}{1.745692in}}%
\pgfpathlineto{\pgfqpoint{0.909299in}{1.499673in}}%
\pgfpathlineto{\pgfqpoint{0.912729in}{2.155302in}}%
\pgfpathlineto{\pgfqpoint{0.916159in}{1.753886in}}%
\pgfpathlineto{\pgfqpoint{0.919588in}{1.384217in}}%
\pgfpathlineto{\pgfqpoint{0.923018in}{1.315424in}}%
\pgfpathlineto{\pgfqpoint{0.957314in}{2.065572in}}%
\pgfpathlineto{\pgfqpoint{0.991611in}{1.978549in}}%
\pgfpathlineto{\pgfqpoint{1.025907in}{1.084236in}}%
\pgfpathlineto{\pgfqpoint{1.060204in}{1.773489in}}%
\pgfpathlineto{\pgfqpoint{1.094500in}{1.766400in}}%
\pgfpathlineto{\pgfqpoint{1.128797in}{1.376833in}}%
\pgfpathlineto{\pgfqpoint{1.163093in}{1.783165in}}%
\pgfpathlineto{\pgfqpoint{1.197390in}{1.693782in}}%
\pgfpathlineto{\pgfqpoint{1.231686in}{1.317495in}}%
\pgfpathlineto{\pgfqpoint{1.265983in}{0.838088in}}%
\pgfpathlineto{\pgfqpoint{1.300279in}{1.716686in}}%
\pgfpathlineto{\pgfqpoint{1.334576in}{1.639396in}}%
\pgfpathlineto{\pgfqpoint{1.368872in}{0.966366in}}%
\pgfpathlineto{\pgfqpoint{1.403169in}{1.232031in}}%
\pgfpathlineto{\pgfqpoint{1.437465in}{1.619517in}}%
\pgfpathlineto{\pgfqpoint{1.471762in}{0.962267in}}%
\pgfpathlineto{\pgfqpoint{1.506058in}{1.757651in}}%
\pgfpathlineto{\pgfqpoint{1.540355in}{0.824086in}}%
\pgfpathlineto{\pgfqpoint{1.574651in}{1.929573in}}%
\pgfpathlineto{\pgfqpoint{1.608948in}{1.963572in}}%
\pgfpathlineto{\pgfqpoint{1.643244in}{1.512953in}}%
\pgfpathlineto{\pgfqpoint{1.677541in}{1.489338in}}%
\pgfpathlineto{\pgfqpoint{1.711837in}{1.415536in}}%
\pgfpathlineto{\pgfqpoint{1.746134in}{1.485719in}}%
\pgfpathlineto{\pgfqpoint{1.780430in}{1.678788in}}%
\pgfpathlineto{\pgfqpoint{1.814726in}{0.757512in}}%
\pgfpathlineto{\pgfqpoint{1.849023in}{2.173873in}}%
\pgfpathlineto{\pgfqpoint{1.883319in}{1.991787in}}%
\pgfpathlineto{\pgfqpoint{1.917616in}{1.675985in}}%
\pgfpathlineto{\pgfqpoint{1.951912in}{1.373859in}}%
\pgfpathlineto{\pgfqpoint{1.986209in}{1.659996in}}%
\pgfpathlineto{\pgfqpoint{2.020505in}{1.405660in}}%
\pgfpathlineto{\pgfqpoint{2.054802in}{1.662742in}}%
\pgfpathlineto{\pgfqpoint{2.089098in}{0.848598in}}%
\pgfpathlineto{\pgfqpoint{2.123395in}{1.659797in}}%
\pgfpathlineto{\pgfqpoint{2.157691in}{1.963094in}}%
\pgfpathlineto{\pgfqpoint{2.191988in}{1.167796in}}%
\pgfpathlineto{\pgfqpoint{2.226284in}{0.680738in}}%
\pgfpathlineto{\pgfqpoint{2.260581in}{1.564419in}}%
\pgfpathlineto{\pgfqpoint{2.294877in}{1.411322in}}%
\pgfpathlineto{\pgfqpoint{2.329174in}{1.449238in}}%
\pgfpathlineto{\pgfqpoint{2.363470in}{1.633969in}}%
\pgfpathlineto{\pgfqpoint{2.397767in}{1.419325in}}%
\pgfpathlineto{\pgfqpoint{2.432063in}{1.584131in}}%
\pgfpathlineto{\pgfqpoint{2.466360in}{1.519538in}}%
\pgfpathlineto{\pgfqpoint{2.500656in}{0.955262in}}%
\pgfpathlineto{\pgfqpoint{2.534953in}{1.427769in}}%
\pgfpathlineto{\pgfqpoint{2.569249in}{1.725363in}}%
\pgfpathlineto{\pgfqpoint{2.603546in}{1.508462in}}%
\pgfpathlineto{\pgfqpoint{2.637842in}{1.447854in}}%
\pgfpathlineto{\pgfqpoint{2.672139in}{1.775901in}}%
\pgfpathlineto{\pgfqpoint{2.706435in}{2.042228in}}%
\pgfpathlineto{\pgfqpoint{2.740732in}{1.864537in}}%
\pgfpathlineto{\pgfqpoint{2.775028in}{1.547494in}}%
\pgfpathlineto{\pgfqpoint{2.809324in}{1.385320in}}%
\pgfpathlineto{\pgfqpoint{2.843621in}{1.403515in}}%
\pgfpathlineto{\pgfqpoint{2.877917in}{1.541655in}}%
\pgfpathlineto{\pgfqpoint{2.912214in}{1.400887in}}%
\pgfpathlineto{\pgfqpoint{2.946510in}{1.711402in}}%
\pgfpathlineto{\pgfqpoint{2.980807in}{1.359502in}}%
\pgfpathlineto{\pgfqpoint{3.015103in}{1.531373in}}%
\pgfpathlineto{\pgfqpoint{3.049400in}{1.446624in}}%
\pgfpathlineto{\pgfqpoint{3.083696in}{1.527837in}}%
\pgfpathlineto{\pgfqpoint{3.117993in}{0.739919in}}%
\pgfpathlineto{\pgfqpoint{3.152289in}{1.136003in}}%
\pgfpathlineto{\pgfqpoint{3.186586in}{1.413859in}}%
\pgfpathlineto{\pgfqpoint{3.220882in}{1.687121in}}%
\pgfpathlineto{\pgfqpoint{3.255179in}{1.353752in}}%
\pgfpathlineto{\pgfqpoint{3.289475in}{0.703745in}}%
\pgfpathlineto{\pgfqpoint{3.323772in}{1.684245in}}%
\pgfpathlineto{\pgfqpoint{3.358068in}{1.582656in}}%
\pgfpathlineto{\pgfqpoint{3.392365in}{1.245795in}}%
\pgfpathlineto{\pgfqpoint{3.426661in}{1.679227in}}%
\pgfpathlineto{\pgfqpoint{3.460958in}{1.505038in}}%
\pgfpathlineto{\pgfqpoint{3.495254in}{1.503346in}}%
\pgfpathlineto{\pgfqpoint{3.529551in}{1.496874in}}%
\pgfpathlineto{\pgfqpoint{3.563847in}{1.291649in}}%
\pgfpathlineto{\pgfqpoint{3.598144in}{1.684150in}}%
\pgfpathlineto{\pgfqpoint{3.632440in}{1.639397in}}%
\pgfpathlineto{\pgfqpoint{3.666737in}{1.489464in}}%
\pgfpathlineto{\pgfqpoint{3.701033in}{1.668347in}}%
\pgfpathlineto{\pgfqpoint{3.735329in}{1.665050in}}%
\pgfpathlineto{\pgfqpoint{3.769626in}{1.481036in}}%
\pgfpathlineto{\pgfqpoint{3.803922in}{1.479502in}}%
\pgfpathlineto{\pgfqpoint{3.838219in}{1.722405in}}%
\pgfpathlineto{\pgfqpoint{3.872515in}{1.338636in}}%
\pgfpathlineto{\pgfqpoint{3.906812in}{1.475013in}}%
\pgfpathlineto{\pgfqpoint{3.941108in}{1.652113in}}%
\pgfpathlineto{\pgfqpoint{3.975405in}{2.114038in}}%
\pgfusepath{stroke}%
\end{pgfscope}%
\begin{pgfscope}%
\pgfsetrectcap%
\pgfsetmiterjoin%
\pgfsetlinewidth{1.254687pt}%
\definecolor{currentstroke}{rgb}{0.150000,0.150000,0.150000}%
\pgfsetstrokecolor{currentstroke}%
\pgfsetdash{}{0pt}%
\pgfpathmoveto{\pgfqpoint{0.734747in}{0.594647in}}%
\pgfpathlineto{\pgfqpoint{0.734747in}{2.488647in}}%
\pgfusepath{stroke}%
\end{pgfscope}%
\begin{pgfscope}%
\pgfsetrectcap%
\pgfsetmiterjoin%
\pgfsetlinewidth{1.254687pt}%
\definecolor{currentstroke}{rgb}{0.150000,0.150000,0.150000}%
\pgfsetstrokecolor{currentstroke}%
\pgfsetdash{}{0pt}%
\pgfpathmoveto{\pgfqpoint{0.734747in}{0.594647in}}%
\pgfpathlineto{\pgfqpoint{4.129722in}{0.594647in}}%
\pgfusepath{stroke}%
\end{pgfscope}%
\begin{pgfscope}%
\pgfsetroundcap%
\pgfsetroundjoin%
\pgfsetlinewidth{1.756562pt}%
\definecolor{currentstroke}{rgb}{0.400000,0.760784,0.647059}%
\pgfsetstrokecolor{currentstroke}%
\pgfsetdash{}{0pt}%
\pgfpathmoveto{\pgfqpoint{3.348923in}{0.991986in}}%
\pgfpathlineto{\pgfqpoint{3.626701in}{0.991986in}}%
\pgfusepath{stroke}%
\end{pgfscope}%
\begin{pgfscope}%
\definecolor{textcolor}{rgb}{0.150000,0.150000,0.150000}%
\pgfsetstrokecolor{textcolor}%
\pgfsetfillcolor{textcolor}%
\pgftext[x=3.737812in,y=0.943375in,left,base]{\color{textcolor}\sffamily\fontsize{10.000000}{12.000000}\selectfont fc}%
\end{pgfscope}%
\begin{pgfscope}%
\pgfsetroundcap%
\pgfsetroundjoin%
\pgfsetlinewidth{1.756562pt}%
\definecolor{currentstroke}{rgb}{0.988235,0.552941,0.384314}%
\pgfsetstrokecolor{currentstroke}%
\pgfsetdash{}{0pt}%
\pgfpathmoveto{\pgfqpoint{3.348923in}{0.795258in}}%
\pgfpathlineto{\pgfqpoint{3.626701in}{0.795258in}}%
\pgfusepath{stroke}%
\end{pgfscope}%
\begin{pgfscope}%
\definecolor{textcolor}{rgb}{0.150000,0.150000,0.150000}%
\pgfsetstrokecolor{textcolor}%
\pgfsetfillcolor{textcolor}%
\pgftext[x=3.737812in,y=0.746647in,left,base]{\color{textcolor}\sffamily\fontsize{10.000000}{12.000000}\selectfont conv}%
\end{pgfscope}%
\end{pgfpicture}%
\makeatother%
\endgroup%
}
			\caption{Training Loss}
			\label{fig:tomplotmintrainloss}
		\end{subfigure}
		~
		\begin{subfigure}[t]{0.49\textwidth}
			\resizebox{\linewidth}{!}{%% Creator: Matplotlib, PGF backend
%%
%% To include the figure in your LaTeX document, write
%%   \input{<filename>.pgf}
%%
%% Make sure the required packages are loaded in your preamble
%%   \usepackage{pgf}
%%
%% Figures using additional raster images can only be included by \input if
%% they are in the same directory as the main LaTeX file. For loading figures
%% from other directories you can use the `import` package
%%   \usepackage{import}
%% and then include the figures with
%%   \import{<path to file>}{<filename>.pgf}
%%
%% Matplotlib used the following preamble
%%   \usepackage[utf8x]{inputenc}
%%   \usepackage[T1]{fontenc}
%%
\begingroup%
\makeatletter%
\begin{pgfpicture}%
\pgfpathrectangle{\pgfpointorigin}{\pgfqpoint{4.296389in}{2.655314in}}%
\pgfusepath{use as bounding box, clip}%
\begin{pgfscope}%
\pgfsetbuttcap%
\pgfsetmiterjoin%
\definecolor{currentfill}{rgb}{1.000000,1.000000,1.000000}%
\pgfsetfillcolor{currentfill}%
\pgfsetlinewidth{0.000000pt}%
\definecolor{currentstroke}{rgb}{1.000000,1.000000,1.000000}%
\pgfsetstrokecolor{currentstroke}%
\pgfsetdash{}{0pt}%
\pgfpathmoveto{\pgfqpoint{0.000000in}{0.000000in}}%
\pgfpathlineto{\pgfqpoint{4.296389in}{0.000000in}}%
\pgfpathlineto{\pgfqpoint{4.296389in}{2.655314in}}%
\pgfpathlineto{\pgfqpoint{0.000000in}{2.655314in}}%
\pgfpathclose%
\pgfusepath{fill}%
\end{pgfscope}%
\begin{pgfscope}%
\pgfsetbuttcap%
\pgfsetmiterjoin%
\definecolor{currentfill}{rgb}{1.000000,1.000000,1.000000}%
\pgfsetfillcolor{currentfill}%
\pgfsetlinewidth{0.000000pt}%
\definecolor{currentstroke}{rgb}{0.000000,0.000000,0.000000}%
\pgfsetstrokecolor{currentstroke}%
\pgfsetstrokeopacity{0.000000}%
\pgfsetdash{}{0pt}%
\pgfpathmoveto{\pgfqpoint{0.734747in}{0.594647in}}%
\pgfpathlineto{\pgfqpoint{4.129722in}{0.594647in}}%
\pgfpathlineto{\pgfqpoint{4.129722in}{2.488647in}}%
\pgfpathlineto{\pgfqpoint{0.734747in}{2.488647in}}%
\pgfpathclose%
\pgfusepath{fill}%
\end{pgfscope}%
\begin{pgfscope}%
\pgfsetbuttcap%
\pgfsetroundjoin%
\definecolor{currentfill}{rgb}{0.150000,0.150000,0.150000}%
\pgfsetfillcolor{currentfill}%
\pgfsetlinewidth{1.003750pt}%
\definecolor{currentstroke}{rgb}{0.150000,0.150000,0.150000}%
\pgfsetstrokecolor{currentstroke}%
\pgfsetdash{}{0pt}%
\pgfsys@defobject{currentmarker}{\pgfqpoint{0.000000in}{-0.083333in}}{\pgfqpoint{0.000000in}{0.000000in}}{%
\pgfpathmoveto{\pgfqpoint{0.000000in}{0.000000in}}%
\pgfpathlineto{\pgfqpoint{0.000000in}{-0.083333in}}%
\pgfusepath{stroke,fill}%
}%
\begin{pgfscope}%
\pgfsys@transformshift{0.888721in}{0.594647in}%
\pgfsys@useobject{currentmarker}{}%
\end{pgfscope}%
\end{pgfscope}%
\begin{pgfscope}%
\definecolor{textcolor}{rgb}{0.150000,0.150000,0.150000}%
\pgfsetstrokecolor{textcolor}%
\pgfsetfillcolor{textcolor}%
\pgftext[x=0.888721in,y=0.414091in,,top]{\color{textcolor}\sffamily\fontsize{10.000000}{12.000000}\selectfont \(\displaystyle 0\)}%
\end{pgfscope}%
\begin{pgfscope}%
\pgfsetbuttcap%
\pgfsetroundjoin%
\definecolor{currentfill}{rgb}{0.150000,0.150000,0.150000}%
\pgfsetfillcolor{currentfill}%
\pgfsetlinewidth{1.003750pt}%
\definecolor{currentstroke}{rgb}{0.150000,0.150000,0.150000}%
\pgfsetstrokecolor{currentstroke}%
\pgfsetdash{}{0pt}%
\pgfsys@defobject{currentmarker}{\pgfqpoint{0.000000in}{-0.083333in}}{\pgfqpoint{0.000000in}{0.000000in}}{%
\pgfpathmoveto{\pgfqpoint{0.000000in}{0.000000in}}%
\pgfpathlineto{\pgfqpoint{0.000000in}{-0.083333in}}%
\pgfusepath{stroke,fill}%
}%
\begin{pgfscope}%
\pgfsys@transformshift{1.574651in}{0.594647in}%
\pgfsys@useobject{currentmarker}{}%
\end{pgfscope}%
\end{pgfscope}%
\begin{pgfscope}%
\definecolor{textcolor}{rgb}{0.150000,0.150000,0.150000}%
\pgfsetstrokecolor{textcolor}%
\pgfsetfillcolor{textcolor}%
\pgftext[x=1.574651in,y=0.414091in,,top]{\color{textcolor}\sffamily\fontsize{10.000000}{12.000000}\selectfont \(\displaystyle 2000\)}%
\end{pgfscope}%
\begin{pgfscope}%
\pgfsetbuttcap%
\pgfsetroundjoin%
\definecolor{currentfill}{rgb}{0.150000,0.150000,0.150000}%
\pgfsetfillcolor{currentfill}%
\pgfsetlinewidth{1.003750pt}%
\definecolor{currentstroke}{rgb}{0.150000,0.150000,0.150000}%
\pgfsetstrokecolor{currentstroke}%
\pgfsetdash{}{0pt}%
\pgfsys@defobject{currentmarker}{\pgfqpoint{0.000000in}{-0.083333in}}{\pgfqpoint{0.000000in}{0.000000in}}{%
\pgfpathmoveto{\pgfqpoint{0.000000in}{0.000000in}}%
\pgfpathlineto{\pgfqpoint{0.000000in}{-0.083333in}}%
\pgfusepath{stroke,fill}%
}%
\begin{pgfscope}%
\pgfsys@transformshift{2.260581in}{0.594647in}%
\pgfsys@useobject{currentmarker}{}%
\end{pgfscope}%
\end{pgfscope}%
\begin{pgfscope}%
\definecolor{textcolor}{rgb}{0.150000,0.150000,0.150000}%
\pgfsetstrokecolor{textcolor}%
\pgfsetfillcolor{textcolor}%
\pgftext[x=2.260581in,y=0.414091in,,top]{\color{textcolor}\sffamily\fontsize{10.000000}{12.000000}\selectfont \(\displaystyle 4000\)}%
\end{pgfscope}%
\begin{pgfscope}%
\pgfsetbuttcap%
\pgfsetroundjoin%
\definecolor{currentfill}{rgb}{0.150000,0.150000,0.150000}%
\pgfsetfillcolor{currentfill}%
\pgfsetlinewidth{1.003750pt}%
\definecolor{currentstroke}{rgb}{0.150000,0.150000,0.150000}%
\pgfsetstrokecolor{currentstroke}%
\pgfsetdash{}{0pt}%
\pgfsys@defobject{currentmarker}{\pgfqpoint{0.000000in}{-0.083333in}}{\pgfqpoint{0.000000in}{0.000000in}}{%
\pgfpathmoveto{\pgfqpoint{0.000000in}{0.000000in}}%
\pgfpathlineto{\pgfqpoint{0.000000in}{-0.083333in}}%
\pgfusepath{stroke,fill}%
}%
\begin{pgfscope}%
\pgfsys@transformshift{2.946510in}{0.594647in}%
\pgfsys@useobject{currentmarker}{}%
\end{pgfscope}%
\end{pgfscope}%
\begin{pgfscope}%
\definecolor{textcolor}{rgb}{0.150000,0.150000,0.150000}%
\pgfsetstrokecolor{textcolor}%
\pgfsetfillcolor{textcolor}%
\pgftext[x=2.946510in,y=0.414091in,,top]{\color{textcolor}\sffamily\fontsize{10.000000}{12.000000}\selectfont \(\displaystyle 6000\)}%
\end{pgfscope}%
\begin{pgfscope}%
\pgfsetbuttcap%
\pgfsetroundjoin%
\definecolor{currentfill}{rgb}{0.150000,0.150000,0.150000}%
\pgfsetfillcolor{currentfill}%
\pgfsetlinewidth{1.003750pt}%
\definecolor{currentstroke}{rgb}{0.150000,0.150000,0.150000}%
\pgfsetstrokecolor{currentstroke}%
\pgfsetdash{}{0pt}%
\pgfsys@defobject{currentmarker}{\pgfqpoint{0.000000in}{-0.083333in}}{\pgfqpoint{0.000000in}{0.000000in}}{%
\pgfpathmoveto{\pgfqpoint{0.000000in}{0.000000in}}%
\pgfpathlineto{\pgfqpoint{0.000000in}{-0.083333in}}%
\pgfusepath{stroke,fill}%
}%
\begin{pgfscope}%
\pgfsys@transformshift{3.632440in}{0.594647in}%
\pgfsys@useobject{currentmarker}{}%
\end{pgfscope}%
\end{pgfscope}%
\begin{pgfscope}%
\definecolor{textcolor}{rgb}{0.150000,0.150000,0.150000}%
\pgfsetstrokecolor{textcolor}%
\pgfsetfillcolor{textcolor}%
\pgftext[x=3.632440in,y=0.414091in,,top]{\color{textcolor}\sffamily\fontsize{10.000000}{12.000000}\selectfont \(\displaystyle 8000\)}%
\end{pgfscope}%
\begin{pgfscope}%
\definecolor{textcolor}{rgb}{0.150000,0.150000,0.150000}%
\pgfsetstrokecolor{textcolor}%
\pgfsetfillcolor{textcolor}%
\pgftext[x=2.432235in,y=0.231252in,,top]{\color{textcolor}\sffamily\fontsize{11.000000}{13.200000}\selectfont Train Data (\# samples)}%
\end{pgfscope}%
\begin{pgfscope}%
\pgfsetbuttcap%
\pgfsetroundjoin%
\definecolor{currentfill}{rgb}{0.150000,0.150000,0.150000}%
\pgfsetfillcolor{currentfill}%
\pgfsetlinewidth{1.003750pt}%
\definecolor{currentstroke}{rgb}{0.150000,0.150000,0.150000}%
\pgfsetstrokecolor{currentstroke}%
\pgfsetdash{}{0pt}%
\pgfsys@defobject{currentmarker}{\pgfqpoint{-0.083333in}{0.000000in}}{\pgfqpoint{0.000000in}{0.000000in}}{%
\pgfpathmoveto{\pgfqpoint{0.000000in}{0.000000in}}%
\pgfpathlineto{\pgfqpoint{-0.083333in}{0.000000in}}%
\pgfusepath{stroke,fill}%
}%
\begin{pgfscope}%
\pgfsys@transformshift{0.734747in}{1.082522in}%
\pgfsys@useobject{currentmarker}{}%
\end{pgfscope}%
\end{pgfscope}%
\begin{pgfscope}%
\definecolor{textcolor}{rgb}{0.150000,0.150000,0.150000}%
\pgfsetstrokecolor{textcolor}%
\pgfsetfillcolor{textcolor}%
\pgftext[x=0.266189in,y=1.032380in,left,base]{\color{textcolor}\sffamily\fontsize{10.000000}{12.000000}\selectfont \(\displaystyle 10^{-2}\)}%
\end{pgfscope}%
\begin{pgfscope}%
\pgfsetbuttcap%
\pgfsetroundjoin%
\definecolor{currentfill}{rgb}{0.150000,0.150000,0.150000}%
\pgfsetfillcolor{currentfill}%
\pgfsetlinewidth{1.003750pt}%
\definecolor{currentstroke}{rgb}{0.150000,0.150000,0.150000}%
\pgfsetstrokecolor{currentstroke}%
\pgfsetdash{}{0pt}%
\pgfsys@defobject{currentmarker}{\pgfqpoint{-0.083333in}{0.000000in}}{\pgfqpoint{0.000000in}{0.000000in}}{%
\pgfpathmoveto{\pgfqpoint{0.000000in}{0.000000in}}%
\pgfpathlineto{\pgfqpoint{-0.083333in}{0.000000in}}%
\pgfusepath{stroke,fill}%
}%
\begin{pgfscope}%
\pgfsys@transformshift{0.734747in}{1.611253in}%
\pgfsys@useobject{currentmarker}{}%
\end{pgfscope}%
\end{pgfscope}%
\begin{pgfscope}%
\definecolor{textcolor}{rgb}{0.150000,0.150000,0.150000}%
\pgfsetstrokecolor{textcolor}%
\pgfsetfillcolor{textcolor}%
\pgftext[x=0.266189in,y=1.561111in,left,base]{\color{textcolor}\sffamily\fontsize{10.000000}{12.000000}\selectfont \(\displaystyle 10^{-1}\)}%
\end{pgfscope}%
\begin{pgfscope}%
\pgfsetbuttcap%
\pgfsetroundjoin%
\definecolor{currentfill}{rgb}{0.150000,0.150000,0.150000}%
\pgfsetfillcolor{currentfill}%
\pgfsetlinewidth{1.003750pt}%
\definecolor{currentstroke}{rgb}{0.150000,0.150000,0.150000}%
\pgfsetstrokecolor{currentstroke}%
\pgfsetdash{}{0pt}%
\pgfsys@defobject{currentmarker}{\pgfqpoint{-0.083333in}{0.000000in}}{\pgfqpoint{0.000000in}{0.000000in}}{%
\pgfpathmoveto{\pgfqpoint{0.000000in}{0.000000in}}%
\pgfpathlineto{\pgfqpoint{-0.083333in}{0.000000in}}%
\pgfusepath{stroke,fill}%
}%
\begin{pgfscope}%
\pgfsys@transformshift{0.734747in}{2.139985in}%
\pgfsys@useobject{currentmarker}{}%
\end{pgfscope}%
\end{pgfscope}%
\begin{pgfscope}%
\definecolor{textcolor}{rgb}{0.150000,0.150000,0.150000}%
\pgfsetstrokecolor{textcolor}%
\pgfsetfillcolor{textcolor}%
\pgftext[x=0.352995in,y=2.089842in,left,base]{\color{textcolor}\sffamily\fontsize{10.000000}{12.000000}\selectfont \(\displaystyle 10^{0}\)}%
\end{pgfscope}%
\begin{pgfscope}%
\pgfsetbuttcap%
\pgfsetroundjoin%
\definecolor{currentfill}{rgb}{0.150000,0.150000,0.150000}%
\pgfsetfillcolor{currentfill}%
\pgfsetlinewidth{0.501875pt}%
\definecolor{currentstroke}{rgb}{0.150000,0.150000,0.150000}%
\pgfsetstrokecolor{currentstroke}%
\pgfsetdash{}{0pt}%
\pgfsys@defobject{currentmarker}{\pgfqpoint{-0.041667in}{0.000000in}}{\pgfqpoint{0.000000in}{0.000000in}}{%
\pgfpathmoveto{\pgfqpoint{0.000000in}{0.000000in}}%
\pgfpathlineto{\pgfqpoint{-0.041667in}{0.000000in}}%
\pgfusepath{stroke,fill}%
}%
\begin{pgfscope}%
\pgfsys@transformshift{0.734747in}{0.712955in}%
\pgfsys@useobject{currentmarker}{}%
\end{pgfscope}%
\end{pgfscope}%
\begin{pgfscope}%
\pgfsetbuttcap%
\pgfsetroundjoin%
\definecolor{currentfill}{rgb}{0.150000,0.150000,0.150000}%
\pgfsetfillcolor{currentfill}%
\pgfsetlinewidth{0.501875pt}%
\definecolor{currentstroke}{rgb}{0.150000,0.150000,0.150000}%
\pgfsetstrokecolor{currentstroke}%
\pgfsetdash{}{0pt}%
\pgfsys@defobject{currentmarker}{\pgfqpoint{-0.041667in}{0.000000in}}{\pgfqpoint{0.000000in}{0.000000in}}{%
\pgfpathmoveto{\pgfqpoint{0.000000in}{0.000000in}}%
\pgfpathlineto{\pgfqpoint{-0.041667in}{0.000000in}}%
\pgfusepath{stroke,fill}%
}%
\begin{pgfscope}%
\pgfsys@transformshift{0.734747in}{0.806060in}%
\pgfsys@useobject{currentmarker}{}%
\end{pgfscope}%
\end{pgfscope}%
\begin{pgfscope}%
\pgfsetbuttcap%
\pgfsetroundjoin%
\definecolor{currentfill}{rgb}{0.150000,0.150000,0.150000}%
\pgfsetfillcolor{currentfill}%
\pgfsetlinewidth{0.501875pt}%
\definecolor{currentstroke}{rgb}{0.150000,0.150000,0.150000}%
\pgfsetstrokecolor{currentstroke}%
\pgfsetdash{}{0pt}%
\pgfsys@defobject{currentmarker}{\pgfqpoint{-0.041667in}{0.000000in}}{\pgfqpoint{0.000000in}{0.000000in}}{%
\pgfpathmoveto{\pgfqpoint{0.000000in}{0.000000in}}%
\pgfpathlineto{\pgfqpoint{-0.041667in}{0.000000in}}%
\pgfusepath{stroke,fill}%
}%
\begin{pgfscope}%
\pgfsys@transformshift{0.734747in}{0.872119in}%
\pgfsys@useobject{currentmarker}{}%
\end{pgfscope}%
\end{pgfscope}%
\begin{pgfscope}%
\pgfsetbuttcap%
\pgfsetroundjoin%
\definecolor{currentfill}{rgb}{0.150000,0.150000,0.150000}%
\pgfsetfillcolor{currentfill}%
\pgfsetlinewidth{0.501875pt}%
\definecolor{currentstroke}{rgb}{0.150000,0.150000,0.150000}%
\pgfsetstrokecolor{currentstroke}%
\pgfsetdash{}{0pt}%
\pgfsys@defobject{currentmarker}{\pgfqpoint{-0.041667in}{0.000000in}}{\pgfqpoint{0.000000in}{0.000000in}}{%
\pgfpathmoveto{\pgfqpoint{0.000000in}{0.000000in}}%
\pgfpathlineto{\pgfqpoint{-0.041667in}{0.000000in}}%
\pgfusepath{stroke,fill}%
}%
\begin{pgfscope}%
\pgfsys@transformshift{0.734747in}{0.923358in}%
\pgfsys@useobject{currentmarker}{}%
\end{pgfscope}%
\end{pgfscope}%
\begin{pgfscope}%
\pgfsetbuttcap%
\pgfsetroundjoin%
\definecolor{currentfill}{rgb}{0.150000,0.150000,0.150000}%
\pgfsetfillcolor{currentfill}%
\pgfsetlinewidth{0.501875pt}%
\definecolor{currentstroke}{rgb}{0.150000,0.150000,0.150000}%
\pgfsetstrokecolor{currentstroke}%
\pgfsetdash{}{0pt}%
\pgfsys@defobject{currentmarker}{\pgfqpoint{-0.041667in}{0.000000in}}{\pgfqpoint{0.000000in}{0.000000in}}{%
\pgfpathmoveto{\pgfqpoint{0.000000in}{0.000000in}}%
\pgfpathlineto{\pgfqpoint{-0.041667in}{0.000000in}}%
\pgfusepath{stroke,fill}%
}%
\begin{pgfscope}%
\pgfsys@transformshift{0.734747in}{0.965224in}%
\pgfsys@useobject{currentmarker}{}%
\end{pgfscope}%
\end{pgfscope}%
\begin{pgfscope}%
\pgfsetbuttcap%
\pgfsetroundjoin%
\definecolor{currentfill}{rgb}{0.150000,0.150000,0.150000}%
\pgfsetfillcolor{currentfill}%
\pgfsetlinewidth{0.501875pt}%
\definecolor{currentstroke}{rgb}{0.150000,0.150000,0.150000}%
\pgfsetstrokecolor{currentstroke}%
\pgfsetdash{}{0pt}%
\pgfsys@defobject{currentmarker}{\pgfqpoint{-0.041667in}{0.000000in}}{\pgfqpoint{0.000000in}{0.000000in}}{%
\pgfpathmoveto{\pgfqpoint{0.000000in}{0.000000in}}%
\pgfpathlineto{\pgfqpoint{-0.041667in}{0.000000in}}%
\pgfusepath{stroke,fill}%
}%
\begin{pgfscope}%
\pgfsys@transformshift{0.734747in}{1.000621in}%
\pgfsys@useobject{currentmarker}{}%
\end{pgfscope}%
\end{pgfscope}%
\begin{pgfscope}%
\pgfsetbuttcap%
\pgfsetroundjoin%
\definecolor{currentfill}{rgb}{0.150000,0.150000,0.150000}%
\pgfsetfillcolor{currentfill}%
\pgfsetlinewidth{0.501875pt}%
\definecolor{currentstroke}{rgb}{0.150000,0.150000,0.150000}%
\pgfsetstrokecolor{currentstroke}%
\pgfsetdash{}{0pt}%
\pgfsys@defobject{currentmarker}{\pgfqpoint{-0.041667in}{0.000000in}}{\pgfqpoint{0.000000in}{0.000000in}}{%
\pgfpathmoveto{\pgfqpoint{0.000000in}{0.000000in}}%
\pgfpathlineto{\pgfqpoint{-0.041667in}{0.000000in}}%
\pgfusepath{stroke,fill}%
}%
\begin{pgfscope}%
\pgfsys@transformshift{0.734747in}{1.031283in}%
\pgfsys@useobject{currentmarker}{}%
\end{pgfscope}%
\end{pgfscope}%
\begin{pgfscope}%
\pgfsetbuttcap%
\pgfsetroundjoin%
\definecolor{currentfill}{rgb}{0.150000,0.150000,0.150000}%
\pgfsetfillcolor{currentfill}%
\pgfsetlinewidth{0.501875pt}%
\definecolor{currentstroke}{rgb}{0.150000,0.150000,0.150000}%
\pgfsetstrokecolor{currentstroke}%
\pgfsetdash{}{0pt}%
\pgfsys@defobject{currentmarker}{\pgfqpoint{-0.041667in}{0.000000in}}{\pgfqpoint{0.000000in}{0.000000in}}{%
\pgfpathmoveto{\pgfqpoint{0.000000in}{0.000000in}}%
\pgfpathlineto{\pgfqpoint{-0.041667in}{0.000000in}}%
\pgfusepath{stroke,fill}%
}%
\begin{pgfscope}%
\pgfsys@transformshift{0.734747in}{1.058329in}%
\pgfsys@useobject{currentmarker}{}%
\end{pgfscope}%
\end{pgfscope}%
\begin{pgfscope}%
\pgfsetbuttcap%
\pgfsetroundjoin%
\definecolor{currentfill}{rgb}{0.150000,0.150000,0.150000}%
\pgfsetfillcolor{currentfill}%
\pgfsetlinewidth{0.501875pt}%
\definecolor{currentstroke}{rgb}{0.150000,0.150000,0.150000}%
\pgfsetstrokecolor{currentstroke}%
\pgfsetdash{}{0pt}%
\pgfsys@defobject{currentmarker}{\pgfqpoint{-0.041667in}{0.000000in}}{\pgfqpoint{0.000000in}{0.000000in}}{%
\pgfpathmoveto{\pgfqpoint{0.000000in}{0.000000in}}%
\pgfpathlineto{\pgfqpoint{-0.041667in}{0.000000in}}%
\pgfusepath{stroke,fill}%
}%
\begin{pgfscope}%
\pgfsys@transformshift{0.734747in}{1.241686in}%
\pgfsys@useobject{currentmarker}{}%
\end{pgfscope}%
\end{pgfscope}%
\begin{pgfscope}%
\pgfsetbuttcap%
\pgfsetroundjoin%
\definecolor{currentfill}{rgb}{0.150000,0.150000,0.150000}%
\pgfsetfillcolor{currentfill}%
\pgfsetlinewidth{0.501875pt}%
\definecolor{currentstroke}{rgb}{0.150000,0.150000,0.150000}%
\pgfsetstrokecolor{currentstroke}%
\pgfsetdash{}{0pt}%
\pgfsys@defobject{currentmarker}{\pgfqpoint{-0.041667in}{0.000000in}}{\pgfqpoint{0.000000in}{0.000000in}}{%
\pgfpathmoveto{\pgfqpoint{0.000000in}{0.000000in}}%
\pgfpathlineto{\pgfqpoint{-0.041667in}{0.000000in}}%
\pgfusepath{stroke,fill}%
}%
\begin{pgfscope}%
\pgfsys@transformshift{0.734747in}{1.334791in}%
\pgfsys@useobject{currentmarker}{}%
\end{pgfscope}%
\end{pgfscope}%
\begin{pgfscope}%
\pgfsetbuttcap%
\pgfsetroundjoin%
\definecolor{currentfill}{rgb}{0.150000,0.150000,0.150000}%
\pgfsetfillcolor{currentfill}%
\pgfsetlinewidth{0.501875pt}%
\definecolor{currentstroke}{rgb}{0.150000,0.150000,0.150000}%
\pgfsetstrokecolor{currentstroke}%
\pgfsetdash{}{0pt}%
\pgfsys@defobject{currentmarker}{\pgfqpoint{-0.041667in}{0.000000in}}{\pgfqpoint{0.000000in}{0.000000in}}{%
\pgfpathmoveto{\pgfqpoint{0.000000in}{0.000000in}}%
\pgfpathlineto{\pgfqpoint{-0.041667in}{0.000000in}}%
\pgfusepath{stroke,fill}%
}%
\begin{pgfscope}%
\pgfsys@transformshift{0.734747in}{1.400850in}%
\pgfsys@useobject{currentmarker}{}%
\end{pgfscope}%
\end{pgfscope}%
\begin{pgfscope}%
\pgfsetbuttcap%
\pgfsetroundjoin%
\definecolor{currentfill}{rgb}{0.150000,0.150000,0.150000}%
\pgfsetfillcolor{currentfill}%
\pgfsetlinewidth{0.501875pt}%
\definecolor{currentstroke}{rgb}{0.150000,0.150000,0.150000}%
\pgfsetstrokecolor{currentstroke}%
\pgfsetdash{}{0pt}%
\pgfsys@defobject{currentmarker}{\pgfqpoint{-0.041667in}{0.000000in}}{\pgfqpoint{0.000000in}{0.000000in}}{%
\pgfpathmoveto{\pgfqpoint{0.000000in}{0.000000in}}%
\pgfpathlineto{\pgfqpoint{-0.041667in}{0.000000in}}%
\pgfusepath{stroke,fill}%
}%
\begin{pgfscope}%
\pgfsys@transformshift{0.734747in}{1.452090in}%
\pgfsys@useobject{currentmarker}{}%
\end{pgfscope}%
\end{pgfscope}%
\begin{pgfscope}%
\pgfsetbuttcap%
\pgfsetroundjoin%
\definecolor{currentfill}{rgb}{0.150000,0.150000,0.150000}%
\pgfsetfillcolor{currentfill}%
\pgfsetlinewidth{0.501875pt}%
\definecolor{currentstroke}{rgb}{0.150000,0.150000,0.150000}%
\pgfsetstrokecolor{currentstroke}%
\pgfsetdash{}{0pt}%
\pgfsys@defobject{currentmarker}{\pgfqpoint{-0.041667in}{0.000000in}}{\pgfqpoint{0.000000in}{0.000000in}}{%
\pgfpathmoveto{\pgfqpoint{0.000000in}{0.000000in}}%
\pgfpathlineto{\pgfqpoint{-0.041667in}{0.000000in}}%
\pgfusepath{stroke,fill}%
}%
\begin{pgfscope}%
\pgfsys@transformshift{0.734747in}{1.493955in}%
\pgfsys@useobject{currentmarker}{}%
\end{pgfscope}%
\end{pgfscope}%
\begin{pgfscope}%
\pgfsetbuttcap%
\pgfsetroundjoin%
\definecolor{currentfill}{rgb}{0.150000,0.150000,0.150000}%
\pgfsetfillcolor{currentfill}%
\pgfsetlinewidth{0.501875pt}%
\definecolor{currentstroke}{rgb}{0.150000,0.150000,0.150000}%
\pgfsetstrokecolor{currentstroke}%
\pgfsetdash{}{0pt}%
\pgfsys@defobject{currentmarker}{\pgfqpoint{-0.041667in}{0.000000in}}{\pgfqpoint{0.000000in}{0.000000in}}{%
\pgfpathmoveto{\pgfqpoint{0.000000in}{0.000000in}}%
\pgfpathlineto{\pgfqpoint{-0.041667in}{0.000000in}}%
\pgfusepath{stroke,fill}%
}%
\begin{pgfscope}%
\pgfsys@transformshift{0.734747in}{1.529352in}%
\pgfsys@useobject{currentmarker}{}%
\end{pgfscope}%
\end{pgfscope}%
\begin{pgfscope}%
\pgfsetbuttcap%
\pgfsetroundjoin%
\definecolor{currentfill}{rgb}{0.150000,0.150000,0.150000}%
\pgfsetfillcolor{currentfill}%
\pgfsetlinewidth{0.501875pt}%
\definecolor{currentstroke}{rgb}{0.150000,0.150000,0.150000}%
\pgfsetstrokecolor{currentstroke}%
\pgfsetdash{}{0pt}%
\pgfsys@defobject{currentmarker}{\pgfqpoint{-0.041667in}{0.000000in}}{\pgfqpoint{0.000000in}{0.000000in}}{%
\pgfpathmoveto{\pgfqpoint{0.000000in}{0.000000in}}%
\pgfpathlineto{\pgfqpoint{-0.041667in}{0.000000in}}%
\pgfusepath{stroke,fill}%
}%
\begin{pgfscope}%
\pgfsys@transformshift{0.734747in}{1.560014in}%
\pgfsys@useobject{currentmarker}{}%
\end{pgfscope}%
\end{pgfscope}%
\begin{pgfscope}%
\pgfsetbuttcap%
\pgfsetroundjoin%
\definecolor{currentfill}{rgb}{0.150000,0.150000,0.150000}%
\pgfsetfillcolor{currentfill}%
\pgfsetlinewidth{0.501875pt}%
\definecolor{currentstroke}{rgb}{0.150000,0.150000,0.150000}%
\pgfsetstrokecolor{currentstroke}%
\pgfsetdash{}{0pt}%
\pgfsys@defobject{currentmarker}{\pgfqpoint{-0.041667in}{0.000000in}}{\pgfqpoint{0.000000in}{0.000000in}}{%
\pgfpathmoveto{\pgfqpoint{0.000000in}{0.000000in}}%
\pgfpathlineto{\pgfqpoint{-0.041667in}{0.000000in}}%
\pgfusepath{stroke,fill}%
}%
\begin{pgfscope}%
\pgfsys@transformshift{0.734747in}{1.587060in}%
\pgfsys@useobject{currentmarker}{}%
\end{pgfscope}%
\end{pgfscope}%
\begin{pgfscope}%
\pgfsetbuttcap%
\pgfsetroundjoin%
\definecolor{currentfill}{rgb}{0.150000,0.150000,0.150000}%
\pgfsetfillcolor{currentfill}%
\pgfsetlinewidth{0.501875pt}%
\definecolor{currentstroke}{rgb}{0.150000,0.150000,0.150000}%
\pgfsetstrokecolor{currentstroke}%
\pgfsetdash{}{0pt}%
\pgfsys@defobject{currentmarker}{\pgfqpoint{-0.041667in}{0.000000in}}{\pgfqpoint{0.000000in}{0.000000in}}{%
\pgfpathmoveto{\pgfqpoint{0.000000in}{0.000000in}}%
\pgfpathlineto{\pgfqpoint{-0.041667in}{0.000000in}}%
\pgfusepath{stroke,fill}%
}%
\begin{pgfscope}%
\pgfsys@transformshift{0.734747in}{1.770417in}%
\pgfsys@useobject{currentmarker}{}%
\end{pgfscope}%
\end{pgfscope}%
\begin{pgfscope}%
\pgfsetbuttcap%
\pgfsetroundjoin%
\definecolor{currentfill}{rgb}{0.150000,0.150000,0.150000}%
\pgfsetfillcolor{currentfill}%
\pgfsetlinewidth{0.501875pt}%
\definecolor{currentstroke}{rgb}{0.150000,0.150000,0.150000}%
\pgfsetstrokecolor{currentstroke}%
\pgfsetdash{}{0pt}%
\pgfsys@defobject{currentmarker}{\pgfqpoint{-0.041667in}{0.000000in}}{\pgfqpoint{0.000000in}{0.000000in}}{%
\pgfpathmoveto{\pgfqpoint{0.000000in}{0.000000in}}%
\pgfpathlineto{\pgfqpoint{-0.041667in}{0.000000in}}%
\pgfusepath{stroke,fill}%
}%
\begin{pgfscope}%
\pgfsys@transformshift{0.734747in}{1.863522in}%
\pgfsys@useobject{currentmarker}{}%
\end{pgfscope}%
\end{pgfscope}%
\begin{pgfscope}%
\pgfsetbuttcap%
\pgfsetroundjoin%
\definecolor{currentfill}{rgb}{0.150000,0.150000,0.150000}%
\pgfsetfillcolor{currentfill}%
\pgfsetlinewidth{0.501875pt}%
\definecolor{currentstroke}{rgb}{0.150000,0.150000,0.150000}%
\pgfsetstrokecolor{currentstroke}%
\pgfsetdash{}{0pt}%
\pgfsys@defobject{currentmarker}{\pgfqpoint{-0.041667in}{0.000000in}}{\pgfqpoint{0.000000in}{0.000000in}}{%
\pgfpathmoveto{\pgfqpoint{0.000000in}{0.000000in}}%
\pgfpathlineto{\pgfqpoint{-0.041667in}{0.000000in}}%
\pgfusepath{stroke,fill}%
}%
\begin{pgfscope}%
\pgfsys@transformshift{0.734747in}{1.929581in}%
\pgfsys@useobject{currentmarker}{}%
\end{pgfscope}%
\end{pgfscope}%
\begin{pgfscope}%
\pgfsetbuttcap%
\pgfsetroundjoin%
\definecolor{currentfill}{rgb}{0.150000,0.150000,0.150000}%
\pgfsetfillcolor{currentfill}%
\pgfsetlinewidth{0.501875pt}%
\definecolor{currentstroke}{rgb}{0.150000,0.150000,0.150000}%
\pgfsetstrokecolor{currentstroke}%
\pgfsetdash{}{0pt}%
\pgfsys@defobject{currentmarker}{\pgfqpoint{-0.041667in}{0.000000in}}{\pgfqpoint{0.000000in}{0.000000in}}{%
\pgfpathmoveto{\pgfqpoint{0.000000in}{0.000000in}}%
\pgfpathlineto{\pgfqpoint{-0.041667in}{0.000000in}}%
\pgfusepath{stroke,fill}%
}%
\begin{pgfscope}%
\pgfsys@transformshift{0.734747in}{1.980821in}%
\pgfsys@useobject{currentmarker}{}%
\end{pgfscope}%
\end{pgfscope}%
\begin{pgfscope}%
\pgfsetbuttcap%
\pgfsetroundjoin%
\definecolor{currentfill}{rgb}{0.150000,0.150000,0.150000}%
\pgfsetfillcolor{currentfill}%
\pgfsetlinewidth{0.501875pt}%
\definecolor{currentstroke}{rgb}{0.150000,0.150000,0.150000}%
\pgfsetstrokecolor{currentstroke}%
\pgfsetdash{}{0pt}%
\pgfsys@defobject{currentmarker}{\pgfqpoint{-0.041667in}{0.000000in}}{\pgfqpoint{0.000000in}{0.000000in}}{%
\pgfpathmoveto{\pgfqpoint{0.000000in}{0.000000in}}%
\pgfpathlineto{\pgfqpoint{-0.041667in}{0.000000in}}%
\pgfusepath{stroke,fill}%
}%
\begin{pgfscope}%
\pgfsys@transformshift{0.734747in}{2.022686in}%
\pgfsys@useobject{currentmarker}{}%
\end{pgfscope}%
\end{pgfscope}%
\begin{pgfscope}%
\pgfsetbuttcap%
\pgfsetroundjoin%
\definecolor{currentfill}{rgb}{0.150000,0.150000,0.150000}%
\pgfsetfillcolor{currentfill}%
\pgfsetlinewidth{0.501875pt}%
\definecolor{currentstroke}{rgb}{0.150000,0.150000,0.150000}%
\pgfsetstrokecolor{currentstroke}%
\pgfsetdash{}{0pt}%
\pgfsys@defobject{currentmarker}{\pgfqpoint{-0.041667in}{0.000000in}}{\pgfqpoint{0.000000in}{0.000000in}}{%
\pgfpathmoveto{\pgfqpoint{0.000000in}{0.000000in}}%
\pgfpathlineto{\pgfqpoint{-0.041667in}{0.000000in}}%
\pgfusepath{stroke,fill}%
}%
\begin{pgfscope}%
\pgfsys@transformshift{0.734747in}{2.058083in}%
\pgfsys@useobject{currentmarker}{}%
\end{pgfscope}%
\end{pgfscope}%
\begin{pgfscope}%
\pgfsetbuttcap%
\pgfsetroundjoin%
\definecolor{currentfill}{rgb}{0.150000,0.150000,0.150000}%
\pgfsetfillcolor{currentfill}%
\pgfsetlinewidth{0.501875pt}%
\definecolor{currentstroke}{rgb}{0.150000,0.150000,0.150000}%
\pgfsetstrokecolor{currentstroke}%
\pgfsetdash{}{0pt}%
\pgfsys@defobject{currentmarker}{\pgfqpoint{-0.041667in}{0.000000in}}{\pgfqpoint{0.000000in}{0.000000in}}{%
\pgfpathmoveto{\pgfqpoint{0.000000in}{0.000000in}}%
\pgfpathlineto{\pgfqpoint{-0.041667in}{0.000000in}}%
\pgfusepath{stroke,fill}%
}%
\begin{pgfscope}%
\pgfsys@transformshift{0.734747in}{2.088745in}%
\pgfsys@useobject{currentmarker}{}%
\end{pgfscope}%
\end{pgfscope}%
\begin{pgfscope}%
\pgfsetbuttcap%
\pgfsetroundjoin%
\definecolor{currentfill}{rgb}{0.150000,0.150000,0.150000}%
\pgfsetfillcolor{currentfill}%
\pgfsetlinewidth{0.501875pt}%
\definecolor{currentstroke}{rgb}{0.150000,0.150000,0.150000}%
\pgfsetstrokecolor{currentstroke}%
\pgfsetdash{}{0pt}%
\pgfsys@defobject{currentmarker}{\pgfqpoint{-0.041667in}{0.000000in}}{\pgfqpoint{0.000000in}{0.000000in}}{%
\pgfpathmoveto{\pgfqpoint{0.000000in}{0.000000in}}%
\pgfpathlineto{\pgfqpoint{-0.041667in}{0.000000in}}%
\pgfusepath{stroke,fill}%
}%
\begin{pgfscope}%
\pgfsys@transformshift{0.734747in}{2.115791in}%
\pgfsys@useobject{currentmarker}{}%
\end{pgfscope}%
\end{pgfscope}%
\begin{pgfscope}%
\pgfsetbuttcap%
\pgfsetroundjoin%
\definecolor{currentfill}{rgb}{0.150000,0.150000,0.150000}%
\pgfsetfillcolor{currentfill}%
\pgfsetlinewidth{0.501875pt}%
\definecolor{currentstroke}{rgb}{0.150000,0.150000,0.150000}%
\pgfsetstrokecolor{currentstroke}%
\pgfsetdash{}{0pt}%
\pgfsys@defobject{currentmarker}{\pgfqpoint{-0.041667in}{0.000000in}}{\pgfqpoint{0.000000in}{0.000000in}}{%
\pgfpathmoveto{\pgfqpoint{0.000000in}{0.000000in}}%
\pgfpathlineto{\pgfqpoint{-0.041667in}{0.000000in}}%
\pgfusepath{stroke,fill}%
}%
\begin{pgfscope}%
\pgfsys@transformshift{0.734747in}{2.299149in}%
\pgfsys@useobject{currentmarker}{}%
\end{pgfscope}%
\end{pgfscope}%
\begin{pgfscope}%
\pgfsetbuttcap%
\pgfsetroundjoin%
\definecolor{currentfill}{rgb}{0.150000,0.150000,0.150000}%
\pgfsetfillcolor{currentfill}%
\pgfsetlinewidth{0.501875pt}%
\definecolor{currentstroke}{rgb}{0.150000,0.150000,0.150000}%
\pgfsetstrokecolor{currentstroke}%
\pgfsetdash{}{0pt}%
\pgfsys@defobject{currentmarker}{\pgfqpoint{-0.041667in}{0.000000in}}{\pgfqpoint{0.000000in}{0.000000in}}{%
\pgfpathmoveto{\pgfqpoint{0.000000in}{0.000000in}}%
\pgfpathlineto{\pgfqpoint{-0.041667in}{0.000000in}}%
\pgfusepath{stroke,fill}%
}%
\begin{pgfscope}%
\pgfsys@transformshift{0.734747in}{2.392253in}%
\pgfsys@useobject{currentmarker}{}%
\end{pgfscope}%
\end{pgfscope}%
\begin{pgfscope}%
\pgfsetbuttcap%
\pgfsetroundjoin%
\definecolor{currentfill}{rgb}{0.150000,0.150000,0.150000}%
\pgfsetfillcolor{currentfill}%
\pgfsetlinewidth{0.501875pt}%
\definecolor{currentstroke}{rgb}{0.150000,0.150000,0.150000}%
\pgfsetstrokecolor{currentstroke}%
\pgfsetdash{}{0pt}%
\pgfsys@defobject{currentmarker}{\pgfqpoint{-0.041667in}{0.000000in}}{\pgfqpoint{0.000000in}{0.000000in}}{%
\pgfpathmoveto{\pgfqpoint{0.000000in}{0.000000in}}%
\pgfpathlineto{\pgfqpoint{-0.041667in}{0.000000in}}%
\pgfusepath{stroke,fill}%
}%
\begin{pgfscope}%
\pgfsys@transformshift{0.734747in}{2.458312in}%
\pgfsys@useobject{currentmarker}{}%
\end{pgfscope}%
\end{pgfscope}%
\begin{pgfscope}%
\definecolor{textcolor}{rgb}{0.150000,0.150000,0.150000}%
\pgfsetstrokecolor{textcolor}%
\pgfsetfillcolor{textcolor}%
\pgftext[x=0.210634in,y=1.541647in,,bottom,rotate=90.000000]{\color{textcolor}\sffamily\fontsize{11.000000}{13.200000}\selectfont Log Val. Loss}%
\end{pgfscope}%
\begin{pgfscope}%
\pgfpathrectangle{\pgfqpoint{0.734747in}{0.594647in}}{\pgfqpoint{3.394974in}{1.894001in}} %
\pgfusepath{clip}%
\pgfsetroundcap%
\pgfsetroundjoin%
\pgfsetlinewidth{1.756562pt}%
\definecolor{currentstroke}{rgb}{0.400000,0.760784,0.647059}%
\pgfsetstrokecolor{currentstroke}%
\pgfsetdash{}{0pt}%
\pgfpathmoveto{\pgfqpoint{0.889064in}{2.391406in}}%
\pgfpathlineto{\pgfqpoint{0.889407in}{2.398820in}}%
\pgfpathlineto{\pgfqpoint{0.889750in}{2.350111in}}%
\pgfpathlineto{\pgfqpoint{0.890093in}{2.318367in}}%
\pgfpathlineto{\pgfqpoint{0.890436in}{2.344671in}}%
\pgfpathlineto{\pgfqpoint{0.890779in}{2.341099in}}%
\pgfpathlineto{\pgfqpoint{0.891122in}{2.304816in}}%
\pgfpathlineto{\pgfqpoint{0.891465in}{2.324398in}}%
\pgfpathlineto{\pgfqpoint{0.891808in}{2.370294in}}%
\pgfpathlineto{\pgfqpoint{0.892151in}{2.331151in}}%
\pgfpathlineto{\pgfqpoint{0.895581in}{2.284856in}}%
\pgfpathlineto{\pgfqpoint{0.899010in}{2.251292in}}%
\pgfpathlineto{\pgfqpoint{0.902440in}{2.246820in}}%
\pgfpathlineto{\pgfqpoint{0.905870in}{2.211538in}}%
\pgfpathlineto{\pgfqpoint{0.909299in}{2.216747in}}%
\pgfpathlineto{\pgfqpoint{0.912729in}{2.226290in}}%
\pgfpathlineto{\pgfqpoint{0.916159in}{2.227345in}}%
\pgfpathlineto{\pgfqpoint{0.919588in}{2.215003in}}%
\pgfpathlineto{\pgfqpoint{0.923018in}{2.200530in}}%
\pgfpathlineto{\pgfqpoint{0.957314in}{2.173372in}}%
\pgfpathlineto{\pgfqpoint{0.991611in}{2.066023in}}%
\pgfpathlineto{\pgfqpoint{1.025907in}{2.043926in}}%
\pgfpathlineto{\pgfqpoint{1.060204in}{1.957897in}}%
\pgfpathlineto{\pgfqpoint{1.094500in}{1.915204in}}%
\pgfpathlineto{\pgfqpoint{1.128797in}{1.879504in}}%
\pgfpathlineto{\pgfqpoint{1.163093in}{1.870833in}}%
\pgfpathlineto{\pgfqpoint{1.197390in}{1.814111in}}%
\pgfpathlineto{\pgfqpoint{1.231686in}{1.675760in}}%
\pgfpathlineto{\pgfqpoint{1.265983in}{1.732122in}}%
\pgfpathlineto{\pgfqpoint{1.300279in}{1.794640in}}%
\pgfpathlineto{\pgfqpoint{1.334576in}{1.776080in}}%
\pgfpathlineto{\pgfqpoint{1.368872in}{1.691273in}}%
\pgfpathlineto{\pgfqpoint{1.403169in}{1.734104in}}%
\pgfpathlineto{\pgfqpoint{1.437465in}{1.722780in}}%
\pgfpathlineto{\pgfqpoint{1.471762in}{1.656829in}}%
\pgfpathlineto{\pgfqpoint{1.506058in}{1.680298in}}%
\pgfpathlineto{\pgfqpoint{1.540355in}{1.590382in}}%
\pgfpathlineto{\pgfqpoint{1.574651in}{1.590968in}}%
\pgfpathlineto{\pgfqpoint{1.608948in}{1.644980in}}%
\pgfpathlineto{\pgfqpoint{1.643244in}{1.655181in}}%
\pgfpathlineto{\pgfqpoint{1.677541in}{1.594207in}}%
\pgfpathlineto{\pgfqpoint{1.711837in}{1.514765in}}%
\pgfpathlineto{\pgfqpoint{1.746134in}{1.624386in}}%
\pgfpathlineto{\pgfqpoint{1.780430in}{1.754484in}}%
\pgfpathlineto{\pgfqpoint{1.814726in}{1.542510in}}%
\pgfpathlineto{\pgfqpoint{1.849023in}{1.698477in}}%
\pgfpathlineto{\pgfqpoint{1.883319in}{1.773380in}}%
\pgfpathlineto{\pgfqpoint{1.917616in}{1.652387in}}%
\pgfpathlineto{\pgfqpoint{1.951912in}{1.820971in}}%
\pgfpathlineto{\pgfqpoint{1.986209in}{1.719429in}}%
\pgfpathlineto{\pgfqpoint{2.020505in}{1.567519in}}%
\pgfpathlineto{\pgfqpoint{2.054802in}{1.524601in}}%
\pgfpathlineto{\pgfqpoint{2.089098in}{1.575401in}}%
\pgfpathlineto{\pgfqpoint{2.123395in}{1.435415in}}%
\pgfpathlineto{\pgfqpoint{2.157691in}{1.669352in}}%
\pgfpathlineto{\pgfqpoint{2.191988in}{1.485198in}}%
\pgfpathlineto{\pgfqpoint{2.226284in}{1.484349in}}%
\pgfpathlineto{\pgfqpoint{2.260581in}{1.685034in}}%
\pgfpathlineto{\pgfqpoint{2.294877in}{1.717468in}}%
\pgfpathlineto{\pgfqpoint{2.329174in}{1.532954in}}%
\pgfpathlineto{\pgfqpoint{2.363470in}{1.348983in}}%
\pgfpathlineto{\pgfqpoint{2.397767in}{1.676989in}}%
\pgfpathlineto{\pgfqpoint{2.432063in}{1.484889in}}%
\pgfpathlineto{\pgfqpoint{2.466360in}{1.682355in}}%
\pgfpathlineto{\pgfqpoint{2.500656in}{1.558891in}}%
\pgfpathlineto{\pgfqpoint{2.534953in}{1.662853in}}%
\pgfpathlineto{\pgfqpoint{2.569249in}{1.491662in}}%
\pgfpathlineto{\pgfqpoint{2.603546in}{1.480068in}}%
\pgfpathlineto{\pgfqpoint{2.637842in}{1.852784in}}%
\pgfpathlineto{\pgfqpoint{2.672139in}{1.532603in}}%
\pgfpathlineto{\pgfqpoint{2.706435in}{1.431859in}}%
\pgfpathlineto{\pgfqpoint{2.740732in}{1.442385in}}%
\pgfpathlineto{\pgfqpoint{2.775028in}{1.724207in}}%
\pgfpathlineto{\pgfqpoint{2.809324in}{1.496385in}}%
\pgfpathlineto{\pgfqpoint{2.843621in}{1.444964in}}%
\pgfpathlineto{\pgfqpoint{2.877917in}{1.389417in}}%
\pgfpathlineto{\pgfqpoint{2.912214in}{1.704157in}}%
\pgfpathlineto{\pgfqpoint{2.946510in}{1.472678in}}%
\pgfpathlineto{\pgfqpoint{2.980807in}{1.461657in}}%
\pgfpathlineto{\pgfqpoint{3.015103in}{1.384276in}}%
\pgfpathlineto{\pgfqpoint{3.049400in}{1.767464in}}%
\pgfpathlineto{\pgfqpoint{3.083696in}{1.512749in}}%
\pgfpathlineto{\pgfqpoint{3.117993in}{1.704005in}}%
\pgfpathlineto{\pgfqpoint{3.152289in}{1.716330in}}%
\pgfpathlineto{\pgfqpoint{3.186586in}{1.426386in}}%
\pgfpathlineto{\pgfqpoint{3.220882in}{1.435207in}}%
\pgfpathlineto{\pgfqpoint{3.255179in}{1.359452in}}%
\pgfpathlineto{\pgfqpoint{3.289475in}{1.364771in}}%
\pgfpathlineto{\pgfqpoint{3.323772in}{1.458527in}}%
\pgfpathlineto{\pgfqpoint{3.358068in}{1.833039in}}%
\pgfpathlineto{\pgfqpoint{3.392365in}{1.407164in}}%
\pgfpathlineto{\pgfqpoint{3.426661in}{1.395978in}}%
\pgfpathlineto{\pgfqpoint{3.460958in}{1.718312in}}%
\pgfpathlineto{\pgfqpoint{3.495254in}{1.283427in}}%
\pgfpathlineto{\pgfqpoint{3.529551in}{1.431599in}}%
\pgfpathlineto{\pgfqpoint{3.563847in}{1.457349in}}%
\pgfpathlineto{\pgfqpoint{3.598144in}{1.514233in}}%
\pgfpathlineto{\pgfqpoint{3.632440in}{1.395933in}}%
\pgfpathlineto{\pgfqpoint{3.666737in}{1.419060in}}%
\pgfpathlineto{\pgfqpoint{3.701033in}{1.474689in}}%
\pgfpathlineto{\pgfqpoint{3.735329in}{1.725258in}}%
\pgfpathlineto{\pgfqpoint{3.769626in}{1.402689in}}%
\pgfpathlineto{\pgfqpoint{3.803922in}{1.363251in}}%
\pgfpathlineto{\pgfqpoint{3.838219in}{1.479861in}}%
\pgfpathlineto{\pgfqpoint{3.872515in}{1.737416in}}%
\pgfpathlineto{\pgfqpoint{3.906812in}{1.358177in}}%
\pgfpathlineto{\pgfqpoint{3.941108in}{1.438535in}}%
\pgfpathlineto{\pgfqpoint{3.975405in}{1.558659in}}%
\pgfusepath{stroke}%
\end{pgfscope}%
\begin{pgfscope}%
\pgfpathrectangle{\pgfqpoint{0.734747in}{0.594647in}}{\pgfqpoint{3.394974in}{1.894001in}} %
\pgfusepath{clip}%
\pgfsetroundcap%
\pgfsetroundjoin%
\pgfsetlinewidth{1.756562pt}%
\definecolor{currentstroke}{rgb}{0.988235,0.552941,0.384314}%
\pgfsetstrokecolor{currentstroke}%
\pgfsetdash{}{0pt}%
\pgfpathmoveto{\pgfqpoint{0.889064in}{2.402557in}}%
\pgfpathlineto{\pgfqpoint{0.889407in}{2.389882in}}%
\pgfpathlineto{\pgfqpoint{0.889750in}{2.376484in}}%
\pgfpathlineto{\pgfqpoint{0.890093in}{2.346335in}}%
\pgfpathlineto{\pgfqpoint{0.890436in}{2.302413in}}%
\pgfpathlineto{\pgfqpoint{0.890779in}{2.380100in}}%
\pgfpathlineto{\pgfqpoint{0.891122in}{2.250096in}}%
\pgfpathlineto{\pgfqpoint{0.891465in}{2.280801in}}%
\pgfpathlineto{\pgfqpoint{0.891808in}{2.348890in}}%
\pgfpathlineto{\pgfqpoint{0.892151in}{2.278816in}}%
\pgfpathlineto{\pgfqpoint{0.895581in}{2.267893in}}%
\pgfpathlineto{\pgfqpoint{0.899010in}{2.250888in}}%
\pgfpathlineto{\pgfqpoint{0.902440in}{2.288895in}}%
\pgfpathlineto{\pgfqpoint{0.905870in}{2.252160in}}%
\pgfpathlineto{\pgfqpoint{0.909299in}{2.208274in}}%
\pgfpathlineto{\pgfqpoint{0.912729in}{2.305300in}}%
\pgfpathlineto{\pgfqpoint{0.916159in}{2.209852in}}%
\pgfpathlineto{\pgfqpoint{0.919588in}{2.173178in}}%
\pgfpathlineto{\pgfqpoint{0.923018in}{2.112918in}}%
\pgfpathlineto{\pgfqpoint{0.957314in}{2.143005in}}%
\pgfpathlineto{\pgfqpoint{0.991611in}{2.116938in}}%
\pgfpathlineto{\pgfqpoint{1.025907in}{1.777123in}}%
\pgfpathlineto{\pgfqpoint{1.060204in}{1.583236in}}%
\pgfpathlineto{\pgfqpoint{1.094500in}{1.509542in}}%
\pgfpathlineto{\pgfqpoint{1.128797in}{1.574999in}}%
\pgfpathlineto{\pgfqpoint{1.163093in}{1.477739in}}%
\pgfpathlineto{\pgfqpoint{1.197390in}{1.473941in}}%
\pgfpathlineto{\pgfqpoint{1.231686in}{1.555934in}}%
\pgfpathlineto{\pgfqpoint{1.265983in}{1.710472in}}%
\pgfpathlineto{\pgfqpoint{1.300279in}{1.847113in}}%
\pgfpathlineto{\pgfqpoint{1.334576in}{1.234591in}}%
\pgfpathlineto{\pgfqpoint{1.368872in}{1.637170in}}%
\pgfpathlineto{\pgfqpoint{1.403169in}{1.703366in}}%
\pgfpathlineto{\pgfqpoint{1.437465in}{1.694356in}}%
\pgfpathlineto{\pgfqpoint{1.471762in}{1.372650in}}%
\pgfpathlineto{\pgfqpoint{1.506058in}{1.425055in}}%
\pgfpathlineto{\pgfqpoint{1.540355in}{1.614032in}}%
\pgfpathlineto{\pgfqpoint{1.574651in}{1.893077in}}%
\pgfpathlineto{\pgfqpoint{1.608948in}{1.974767in}}%
\pgfpathlineto{\pgfqpoint{1.643244in}{1.273029in}}%
\pgfpathlineto{\pgfqpoint{1.677541in}{1.151222in}}%
\pgfpathlineto{\pgfqpoint{1.711837in}{1.747756in}}%
\pgfpathlineto{\pgfqpoint{1.746134in}{1.180163in}}%
\pgfpathlineto{\pgfqpoint{1.780430in}{1.444705in}}%
\pgfpathlineto{\pgfqpoint{1.814726in}{1.598020in}}%
\pgfpathlineto{\pgfqpoint{1.849023in}{2.110555in}}%
\pgfpathlineto{\pgfqpoint{1.883319in}{1.845987in}}%
\pgfpathlineto{\pgfqpoint{1.917616in}{1.414465in}}%
\pgfpathlineto{\pgfqpoint{1.951912in}{1.771713in}}%
\pgfpathlineto{\pgfqpoint{1.986209in}{1.487023in}}%
\pgfpathlineto{\pgfqpoint{2.020505in}{1.420969in}}%
\pgfpathlineto{\pgfqpoint{2.054802in}{1.280609in}}%
\pgfpathlineto{\pgfqpoint{2.089098in}{0.680738in}}%
\pgfpathlineto{\pgfqpoint{2.123395in}{1.228133in}}%
\pgfpathlineto{\pgfqpoint{2.157691in}{1.821159in}}%
\pgfpathlineto{\pgfqpoint{2.191988in}{0.959595in}}%
\pgfpathlineto{\pgfqpoint{2.226284in}{1.648371in}}%
\pgfpathlineto{\pgfqpoint{2.260581in}{1.390225in}}%
\pgfpathlineto{\pgfqpoint{2.294877in}{1.026959in}}%
\pgfpathlineto{\pgfqpoint{2.329174in}{1.304518in}}%
\pgfpathlineto{\pgfqpoint{2.363470in}{1.378418in}}%
\pgfpathlineto{\pgfqpoint{2.397767in}{0.943459in}}%
\pgfpathlineto{\pgfqpoint{2.432063in}{1.024744in}}%
\pgfpathlineto{\pgfqpoint{2.466360in}{0.963096in}}%
\pgfpathlineto{\pgfqpoint{2.500656in}{0.877612in}}%
\pgfpathlineto{\pgfqpoint{2.534953in}{1.631991in}}%
\pgfpathlineto{\pgfqpoint{2.569249in}{1.500664in}}%
\pgfpathlineto{\pgfqpoint{2.603546in}{1.155776in}}%
\pgfpathlineto{\pgfqpoint{2.637842in}{0.861231in}}%
\pgfpathlineto{\pgfqpoint{2.672139in}{1.487869in}}%
\pgfpathlineto{\pgfqpoint{2.706435in}{2.007905in}}%
\pgfpathlineto{\pgfqpoint{2.740732in}{1.708576in}}%
\pgfpathlineto{\pgfqpoint{2.775028in}{0.950084in}}%
\pgfpathlineto{\pgfqpoint{2.809324in}{1.611544in}}%
\pgfpathlineto{\pgfqpoint{2.843621in}{1.649836in}}%
\pgfpathlineto{\pgfqpoint{2.877917in}{0.960173in}}%
\pgfpathlineto{\pgfqpoint{2.912214in}{1.576109in}}%
\pgfpathlineto{\pgfqpoint{2.946510in}{1.486292in}}%
\pgfpathlineto{\pgfqpoint{2.980807in}{0.919577in}}%
\pgfpathlineto{\pgfqpoint{3.015103in}{0.894870in}}%
\pgfpathlineto{\pgfqpoint{3.049400in}{0.949862in}}%
\pgfpathlineto{\pgfqpoint{3.083696in}{0.873014in}}%
\pgfpathlineto{\pgfqpoint{3.117993in}{1.622611in}}%
\pgfpathlineto{\pgfqpoint{3.152289in}{1.615524in}}%
\pgfpathlineto{\pgfqpoint{3.186586in}{0.796442in}}%
\pgfpathlineto{\pgfqpoint{3.220882in}{1.335950in}}%
\pgfpathlineto{\pgfqpoint{3.255179in}{1.625389in}}%
\pgfpathlineto{\pgfqpoint{3.289475in}{1.644568in}}%
\pgfpathlineto{\pgfqpoint{3.323772in}{1.306488in}}%
\pgfpathlineto{\pgfqpoint{3.358068in}{1.768754in}}%
\pgfpathlineto{\pgfqpoint{3.392365in}{0.836242in}}%
\pgfpathlineto{\pgfqpoint{3.426661in}{1.464755in}}%
\pgfpathlineto{\pgfqpoint{3.460958in}{0.848904in}}%
\pgfpathlineto{\pgfqpoint{3.495254in}{0.838707in}}%
\pgfpathlineto{\pgfqpoint{3.529551in}{1.290907in}}%
\pgfpathlineto{\pgfqpoint{3.563847in}{1.614910in}}%
\pgfpathlineto{\pgfqpoint{3.598144in}{1.367364in}}%
\pgfpathlineto{\pgfqpoint{3.632440in}{1.837276in}}%
\pgfpathlineto{\pgfqpoint{3.666737in}{0.744117in}}%
\pgfpathlineto{\pgfqpoint{3.701033in}{1.369642in}}%
\pgfpathlineto{\pgfqpoint{3.735329in}{1.307899in}}%
\pgfpathlineto{\pgfqpoint{3.769626in}{0.857995in}}%
\pgfpathlineto{\pgfqpoint{3.803922in}{0.769633in}}%
\pgfpathlineto{\pgfqpoint{3.838219in}{1.507450in}}%
\pgfpathlineto{\pgfqpoint{3.872515in}{0.814057in}}%
\pgfpathlineto{\pgfqpoint{3.906812in}{1.543155in}}%
\pgfpathlineto{\pgfqpoint{3.941108in}{1.784426in}}%
\pgfpathlineto{\pgfqpoint{3.975405in}{1.990035in}}%
\pgfusepath{stroke}%
\end{pgfscope}%
\begin{pgfscope}%
\pgfsetrectcap%
\pgfsetmiterjoin%
\pgfsetlinewidth{1.254687pt}%
\definecolor{currentstroke}{rgb}{0.150000,0.150000,0.150000}%
\pgfsetstrokecolor{currentstroke}%
\pgfsetdash{}{0pt}%
\pgfpathmoveto{\pgfqpoint{0.734747in}{0.594647in}}%
\pgfpathlineto{\pgfqpoint{0.734747in}{2.488647in}}%
\pgfusepath{stroke}%
\end{pgfscope}%
\begin{pgfscope}%
\pgfsetrectcap%
\pgfsetmiterjoin%
\pgfsetlinewidth{1.254687pt}%
\definecolor{currentstroke}{rgb}{0.150000,0.150000,0.150000}%
\pgfsetstrokecolor{currentstroke}%
\pgfsetdash{}{0pt}%
\pgfpathmoveto{\pgfqpoint{0.734747in}{0.594647in}}%
\pgfpathlineto{\pgfqpoint{4.129722in}{0.594647in}}%
\pgfusepath{stroke}%
\end{pgfscope}%
\begin{pgfscope}%
\pgfsetroundcap%
\pgfsetroundjoin%
\pgfsetlinewidth{1.756562pt}%
\definecolor{currentstroke}{rgb}{0.400000,0.760784,0.647059}%
\pgfsetstrokecolor{currentstroke}%
\pgfsetdash{}{0pt}%
\pgfpathmoveto{\pgfqpoint{0.859747in}{0.991986in}}%
\pgfpathlineto{\pgfqpoint{1.137525in}{0.991986in}}%
\pgfusepath{stroke}%
\end{pgfscope}%
\begin{pgfscope}%
\definecolor{textcolor}{rgb}{0.150000,0.150000,0.150000}%
\pgfsetstrokecolor{textcolor}%
\pgfsetfillcolor{textcolor}%
\pgftext[x=1.248636in,y=0.943375in,left,base]{\color{textcolor}\sffamily\fontsize{10.000000}{12.000000}\selectfont fc}%
\end{pgfscope}%
\begin{pgfscope}%
\pgfsetroundcap%
\pgfsetroundjoin%
\pgfsetlinewidth{1.756562pt}%
\definecolor{currentstroke}{rgb}{0.988235,0.552941,0.384314}%
\pgfsetstrokecolor{currentstroke}%
\pgfsetdash{}{0pt}%
\pgfpathmoveto{\pgfqpoint{0.859747in}{0.795258in}}%
\pgfpathlineto{\pgfqpoint{1.137525in}{0.795258in}}%
\pgfusepath{stroke}%
\end{pgfscope}%
\begin{pgfscope}%
\definecolor{textcolor}{rgb}{0.150000,0.150000,0.150000}%
\pgfsetstrokecolor{textcolor}%
\pgfsetfillcolor{textcolor}%
\pgftext[x=1.248636in,y=0.746647in,left,base]{\color{textcolor}\sffamily\fontsize{10.000000}{12.000000}\selectfont conv}%
\end{pgfscope}%
\end{pgfpicture}%
\makeatother%
\endgroup%
}
			\caption{Validation Loss}
			\label{fig:tomplotminvalloss}
		\end{subfigure}\\
		\begin{subfigure}[t]{0.49\textwidth}
			\resizebox{\linewidth}{!}{%% Creator: Matplotlib, PGF backend
%%
%% To include the figure in your LaTeX document, write
%%   \input{<filename>.pgf}
%%
%% Make sure the required packages are loaded in your preamble
%%   \usepackage{pgf}
%%
%% Figures using additional raster images can only be included by \input if
%% they are in the same directory as the main LaTeX file. For loading figures
%% from other directories you can use the `import` package
%%   \usepackage{import}
%% and then include the figures with
%%   \import{<path to file>}{<filename>.pgf}
%%
%% Matplotlib used the following preamble
%%   \usepackage[utf8x]{inputenc}
%%   \usepackage[T1]{fontenc}
%%
\begingroup%
\makeatletter%
\begin{pgfpicture}%
\pgfpathrectangle{\pgfpointorigin}{\pgfqpoint{4.296389in}{2.655314in}}%
\pgfusepath{use as bounding box, clip}%
\begin{pgfscope}%
\pgfsetbuttcap%
\pgfsetmiterjoin%
\definecolor{currentfill}{rgb}{1.000000,1.000000,1.000000}%
\pgfsetfillcolor{currentfill}%
\pgfsetlinewidth{0.000000pt}%
\definecolor{currentstroke}{rgb}{1.000000,1.000000,1.000000}%
\pgfsetstrokecolor{currentstroke}%
\pgfsetdash{}{0pt}%
\pgfpathmoveto{\pgfqpoint{0.000000in}{0.000000in}}%
\pgfpathlineto{\pgfqpoint{4.296389in}{0.000000in}}%
\pgfpathlineto{\pgfqpoint{4.296389in}{2.655314in}}%
\pgfpathlineto{\pgfqpoint{0.000000in}{2.655314in}}%
\pgfpathclose%
\pgfusepath{fill}%
\end{pgfscope}%
\begin{pgfscope}%
\pgfsetbuttcap%
\pgfsetmiterjoin%
\definecolor{currentfill}{rgb}{1.000000,1.000000,1.000000}%
\pgfsetfillcolor{currentfill}%
\pgfsetlinewidth{0.000000pt}%
\definecolor{currentstroke}{rgb}{0.000000,0.000000,0.000000}%
\pgfsetstrokecolor{currentstroke}%
\pgfsetstrokeopacity{0.000000}%
\pgfsetdash{}{0pt}%
\pgfpathmoveto{\pgfqpoint{0.631546in}{0.594647in}}%
\pgfpathlineto{\pgfqpoint{4.040092in}{0.594647in}}%
\pgfpathlineto{\pgfqpoint{4.040092in}{2.488647in}}%
\pgfpathlineto{\pgfqpoint{0.631546in}{2.488647in}}%
\pgfpathclose%
\pgfusepath{fill}%
\end{pgfscope}%
\begin{pgfscope}%
\pgfsetbuttcap%
\pgfsetroundjoin%
\definecolor{currentfill}{rgb}{0.150000,0.150000,0.150000}%
\pgfsetfillcolor{currentfill}%
\pgfsetlinewidth{1.003750pt}%
\definecolor{currentstroke}{rgb}{0.150000,0.150000,0.150000}%
\pgfsetstrokecolor{currentstroke}%
\pgfsetdash{}{0pt}%
\pgfsys@defobject{currentmarker}{\pgfqpoint{0.000000in}{-0.083333in}}{\pgfqpoint{0.000000in}{0.000000in}}{%
\pgfpathmoveto{\pgfqpoint{0.000000in}{0.000000in}}%
\pgfpathlineto{\pgfqpoint{0.000000in}{-0.083333in}}%
\pgfusepath{stroke,fill}%
}%
\begin{pgfscope}%
\pgfsys@transformshift{0.631546in}{0.594647in}%
\pgfsys@useobject{currentmarker}{}%
\end{pgfscope}%
\end{pgfscope}%
\begin{pgfscope}%
\definecolor{textcolor}{rgb}{0.150000,0.150000,0.150000}%
\pgfsetstrokecolor{textcolor}%
\pgfsetfillcolor{textcolor}%
\pgftext[x=0.631546in,y=0.414091in,,top]{\color{textcolor}\sffamily\fontsize{10.000000}{12.000000}\selectfont \(\displaystyle 0\)}%
\end{pgfscope}%
\begin{pgfscope}%
\pgfsetbuttcap%
\pgfsetroundjoin%
\definecolor{currentfill}{rgb}{0.150000,0.150000,0.150000}%
\pgfsetfillcolor{currentfill}%
\pgfsetlinewidth{1.003750pt}%
\definecolor{currentstroke}{rgb}{0.150000,0.150000,0.150000}%
\pgfsetstrokecolor{currentstroke}%
\pgfsetdash{}{0pt}%
\pgfsys@defobject{currentmarker}{\pgfqpoint{0.000000in}{-0.083333in}}{\pgfqpoint{0.000000in}{0.000000in}}{%
\pgfpathmoveto{\pgfqpoint{0.000000in}{0.000000in}}%
\pgfpathlineto{\pgfqpoint{0.000000in}{-0.083333in}}%
\pgfusepath{stroke,fill}%
}%
\begin{pgfscope}%
\pgfsys@transformshift{1.313255in}{0.594647in}%
\pgfsys@useobject{currentmarker}{}%
\end{pgfscope}%
\end{pgfscope}%
\begin{pgfscope}%
\definecolor{textcolor}{rgb}{0.150000,0.150000,0.150000}%
\pgfsetstrokecolor{textcolor}%
\pgfsetfillcolor{textcolor}%
\pgftext[x=1.313255in,y=0.414091in,,top]{\color{textcolor}\sffamily\fontsize{10.000000}{12.000000}\selectfont \(\displaystyle 2000\)}%
\end{pgfscope}%
\begin{pgfscope}%
\pgfsetbuttcap%
\pgfsetroundjoin%
\definecolor{currentfill}{rgb}{0.150000,0.150000,0.150000}%
\pgfsetfillcolor{currentfill}%
\pgfsetlinewidth{1.003750pt}%
\definecolor{currentstroke}{rgb}{0.150000,0.150000,0.150000}%
\pgfsetstrokecolor{currentstroke}%
\pgfsetdash{}{0pt}%
\pgfsys@defobject{currentmarker}{\pgfqpoint{0.000000in}{-0.083333in}}{\pgfqpoint{0.000000in}{0.000000in}}{%
\pgfpathmoveto{\pgfqpoint{0.000000in}{0.000000in}}%
\pgfpathlineto{\pgfqpoint{0.000000in}{-0.083333in}}%
\pgfusepath{stroke,fill}%
}%
\begin{pgfscope}%
\pgfsys@transformshift{1.994965in}{0.594647in}%
\pgfsys@useobject{currentmarker}{}%
\end{pgfscope}%
\end{pgfscope}%
\begin{pgfscope}%
\definecolor{textcolor}{rgb}{0.150000,0.150000,0.150000}%
\pgfsetstrokecolor{textcolor}%
\pgfsetfillcolor{textcolor}%
\pgftext[x=1.994965in,y=0.414091in,,top]{\color{textcolor}\sffamily\fontsize{10.000000}{12.000000}\selectfont \(\displaystyle 4000\)}%
\end{pgfscope}%
\begin{pgfscope}%
\pgfsetbuttcap%
\pgfsetroundjoin%
\definecolor{currentfill}{rgb}{0.150000,0.150000,0.150000}%
\pgfsetfillcolor{currentfill}%
\pgfsetlinewidth{1.003750pt}%
\definecolor{currentstroke}{rgb}{0.150000,0.150000,0.150000}%
\pgfsetstrokecolor{currentstroke}%
\pgfsetdash{}{0pt}%
\pgfsys@defobject{currentmarker}{\pgfqpoint{0.000000in}{-0.083333in}}{\pgfqpoint{0.000000in}{0.000000in}}{%
\pgfpathmoveto{\pgfqpoint{0.000000in}{0.000000in}}%
\pgfpathlineto{\pgfqpoint{0.000000in}{-0.083333in}}%
\pgfusepath{stroke,fill}%
}%
\begin{pgfscope}%
\pgfsys@transformshift{2.676674in}{0.594647in}%
\pgfsys@useobject{currentmarker}{}%
\end{pgfscope}%
\end{pgfscope}%
\begin{pgfscope}%
\definecolor{textcolor}{rgb}{0.150000,0.150000,0.150000}%
\pgfsetstrokecolor{textcolor}%
\pgfsetfillcolor{textcolor}%
\pgftext[x=2.676674in,y=0.414091in,,top]{\color{textcolor}\sffamily\fontsize{10.000000}{12.000000}\selectfont \(\displaystyle 6000\)}%
\end{pgfscope}%
\begin{pgfscope}%
\pgfsetbuttcap%
\pgfsetroundjoin%
\definecolor{currentfill}{rgb}{0.150000,0.150000,0.150000}%
\pgfsetfillcolor{currentfill}%
\pgfsetlinewidth{1.003750pt}%
\definecolor{currentstroke}{rgb}{0.150000,0.150000,0.150000}%
\pgfsetstrokecolor{currentstroke}%
\pgfsetdash{}{0pt}%
\pgfsys@defobject{currentmarker}{\pgfqpoint{0.000000in}{-0.083333in}}{\pgfqpoint{0.000000in}{0.000000in}}{%
\pgfpathmoveto{\pgfqpoint{0.000000in}{0.000000in}}%
\pgfpathlineto{\pgfqpoint{0.000000in}{-0.083333in}}%
\pgfusepath{stroke,fill}%
}%
\begin{pgfscope}%
\pgfsys@transformshift{3.358383in}{0.594647in}%
\pgfsys@useobject{currentmarker}{}%
\end{pgfscope}%
\end{pgfscope}%
\begin{pgfscope}%
\definecolor{textcolor}{rgb}{0.150000,0.150000,0.150000}%
\pgfsetstrokecolor{textcolor}%
\pgfsetfillcolor{textcolor}%
\pgftext[x=3.358383in,y=0.414091in,,top]{\color{textcolor}\sffamily\fontsize{10.000000}{12.000000}\selectfont \(\displaystyle 8000\)}%
\end{pgfscope}%
\begin{pgfscope}%
\pgfsetbuttcap%
\pgfsetroundjoin%
\definecolor{currentfill}{rgb}{0.150000,0.150000,0.150000}%
\pgfsetfillcolor{currentfill}%
\pgfsetlinewidth{1.003750pt}%
\definecolor{currentstroke}{rgb}{0.150000,0.150000,0.150000}%
\pgfsetstrokecolor{currentstroke}%
\pgfsetdash{}{0pt}%
\pgfsys@defobject{currentmarker}{\pgfqpoint{0.000000in}{-0.083333in}}{\pgfqpoint{0.000000in}{0.000000in}}{%
\pgfpathmoveto{\pgfqpoint{0.000000in}{0.000000in}}%
\pgfpathlineto{\pgfqpoint{0.000000in}{-0.083333in}}%
\pgfusepath{stroke,fill}%
}%
\begin{pgfscope}%
\pgfsys@transformshift{4.040092in}{0.594647in}%
\pgfsys@useobject{currentmarker}{}%
\end{pgfscope}%
\end{pgfscope}%
\begin{pgfscope}%
\definecolor{textcolor}{rgb}{0.150000,0.150000,0.150000}%
\pgfsetstrokecolor{textcolor}%
\pgfsetfillcolor{textcolor}%
\pgftext[x=4.040092in,y=0.414091in,,top]{\color{textcolor}\sffamily\fontsize{10.000000}{12.000000}\selectfont \(\displaystyle 10000\)}%
\end{pgfscope}%
\begin{pgfscope}%
\definecolor{textcolor}{rgb}{0.150000,0.150000,0.150000}%
\pgfsetstrokecolor{textcolor}%
\pgfsetfillcolor{textcolor}%
\pgftext[x=2.335819in,y=0.231252in,,top]{\color{textcolor}\sffamily\fontsize{11.000000}{13.200000}\selectfont Train Data (\# samples)}%
\end{pgfscope}%
\begin{pgfscope}%
\pgfsetbuttcap%
\pgfsetroundjoin%
\definecolor{currentfill}{rgb}{0.150000,0.150000,0.150000}%
\pgfsetfillcolor{currentfill}%
\pgfsetlinewidth{1.003750pt}%
\definecolor{currentstroke}{rgb}{0.150000,0.150000,0.150000}%
\pgfsetstrokecolor{currentstroke}%
\pgfsetdash{}{0pt}%
\pgfsys@defobject{currentmarker}{\pgfqpoint{-0.083333in}{0.000000in}}{\pgfqpoint{0.000000in}{0.000000in}}{%
\pgfpathmoveto{\pgfqpoint{0.000000in}{0.000000in}}%
\pgfpathlineto{\pgfqpoint{-0.083333in}{0.000000in}}%
\pgfusepath{stroke,fill}%
}%
\begin{pgfscope}%
\pgfsys@transformshift{0.631546in}{0.594647in}%
\pgfsys@useobject{currentmarker}{}%
\end{pgfscope}%
\end{pgfscope}%
\begin{pgfscope}%
\definecolor{textcolor}{rgb}{0.150000,0.150000,0.150000}%
\pgfsetstrokecolor{textcolor}%
\pgfsetfillcolor{textcolor}%
\pgftext[x=0.273521in,y=0.544505in,left,base]{\color{textcolor}\sffamily\fontsize{10.000000}{12.000000}\selectfont \(\displaystyle 0.0\)}%
\end{pgfscope}%
\begin{pgfscope}%
\pgfsetbuttcap%
\pgfsetroundjoin%
\definecolor{currentfill}{rgb}{0.150000,0.150000,0.150000}%
\pgfsetfillcolor{currentfill}%
\pgfsetlinewidth{1.003750pt}%
\definecolor{currentstroke}{rgb}{0.150000,0.150000,0.150000}%
\pgfsetstrokecolor{currentstroke}%
\pgfsetdash{}{0pt}%
\pgfsys@defobject{currentmarker}{\pgfqpoint{-0.083333in}{0.000000in}}{\pgfqpoint{0.000000in}{0.000000in}}{%
\pgfpathmoveto{\pgfqpoint{0.000000in}{0.000000in}}%
\pgfpathlineto{\pgfqpoint{-0.083333in}{0.000000in}}%
\pgfusepath{stroke,fill}%
}%
\begin{pgfscope}%
\pgfsys@transformshift{0.631546in}{0.973447in}%
\pgfsys@useobject{currentmarker}{}%
\end{pgfscope}%
\end{pgfscope}%
\begin{pgfscope}%
\definecolor{textcolor}{rgb}{0.150000,0.150000,0.150000}%
\pgfsetstrokecolor{textcolor}%
\pgfsetfillcolor{textcolor}%
\pgftext[x=0.273521in,y=0.923305in,left,base]{\color{textcolor}\sffamily\fontsize{10.000000}{12.000000}\selectfont \(\displaystyle 0.2\)}%
\end{pgfscope}%
\begin{pgfscope}%
\pgfsetbuttcap%
\pgfsetroundjoin%
\definecolor{currentfill}{rgb}{0.150000,0.150000,0.150000}%
\pgfsetfillcolor{currentfill}%
\pgfsetlinewidth{1.003750pt}%
\definecolor{currentstroke}{rgb}{0.150000,0.150000,0.150000}%
\pgfsetstrokecolor{currentstroke}%
\pgfsetdash{}{0pt}%
\pgfsys@defobject{currentmarker}{\pgfqpoint{-0.083333in}{0.000000in}}{\pgfqpoint{0.000000in}{0.000000in}}{%
\pgfpathmoveto{\pgfqpoint{0.000000in}{0.000000in}}%
\pgfpathlineto{\pgfqpoint{-0.083333in}{0.000000in}}%
\pgfusepath{stroke,fill}%
}%
\begin{pgfscope}%
\pgfsys@transformshift{0.631546in}{1.352247in}%
\pgfsys@useobject{currentmarker}{}%
\end{pgfscope}%
\end{pgfscope}%
\begin{pgfscope}%
\definecolor{textcolor}{rgb}{0.150000,0.150000,0.150000}%
\pgfsetstrokecolor{textcolor}%
\pgfsetfillcolor{textcolor}%
\pgftext[x=0.273521in,y=1.302105in,left,base]{\color{textcolor}\sffamily\fontsize{10.000000}{12.000000}\selectfont \(\displaystyle 0.4\)}%
\end{pgfscope}%
\begin{pgfscope}%
\pgfsetbuttcap%
\pgfsetroundjoin%
\definecolor{currentfill}{rgb}{0.150000,0.150000,0.150000}%
\pgfsetfillcolor{currentfill}%
\pgfsetlinewidth{1.003750pt}%
\definecolor{currentstroke}{rgb}{0.150000,0.150000,0.150000}%
\pgfsetstrokecolor{currentstroke}%
\pgfsetdash{}{0pt}%
\pgfsys@defobject{currentmarker}{\pgfqpoint{-0.083333in}{0.000000in}}{\pgfqpoint{0.000000in}{0.000000in}}{%
\pgfpathmoveto{\pgfqpoint{0.000000in}{0.000000in}}%
\pgfpathlineto{\pgfqpoint{-0.083333in}{0.000000in}}%
\pgfusepath{stroke,fill}%
}%
\begin{pgfscope}%
\pgfsys@transformshift{0.631546in}{1.731047in}%
\pgfsys@useobject{currentmarker}{}%
\end{pgfscope}%
\end{pgfscope}%
\begin{pgfscope}%
\definecolor{textcolor}{rgb}{0.150000,0.150000,0.150000}%
\pgfsetstrokecolor{textcolor}%
\pgfsetfillcolor{textcolor}%
\pgftext[x=0.273521in,y=1.680905in,left,base]{\color{textcolor}\sffamily\fontsize{10.000000}{12.000000}\selectfont \(\displaystyle 0.6\)}%
\end{pgfscope}%
\begin{pgfscope}%
\pgfsetbuttcap%
\pgfsetroundjoin%
\definecolor{currentfill}{rgb}{0.150000,0.150000,0.150000}%
\pgfsetfillcolor{currentfill}%
\pgfsetlinewidth{1.003750pt}%
\definecolor{currentstroke}{rgb}{0.150000,0.150000,0.150000}%
\pgfsetstrokecolor{currentstroke}%
\pgfsetdash{}{0pt}%
\pgfsys@defobject{currentmarker}{\pgfqpoint{-0.083333in}{0.000000in}}{\pgfqpoint{0.000000in}{0.000000in}}{%
\pgfpathmoveto{\pgfqpoint{0.000000in}{0.000000in}}%
\pgfpathlineto{\pgfqpoint{-0.083333in}{0.000000in}}%
\pgfusepath{stroke,fill}%
}%
\begin{pgfscope}%
\pgfsys@transformshift{0.631546in}{2.109847in}%
\pgfsys@useobject{currentmarker}{}%
\end{pgfscope}%
\end{pgfscope}%
\begin{pgfscope}%
\definecolor{textcolor}{rgb}{0.150000,0.150000,0.150000}%
\pgfsetstrokecolor{textcolor}%
\pgfsetfillcolor{textcolor}%
\pgftext[x=0.273521in,y=2.059705in,left,base]{\color{textcolor}\sffamily\fontsize{10.000000}{12.000000}\selectfont \(\displaystyle 0.8\)}%
\end{pgfscope}%
\begin{pgfscope}%
\pgfsetbuttcap%
\pgfsetroundjoin%
\definecolor{currentfill}{rgb}{0.150000,0.150000,0.150000}%
\pgfsetfillcolor{currentfill}%
\pgfsetlinewidth{1.003750pt}%
\definecolor{currentstroke}{rgb}{0.150000,0.150000,0.150000}%
\pgfsetstrokecolor{currentstroke}%
\pgfsetdash{}{0pt}%
\pgfsys@defobject{currentmarker}{\pgfqpoint{-0.083333in}{0.000000in}}{\pgfqpoint{0.000000in}{0.000000in}}{%
\pgfpathmoveto{\pgfqpoint{0.000000in}{0.000000in}}%
\pgfpathlineto{\pgfqpoint{-0.083333in}{0.000000in}}%
\pgfusepath{stroke,fill}%
}%
\begin{pgfscope}%
\pgfsys@transformshift{0.631546in}{2.488647in}%
\pgfsys@useobject{currentmarker}{}%
\end{pgfscope}%
\end{pgfscope}%
\begin{pgfscope}%
\definecolor{textcolor}{rgb}{0.150000,0.150000,0.150000}%
\pgfsetstrokecolor{textcolor}%
\pgfsetfillcolor{textcolor}%
\pgftext[x=0.273521in,y=2.438505in,left,base]{\color{textcolor}\sffamily\fontsize{10.000000}{12.000000}\selectfont \(\displaystyle 1.0\)}%
\end{pgfscope}%
\begin{pgfscope}%
\definecolor{textcolor}{rgb}{0.150000,0.150000,0.150000}%
\pgfsetstrokecolor{textcolor}%
\pgfsetfillcolor{textcolor}%
\pgftext[x=0.217965in,y=1.541647in,,bottom,rotate=90.000000]{\color{textcolor}\sffamily\fontsize{11.000000}{13.200000}\selectfont Train Accuracy}%
\end{pgfscope}%
\begin{pgfscope}%
\pgfpathrectangle{\pgfqpoint{0.631546in}{0.594647in}}{\pgfqpoint{3.408546in}{1.894001in}} %
\pgfusepath{clip}%
\pgfsetroundcap%
\pgfsetroundjoin%
\pgfsetlinewidth{1.756562pt}%
\definecolor{currentstroke}{rgb}{0.400000,0.760784,0.647059}%
\pgfsetstrokecolor{currentstroke}%
\pgfsetdash{}{0pt}%
\pgfpathmoveto{\pgfqpoint{0.631887in}{0.594647in}}%
\pgfpathlineto{\pgfqpoint{0.632228in}{1.541647in}}%
\pgfpathlineto{\pgfqpoint{0.632569in}{1.857314in}}%
\pgfpathlineto{\pgfqpoint{0.632910in}{2.488647in}}%
\pgfpathlineto{\pgfqpoint{0.633250in}{2.109847in}}%
\pgfpathlineto{\pgfqpoint{0.633591in}{1.857314in}}%
\pgfpathlineto{\pgfqpoint{0.633932in}{2.488647in}}%
\pgfpathlineto{\pgfqpoint{0.634273in}{2.488647in}}%
\pgfpathlineto{\pgfqpoint{0.634614in}{1.646870in}}%
\pgfpathlineto{\pgfqpoint{0.634955in}{2.109847in}}%
\pgfpathlineto{\pgfqpoint{0.638363in}{2.488647in}}%
\pgfpathlineto{\pgfqpoint{0.641772in}{2.488647in}}%
\pgfpathlineto{\pgfqpoint{0.645180in}{2.488647in}}%
\pgfpathlineto{\pgfqpoint{0.648589in}{2.488647in}}%
\pgfpathlineto{\pgfqpoint{0.651997in}{2.488647in}}%
\pgfpathlineto{\pgfqpoint{0.655406in}{2.488647in}}%
\pgfpathlineto{\pgfqpoint{0.658814in}{2.488647in}}%
\pgfpathlineto{\pgfqpoint{0.662223in}{2.488647in}}%
\pgfpathlineto{\pgfqpoint{0.665632in}{2.468918in}}%
\pgfpathlineto{\pgfqpoint{0.699717in}{2.468918in}}%
\pgfpathlineto{\pgfqpoint{0.733802in}{2.482417in}}%
\pgfpathlineto{\pgfqpoint{0.767888in}{2.483912in}}%
\pgfpathlineto{\pgfqpoint{0.801973in}{2.469555in}}%
\pgfpathlineto{\pgfqpoint{0.836059in}{2.476187in}}%
\pgfpathlineto{\pgfqpoint{0.870144in}{2.477636in}}%
\pgfpathlineto{\pgfqpoint{0.904230in}{2.474442in}}%
\pgfpathlineto{\pgfqpoint{0.938315in}{2.473851in}}%
\pgfpathlineto{\pgfqpoint{0.972401in}{2.463827in}}%
\pgfpathlineto{\pgfqpoint{1.006486in}{2.467758in}}%
\pgfpathlineto{\pgfqpoint{1.040572in}{2.477599in}}%
\pgfpathlineto{\pgfqpoint{1.074657in}{2.481340in}}%
\pgfpathlineto{\pgfqpoint{1.108743in}{2.468238in}}%
\pgfpathlineto{\pgfqpoint{1.142828in}{2.467239in}}%
\pgfpathlineto{\pgfqpoint{1.176913in}{2.466156in}}%
\pgfpathlineto{\pgfqpoint{1.210999in}{2.459883in}}%
\pgfpathlineto{\pgfqpoint{1.245084in}{2.464553in}}%
\pgfpathlineto{\pgfqpoint{1.279170in}{2.467758in}}%
\pgfpathlineto{\pgfqpoint{1.313255in}{2.464972in}}%
\pgfpathlineto{\pgfqpoint{1.347341in}{2.462442in}}%
\pgfpathlineto{\pgfqpoint{1.381426in}{2.467203in}}%
\pgfpathlineto{\pgfqpoint{1.415512in}{2.467274in}}%
\pgfpathlineto{\pgfqpoint{1.449597in}{2.466551in}}%
\pgfpathlineto{\pgfqpoint{1.483683in}{2.466642in}}%
\pgfpathlineto{\pgfqpoint{1.517768in}{2.474123in}}%
\pgfpathlineto{\pgfqpoint{1.551854in}{2.465533in}}%
\pgfpathlineto{\pgfqpoint{1.585939in}{2.468355in}}%
\pgfpathlineto{\pgfqpoint{1.620024in}{2.476875in}}%
\pgfpathlineto{\pgfqpoint{1.654110in}{2.467758in}}%
\pgfpathlineto{\pgfqpoint{1.688195in}{2.458138in}}%
\pgfpathlineto{\pgfqpoint{1.722281in}{2.469707in}}%
\pgfpathlineto{\pgfqpoint{1.756366in}{2.471492in}}%
\pgfpathlineto{\pgfqpoint{1.790452in}{2.475865in}}%
\pgfpathlineto{\pgfqpoint{1.824537in}{2.475072in}}%
\pgfpathlineto{\pgfqpoint{1.858623in}{2.475495in}}%
\pgfpathlineto{\pgfqpoint{1.892708in}{2.470199in}}%
\pgfpathlineto{\pgfqpoint{1.926794in}{2.470742in}}%
\pgfpathlineto{\pgfqpoint{1.960879in}{2.468271in}}%
\pgfpathlineto{\pgfqpoint{1.994965in}{2.444612in}}%
\pgfpathlineto{\pgfqpoint{2.029050in}{2.473851in}}%
\pgfpathlineto{\pgfqpoint{2.063135in}{2.467864in}}%
\pgfpathlineto{\pgfqpoint{2.097221in}{2.469285in}}%
\pgfpathlineto{\pgfqpoint{2.131306in}{2.452059in}}%
\pgfpathlineto{\pgfqpoint{2.165392in}{2.472640in}}%
\pgfpathlineto{\pgfqpoint{2.199477in}{2.471384in}}%
\pgfpathlineto{\pgfqpoint{2.233563in}{2.466427in}}%
\pgfpathlineto{\pgfqpoint{2.267648in}{2.444060in}}%
\pgfpathlineto{\pgfqpoint{2.301734in}{2.466210in}}%
\pgfpathlineto{\pgfqpoint{2.335819in}{2.469298in}}%
\pgfpathlineto{\pgfqpoint{2.369905in}{2.447086in}}%
\pgfpathlineto{\pgfqpoint{2.403990in}{2.469343in}}%
\pgfpathlineto{\pgfqpoint{2.438076in}{2.468681in}}%
\pgfpathlineto{\pgfqpoint{2.472161in}{2.470436in}}%
\pgfpathlineto{\pgfqpoint{2.506246in}{2.472818in}}%
\pgfpathlineto{\pgfqpoint{2.540332in}{2.472751in}}%
\pgfpathlineto{\pgfqpoint{2.574417in}{2.470742in}}%
\pgfpathlineto{\pgfqpoint{2.608503in}{2.468755in}}%
\pgfpathlineto{\pgfqpoint{2.642588in}{2.470362in}}%
\pgfpathlineto{\pgfqpoint{2.676674in}{2.469076in}}%
\pgfpathlineto{\pgfqpoint{2.710759in}{2.469695in}}%
\pgfpathlineto{\pgfqpoint{2.744845in}{2.471257in}}%
\pgfpathlineto{\pgfqpoint{2.778930in}{2.471222in}}%
\pgfpathlineto{\pgfqpoint{2.813016in}{2.469707in}}%
\pgfpathlineto{\pgfqpoint{2.847101in}{2.472028in}}%
\pgfpathlineto{\pgfqpoint{2.881187in}{2.471408in}}%
\pgfpathlineto{\pgfqpoint{2.915272in}{2.470849in}}%
\pgfpathlineto{\pgfqpoint{2.949357in}{2.469986in}}%
\pgfpathlineto{\pgfqpoint{2.983443in}{2.470520in}}%
\pgfpathlineto{\pgfqpoint{3.017528in}{2.473478in}}%
\pgfpathlineto{\pgfqpoint{3.051614in}{2.470210in}}%
\pgfpathlineto{\pgfqpoint{3.085699in}{2.471286in}}%
\pgfpathlineto{\pgfqpoint{3.119785in}{2.469178in}}%
\pgfpathlineto{\pgfqpoint{3.153870in}{2.471224in}}%
\pgfpathlineto{\pgfqpoint{3.187956in}{2.470980in}}%
\pgfpathlineto{\pgfqpoint{3.222041in}{2.468960in}}%
\pgfpathlineto{\pgfqpoint{3.256127in}{2.470190in}}%
\pgfpathlineto{\pgfqpoint{3.290212in}{2.469445in}}%
\pgfpathlineto{\pgfqpoint{3.324298in}{2.469238in}}%
\pgfpathlineto{\pgfqpoint{3.358383in}{2.470891in}}%
\pgfpathlineto{\pgfqpoint{3.392468in}{2.472271in}}%
\pgfpathlineto{\pgfqpoint{3.426554in}{2.470880in}}%
\pgfpathlineto{\pgfqpoint{3.460639in}{2.471737in}}%
\pgfpathlineto{\pgfqpoint{3.494725in}{2.470384in}}%
\pgfpathlineto{\pgfqpoint{3.528810in}{2.472597in}}%
\pgfpathlineto{\pgfqpoint{3.562896in}{2.469469in}}%
\pgfpathlineto{\pgfqpoint{3.596981in}{2.467758in}}%
\pgfpathlineto{\pgfqpoint{3.631067in}{2.470784in}}%
\pgfpathlineto{\pgfqpoint{3.665152in}{2.470763in}}%
\pgfpathlineto{\pgfqpoint{3.699238in}{2.471586in}}%
\pgfusepath{stroke}%
\end{pgfscope}%
\begin{pgfscope}%
\pgfpathrectangle{\pgfqpoint{0.631546in}{0.594647in}}{\pgfqpoint{3.408546in}{1.894001in}} %
\pgfusepath{clip}%
\pgfsetroundcap%
\pgfsetroundjoin%
\pgfsetlinewidth{1.756562pt}%
\definecolor{currentstroke}{rgb}{0.988235,0.552941,0.384314}%
\pgfsetstrokecolor{currentstroke}%
\pgfsetdash{}{0pt}%
\pgfpathmoveto{\pgfqpoint{0.631887in}{0.594647in}}%
\pgfpathlineto{\pgfqpoint{0.632228in}{0.594647in}}%
\pgfpathlineto{\pgfqpoint{0.632569in}{2.488647in}}%
\pgfpathlineto{\pgfqpoint{0.632910in}{2.488647in}}%
\pgfpathlineto{\pgfqpoint{0.633250in}{2.488647in}}%
\pgfpathlineto{\pgfqpoint{0.633591in}{0.910314in}}%
\pgfpathlineto{\pgfqpoint{0.633932in}{2.488647in}}%
\pgfpathlineto{\pgfqpoint{0.634273in}{2.488647in}}%
\pgfpathlineto{\pgfqpoint{0.634614in}{1.225980in}}%
\pgfpathlineto{\pgfqpoint{0.634955in}{2.299247in}}%
\pgfpathlineto{\pgfqpoint{0.638363in}{2.488647in}}%
\pgfpathlineto{\pgfqpoint{0.641772in}{2.488647in}}%
\pgfpathlineto{\pgfqpoint{0.645180in}{2.488647in}}%
\pgfpathlineto{\pgfqpoint{0.648589in}{2.449189in}}%
\pgfpathlineto{\pgfqpoint{0.651997in}{2.488647in}}%
\pgfpathlineto{\pgfqpoint{0.655406in}{2.015147in}}%
\pgfpathlineto{\pgfqpoint{0.658814in}{2.441298in}}%
\pgfpathlineto{\pgfqpoint{0.662223in}{2.488647in}}%
\pgfpathlineto{\pgfqpoint{0.665632in}{2.488647in}}%
\pgfpathlineto{\pgfqpoint{0.699717in}{2.352061in}}%
\pgfpathlineto{\pgfqpoint{0.733802in}{2.357812in}}%
\pgfpathlineto{\pgfqpoint{0.767888in}{2.488647in}}%
\pgfpathlineto{\pgfqpoint{0.801973in}{2.481249in}}%
\pgfpathlineto{\pgfqpoint{0.836059in}{2.485448in}}%
\pgfpathlineto{\pgfqpoint{0.870144in}{2.488647in}}%
\pgfpathlineto{\pgfqpoint{0.904230in}{2.483912in}}%
\pgfpathlineto{\pgfqpoint{0.938315in}{2.486534in}}%
\pgfpathlineto{\pgfqpoint{0.972401in}{2.488647in}}%
\pgfpathlineto{\pgfqpoint{1.006486in}{2.488647in}}%
\pgfpathlineto{\pgfqpoint{1.040572in}{2.442876in}}%
\pgfpathlineto{\pgfqpoint{1.074657in}{2.487186in}}%
\pgfpathlineto{\pgfqpoint{1.108743in}{2.488647in}}%
\pgfpathlineto{\pgfqpoint{1.142828in}{2.488647in}}%
\pgfpathlineto{\pgfqpoint{1.176913in}{2.485096in}}%
\pgfpathlineto{\pgfqpoint{1.210999in}{2.488647in}}%
\pgfpathlineto{\pgfqpoint{1.245084in}{2.477021in}}%
\pgfpathlineto{\pgfqpoint{1.279170in}{2.488647in}}%
\pgfpathlineto{\pgfqpoint{1.313255in}{2.398682in}}%
\pgfpathlineto{\pgfqpoint{1.347341in}{2.442563in}}%
\pgfpathlineto{\pgfqpoint{1.381426in}{2.487790in}}%
\pgfpathlineto{\pgfqpoint{1.415512in}{2.487825in}}%
\pgfpathlineto{\pgfqpoint{1.449597in}{2.481545in}}%
\pgfpathlineto{\pgfqpoint{1.483683in}{2.487893in}}%
\pgfpathlineto{\pgfqpoint{1.517768in}{2.482071in}}%
\pgfpathlineto{\pgfqpoint{1.551854in}{2.488647in}}%
\pgfpathlineto{\pgfqpoint{1.585939in}{2.308717in}}%
\pgfpathlineto{\pgfqpoint{1.620024in}{2.337572in}}%
\pgfpathlineto{\pgfqpoint{1.654110in}{2.481092in}}%
\pgfpathlineto{\pgfqpoint{1.688195in}{2.483766in}}%
\pgfpathlineto{\pgfqpoint{1.722281in}{2.482137in}}%
\pgfpathlineto{\pgfqpoint{1.756366in}{2.483501in}}%
\pgfpathlineto{\pgfqpoint{1.790452in}{2.480830in}}%
\pgfpathlineto{\pgfqpoint{1.824537in}{2.488647in}}%
\pgfpathlineto{\pgfqpoint{1.858623in}{2.481282in}}%
\pgfpathlineto{\pgfqpoint{1.892708in}{2.344650in}}%
\pgfpathlineto{\pgfqpoint{1.926794in}{2.488647in}}%
\pgfpathlineto{\pgfqpoint{1.960879in}{2.488647in}}%
\pgfpathlineto{\pgfqpoint{1.994965in}{2.480598in}}%
\pgfpathlineto{\pgfqpoint{2.029050in}{2.488187in}}%
\pgfpathlineto{\pgfqpoint{2.063135in}{2.479195in}}%
\pgfpathlineto{\pgfqpoint{2.097221in}{2.459495in}}%
\pgfpathlineto{\pgfqpoint{2.131306in}{2.488217in}}%
\pgfpathlineto{\pgfqpoint{2.165392in}{2.487808in}}%
\pgfpathlineto{\pgfqpoint{2.199477in}{2.487825in}}%
\pgfpathlineto{\pgfqpoint{2.233563in}{2.488647in}}%
\pgfpathlineto{\pgfqpoint{2.267648in}{2.488253in}}%
\pgfpathlineto{\pgfqpoint{2.301734in}{2.465437in}}%
\pgfpathlineto{\pgfqpoint{2.335819in}{2.488268in}}%
\pgfpathlineto{\pgfqpoint{2.369905in}{2.487905in}}%
\pgfpathlineto{\pgfqpoint{2.403990in}{2.462423in}}%
\pgfpathlineto{\pgfqpoint{2.438076in}{2.220426in}}%
\pgfpathlineto{\pgfqpoint{2.472161in}{2.444870in}}%
\pgfpathlineto{\pgfqpoint{2.506246in}{2.487959in}}%
\pgfpathlineto{\pgfqpoint{2.540332in}{2.488309in}}%
\pgfpathlineto{\pgfqpoint{2.574417in}{2.488316in}}%
\pgfpathlineto{\pgfqpoint{2.608503in}{2.487995in}}%
\pgfpathlineto{\pgfqpoint{2.642588in}{2.488327in}}%
\pgfpathlineto{\pgfqpoint{2.676674in}{2.464972in}}%
\pgfpathlineto{\pgfqpoint{2.710759in}{2.488338in}}%
\pgfpathlineto{\pgfqpoint{2.744845in}{2.488037in}}%
\pgfpathlineto{\pgfqpoint{2.778930in}{2.488346in}}%
\pgfpathlineto{\pgfqpoint{2.813016in}{2.488056in}}%
\pgfpathlineto{\pgfqpoint{2.847101in}{2.488647in}}%
\pgfpathlineto{\pgfqpoint{2.881187in}{2.488647in}}%
\pgfpathlineto{\pgfqpoint{2.915272in}{2.488082in}}%
\pgfpathlineto{\pgfqpoint{2.949357in}{2.467758in}}%
\pgfpathlineto{\pgfqpoint{2.983443in}{2.488373in}}%
\pgfpathlineto{\pgfqpoint{3.017528in}{2.488647in}}%
\pgfpathlineto{\pgfqpoint{3.051614in}{2.468072in}}%
\pgfpathlineto{\pgfqpoint{3.085699in}{2.466551in}}%
\pgfpathlineto{\pgfqpoint{3.119785in}{2.488129in}}%
\pgfpathlineto{\pgfqpoint{3.153870in}{2.469175in}}%
\pgfpathlineto{\pgfqpoint{3.187956in}{2.488143in}}%
\pgfpathlineto{\pgfqpoint{3.222041in}{2.488149in}}%
\pgfpathlineto{\pgfqpoint{3.256127in}{2.488156in}}%
\pgfpathlineto{\pgfqpoint{3.290212in}{2.488405in}}%
\pgfpathlineto{\pgfqpoint{3.324298in}{2.467081in}}%
\pgfpathlineto{\pgfqpoint{3.358383in}{2.466630in}}%
\pgfpathlineto{\pgfqpoint{3.392468in}{2.488181in}}%
\pgfpathlineto{\pgfqpoint{3.426554in}{2.469458in}}%
\pgfpathlineto{\pgfqpoint{3.460639in}{2.469260in}}%
\pgfpathlineto{\pgfqpoint{3.494725in}{2.488196in}}%
\pgfpathlineto{\pgfqpoint{3.528810in}{2.488202in}}%
\pgfpathlineto{\pgfqpoint{3.562896in}{2.458888in}}%
\pgfpathlineto{\pgfqpoint{3.596981in}{2.488212in}}%
\pgfpathlineto{\pgfqpoint{3.631067in}{2.488217in}}%
\pgfpathlineto{\pgfqpoint{3.665152in}{2.470763in}}%
\pgfpathlineto{\pgfqpoint{3.699238in}{2.160856in}}%
\pgfusepath{stroke}%
\end{pgfscope}%
\begin{pgfscope}%
\pgfsetrectcap%
\pgfsetmiterjoin%
\pgfsetlinewidth{1.254687pt}%
\definecolor{currentstroke}{rgb}{0.150000,0.150000,0.150000}%
\pgfsetstrokecolor{currentstroke}%
\pgfsetdash{}{0pt}%
\pgfpathmoveto{\pgfqpoint{0.631546in}{0.594647in}}%
\pgfpathlineto{\pgfqpoint{0.631546in}{2.488647in}}%
\pgfusepath{stroke}%
\end{pgfscope}%
\begin{pgfscope}%
\pgfsetrectcap%
\pgfsetmiterjoin%
\pgfsetlinewidth{1.254687pt}%
\definecolor{currentstroke}{rgb}{0.150000,0.150000,0.150000}%
\pgfsetstrokecolor{currentstroke}%
\pgfsetdash{}{0pt}%
\pgfpathmoveto{\pgfqpoint{0.631546in}{0.594647in}}%
\pgfpathlineto{\pgfqpoint{4.040092in}{0.594647in}}%
\pgfusepath{stroke}%
\end{pgfscope}%
\begin{pgfscope}%
\pgfsetroundcap%
\pgfsetroundjoin%
\pgfsetlinewidth{1.756562pt}%
\definecolor{currentstroke}{rgb}{0.400000,0.760784,0.647059}%
\pgfsetstrokecolor{currentstroke}%
\pgfsetdash{}{0pt}%
\pgfpathmoveto{\pgfqpoint{0.756546in}{0.991986in}}%
\pgfpathlineto{\pgfqpoint{1.034324in}{0.991986in}}%
\pgfusepath{stroke}%
\end{pgfscope}%
\begin{pgfscope}%
\definecolor{textcolor}{rgb}{0.150000,0.150000,0.150000}%
\pgfsetstrokecolor{textcolor}%
\pgfsetfillcolor{textcolor}%
\pgftext[x=1.145435in,y=0.943375in,left,base]{\color{textcolor}\sffamily\fontsize{10.000000}{12.000000}\selectfont fc}%
\end{pgfscope}%
\begin{pgfscope}%
\pgfsetroundcap%
\pgfsetroundjoin%
\pgfsetlinewidth{1.756562pt}%
\definecolor{currentstroke}{rgb}{0.988235,0.552941,0.384314}%
\pgfsetstrokecolor{currentstroke}%
\pgfsetdash{}{0pt}%
\pgfpathmoveto{\pgfqpoint{0.756546in}{0.795258in}}%
\pgfpathlineto{\pgfqpoint{1.034324in}{0.795258in}}%
\pgfusepath{stroke}%
\end{pgfscope}%
\begin{pgfscope}%
\definecolor{textcolor}{rgb}{0.150000,0.150000,0.150000}%
\pgfsetstrokecolor{textcolor}%
\pgfsetfillcolor{textcolor}%
\pgftext[x=1.145435in,y=0.746647in,left,base]{\color{textcolor}\sffamily\fontsize{10.000000}{12.000000}\selectfont conv}%
\end{pgfscope}%
\end{pgfpicture}%
\makeatother%
\endgroup%
}
			\caption{Training Accuracy}
			\label{fig:tomplotmaxtrainacc}
		\end{subfigure}
		~
		\begin{subfigure}[t]{0.49\textwidth}
			\resizebox{\linewidth}{!}{%% Creator: Matplotlib, PGF backend
%%
%% To include the figure in your LaTeX document, write
%%   \input{<filename>.pgf}
%%
%% Make sure the required packages are loaded in your preamble
%%   \usepackage{pgf}
%%
%% Figures using additional raster images can only be included by \input if
%% they are in the same directory as the main LaTeX file. For loading figures
%% from other directories you can use the `import` package
%%   \usepackage{import}
%% and then include the figures with
%%   \import{<path to file>}{<filename>.pgf}
%%
%% Matplotlib used the following preamble
%%   \usepackage[utf8x]{inputenc}
%%   \usepackage[T1]{fontenc}
%%
\begingroup%
\makeatletter%
\begin{pgfpicture}%
\pgfpathrectangle{\pgfpointorigin}{\pgfqpoint{4.296389in}{2.655314in}}%
\pgfusepath{use as bounding box, clip}%
\begin{pgfscope}%
\pgfsetbuttcap%
\pgfsetmiterjoin%
\definecolor{currentfill}{rgb}{1.000000,1.000000,1.000000}%
\pgfsetfillcolor{currentfill}%
\pgfsetlinewidth{0.000000pt}%
\definecolor{currentstroke}{rgb}{1.000000,1.000000,1.000000}%
\pgfsetstrokecolor{currentstroke}%
\pgfsetdash{}{0pt}%
\pgfpathmoveto{\pgfqpoint{0.000000in}{0.000000in}}%
\pgfpathlineto{\pgfqpoint{4.296389in}{0.000000in}}%
\pgfpathlineto{\pgfqpoint{4.296389in}{2.655314in}}%
\pgfpathlineto{\pgfqpoint{0.000000in}{2.655314in}}%
\pgfpathclose%
\pgfusepath{fill}%
\end{pgfscope}%
\begin{pgfscope}%
\pgfsetbuttcap%
\pgfsetmiterjoin%
\definecolor{currentfill}{rgb}{1.000000,1.000000,1.000000}%
\pgfsetfillcolor{currentfill}%
\pgfsetlinewidth{0.000000pt}%
\definecolor{currentstroke}{rgb}{0.000000,0.000000,0.000000}%
\pgfsetstrokecolor{currentstroke}%
\pgfsetstrokeopacity{0.000000}%
\pgfsetdash{}{0pt}%
\pgfpathmoveto{\pgfqpoint{0.631546in}{0.594647in}}%
\pgfpathlineto{\pgfqpoint{4.040092in}{0.594647in}}%
\pgfpathlineto{\pgfqpoint{4.040092in}{2.488647in}}%
\pgfpathlineto{\pgfqpoint{0.631546in}{2.488647in}}%
\pgfpathclose%
\pgfusepath{fill}%
\end{pgfscope}%
\begin{pgfscope}%
\pgfsetbuttcap%
\pgfsetroundjoin%
\definecolor{currentfill}{rgb}{0.150000,0.150000,0.150000}%
\pgfsetfillcolor{currentfill}%
\pgfsetlinewidth{1.003750pt}%
\definecolor{currentstroke}{rgb}{0.150000,0.150000,0.150000}%
\pgfsetstrokecolor{currentstroke}%
\pgfsetdash{}{0pt}%
\pgfsys@defobject{currentmarker}{\pgfqpoint{0.000000in}{-0.083333in}}{\pgfqpoint{0.000000in}{0.000000in}}{%
\pgfpathmoveto{\pgfqpoint{0.000000in}{0.000000in}}%
\pgfpathlineto{\pgfqpoint{0.000000in}{-0.083333in}}%
\pgfusepath{stroke,fill}%
}%
\begin{pgfscope}%
\pgfsys@transformshift{0.631546in}{0.594647in}%
\pgfsys@useobject{currentmarker}{}%
\end{pgfscope}%
\end{pgfscope}%
\begin{pgfscope}%
\definecolor{textcolor}{rgb}{0.150000,0.150000,0.150000}%
\pgfsetstrokecolor{textcolor}%
\pgfsetfillcolor{textcolor}%
\pgftext[x=0.631546in,y=0.414091in,,top]{\color{textcolor}\sffamily\fontsize{10.000000}{12.000000}\selectfont \(\displaystyle 0\)}%
\end{pgfscope}%
\begin{pgfscope}%
\pgfsetbuttcap%
\pgfsetroundjoin%
\definecolor{currentfill}{rgb}{0.150000,0.150000,0.150000}%
\pgfsetfillcolor{currentfill}%
\pgfsetlinewidth{1.003750pt}%
\definecolor{currentstroke}{rgb}{0.150000,0.150000,0.150000}%
\pgfsetstrokecolor{currentstroke}%
\pgfsetdash{}{0pt}%
\pgfsys@defobject{currentmarker}{\pgfqpoint{0.000000in}{-0.083333in}}{\pgfqpoint{0.000000in}{0.000000in}}{%
\pgfpathmoveto{\pgfqpoint{0.000000in}{0.000000in}}%
\pgfpathlineto{\pgfqpoint{0.000000in}{-0.083333in}}%
\pgfusepath{stroke,fill}%
}%
\begin{pgfscope}%
\pgfsys@transformshift{1.313255in}{0.594647in}%
\pgfsys@useobject{currentmarker}{}%
\end{pgfscope}%
\end{pgfscope}%
\begin{pgfscope}%
\definecolor{textcolor}{rgb}{0.150000,0.150000,0.150000}%
\pgfsetstrokecolor{textcolor}%
\pgfsetfillcolor{textcolor}%
\pgftext[x=1.313255in,y=0.414091in,,top]{\color{textcolor}\sffamily\fontsize{10.000000}{12.000000}\selectfont \(\displaystyle 2000\)}%
\end{pgfscope}%
\begin{pgfscope}%
\pgfsetbuttcap%
\pgfsetroundjoin%
\definecolor{currentfill}{rgb}{0.150000,0.150000,0.150000}%
\pgfsetfillcolor{currentfill}%
\pgfsetlinewidth{1.003750pt}%
\definecolor{currentstroke}{rgb}{0.150000,0.150000,0.150000}%
\pgfsetstrokecolor{currentstroke}%
\pgfsetdash{}{0pt}%
\pgfsys@defobject{currentmarker}{\pgfqpoint{0.000000in}{-0.083333in}}{\pgfqpoint{0.000000in}{0.000000in}}{%
\pgfpathmoveto{\pgfqpoint{0.000000in}{0.000000in}}%
\pgfpathlineto{\pgfqpoint{0.000000in}{-0.083333in}}%
\pgfusepath{stroke,fill}%
}%
\begin{pgfscope}%
\pgfsys@transformshift{1.994965in}{0.594647in}%
\pgfsys@useobject{currentmarker}{}%
\end{pgfscope}%
\end{pgfscope}%
\begin{pgfscope}%
\definecolor{textcolor}{rgb}{0.150000,0.150000,0.150000}%
\pgfsetstrokecolor{textcolor}%
\pgfsetfillcolor{textcolor}%
\pgftext[x=1.994965in,y=0.414091in,,top]{\color{textcolor}\sffamily\fontsize{10.000000}{12.000000}\selectfont \(\displaystyle 4000\)}%
\end{pgfscope}%
\begin{pgfscope}%
\pgfsetbuttcap%
\pgfsetroundjoin%
\definecolor{currentfill}{rgb}{0.150000,0.150000,0.150000}%
\pgfsetfillcolor{currentfill}%
\pgfsetlinewidth{1.003750pt}%
\definecolor{currentstroke}{rgb}{0.150000,0.150000,0.150000}%
\pgfsetstrokecolor{currentstroke}%
\pgfsetdash{}{0pt}%
\pgfsys@defobject{currentmarker}{\pgfqpoint{0.000000in}{-0.083333in}}{\pgfqpoint{0.000000in}{0.000000in}}{%
\pgfpathmoveto{\pgfqpoint{0.000000in}{0.000000in}}%
\pgfpathlineto{\pgfqpoint{0.000000in}{-0.083333in}}%
\pgfusepath{stroke,fill}%
}%
\begin{pgfscope}%
\pgfsys@transformshift{2.676674in}{0.594647in}%
\pgfsys@useobject{currentmarker}{}%
\end{pgfscope}%
\end{pgfscope}%
\begin{pgfscope}%
\definecolor{textcolor}{rgb}{0.150000,0.150000,0.150000}%
\pgfsetstrokecolor{textcolor}%
\pgfsetfillcolor{textcolor}%
\pgftext[x=2.676674in,y=0.414091in,,top]{\color{textcolor}\sffamily\fontsize{10.000000}{12.000000}\selectfont \(\displaystyle 6000\)}%
\end{pgfscope}%
\begin{pgfscope}%
\pgfsetbuttcap%
\pgfsetroundjoin%
\definecolor{currentfill}{rgb}{0.150000,0.150000,0.150000}%
\pgfsetfillcolor{currentfill}%
\pgfsetlinewidth{1.003750pt}%
\definecolor{currentstroke}{rgb}{0.150000,0.150000,0.150000}%
\pgfsetstrokecolor{currentstroke}%
\pgfsetdash{}{0pt}%
\pgfsys@defobject{currentmarker}{\pgfqpoint{0.000000in}{-0.083333in}}{\pgfqpoint{0.000000in}{0.000000in}}{%
\pgfpathmoveto{\pgfqpoint{0.000000in}{0.000000in}}%
\pgfpathlineto{\pgfqpoint{0.000000in}{-0.083333in}}%
\pgfusepath{stroke,fill}%
}%
\begin{pgfscope}%
\pgfsys@transformshift{3.358383in}{0.594647in}%
\pgfsys@useobject{currentmarker}{}%
\end{pgfscope}%
\end{pgfscope}%
\begin{pgfscope}%
\definecolor{textcolor}{rgb}{0.150000,0.150000,0.150000}%
\pgfsetstrokecolor{textcolor}%
\pgfsetfillcolor{textcolor}%
\pgftext[x=3.358383in,y=0.414091in,,top]{\color{textcolor}\sffamily\fontsize{10.000000}{12.000000}\selectfont \(\displaystyle 8000\)}%
\end{pgfscope}%
\begin{pgfscope}%
\pgfsetbuttcap%
\pgfsetroundjoin%
\definecolor{currentfill}{rgb}{0.150000,0.150000,0.150000}%
\pgfsetfillcolor{currentfill}%
\pgfsetlinewidth{1.003750pt}%
\definecolor{currentstroke}{rgb}{0.150000,0.150000,0.150000}%
\pgfsetstrokecolor{currentstroke}%
\pgfsetdash{}{0pt}%
\pgfsys@defobject{currentmarker}{\pgfqpoint{0.000000in}{-0.083333in}}{\pgfqpoint{0.000000in}{0.000000in}}{%
\pgfpathmoveto{\pgfqpoint{0.000000in}{0.000000in}}%
\pgfpathlineto{\pgfqpoint{0.000000in}{-0.083333in}}%
\pgfusepath{stroke,fill}%
}%
\begin{pgfscope}%
\pgfsys@transformshift{4.040092in}{0.594647in}%
\pgfsys@useobject{currentmarker}{}%
\end{pgfscope}%
\end{pgfscope}%
\begin{pgfscope}%
\definecolor{textcolor}{rgb}{0.150000,0.150000,0.150000}%
\pgfsetstrokecolor{textcolor}%
\pgfsetfillcolor{textcolor}%
\pgftext[x=4.040092in,y=0.414091in,,top]{\color{textcolor}\sffamily\fontsize{10.000000}{12.000000}\selectfont \(\displaystyle 10000\)}%
\end{pgfscope}%
\begin{pgfscope}%
\definecolor{textcolor}{rgb}{0.150000,0.150000,0.150000}%
\pgfsetstrokecolor{textcolor}%
\pgfsetfillcolor{textcolor}%
\pgftext[x=2.335819in,y=0.231252in,,top]{\color{textcolor}\sffamily\fontsize{11.000000}{13.200000}\selectfont Train Data (\# samples)}%
\end{pgfscope}%
\begin{pgfscope}%
\pgfsetbuttcap%
\pgfsetroundjoin%
\definecolor{currentfill}{rgb}{0.150000,0.150000,0.150000}%
\pgfsetfillcolor{currentfill}%
\pgfsetlinewidth{1.003750pt}%
\definecolor{currentstroke}{rgb}{0.150000,0.150000,0.150000}%
\pgfsetstrokecolor{currentstroke}%
\pgfsetdash{}{0pt}%
\pgfsys@defobject{currentmarker}{\pgfqpoint{-0.083333in}{0.000000in}}{\pgfqpoint{0.000000in}{0.000000in}}{%
\pgfpathmoveto{\pgfqpoint{0.000000in}{0.000000in}}%
\pgfpathlineto{\pgfqpoint{-0.083333in}{0.000000in}}%
\pgfusepath{stroke,fill}%
}%
\begin{pgfscope}%
\pgfsys@transformshift{0.631546in}{0.594647in}%
\pgfsys@useobject{currentmarker}{}%
\end{pgfscope}%
\end{pgfscope}%
\begin{pgfscope}%
\definecolor{textcolor}{rgb}{0.150000,0.150000,0.150000}%
\pgfsetstrokecolor{textcolor}%
\pgfsetfillcolor{textcolor}%
\pgftext[x=0.273521in,y=0.544505in,left,base]{\color{textcolor}\sffamily\fontsize{10.000000}{12.000000}\selectfont \(\displaystyle 0.0\)}%
\end{pgfscope}%
\begin{pgfscope}%
\pgfsetbuttcap%
\pgfsetroundjoin%
\definecolor{currentfill}{rgb}{0.150000,0.150000,0.150000}%
\pgfsetfillcolor{currentfill}%
\pgfsetlinewidth{1.003750pt}%
\definecolor{currentstroke}{rgb}{0.150000,0.150000,0.150000}%
\pgfsetstrokecolor{currentstroke}%
\pgfsetdash{}{0pt}%
\pgfsys@defobject{currentmarker}{\pgfqpoint{-0.083333in}{0.000000in}}{\pgfqpoint{0.000000in}{0.000000in}}{%
\pgfpathmoveto{\pgfqpoint{0.000000in}{0.000000in}}%
\pgfpathlineto{\pgfqpoint{-0.083333in}{0.000000in}}%
\pgfusepath{stroke,fill}%
}%
\begin{pgfscope}%
\pgfsys@transformshift{0.631546in}{0.973447in}%
\pgfsys@useobject{currentmarker}{}%
\end{pgfscope}%
\end{pgfscope}%
\begin{pgfscope}%
\definecolor{textcolor}{rgb}{0.150000,0.150000,0.150000}%
\pgfsetstrokecolor{textcolor}%
\pgfsetfillcolor{textcolor}%
\pgftext[x=0.273521in,y=0.923305in,left,base]{\color{textcolor}\sffamily\fontsize{10.000000}{12.000000}\selectfont \(\displaystyle 0.2\)}%
\end{pgfscope}%
\begin{pgfscope}%
\pgfsetbuttcap%
\pgfsetroundjoin%
\definecolor{currentfill}{rgb}{0.150000,0.150000,0.150000}%
\pgfsetfillcolor{currentfill}%
\pgfsetlinewidth{1.003750pt}%
\definecolor{currentstroke}{rgb}{0.150000,0.150000,0.150000}%
\pgfsetstrokecolor{currentstroke}%
\pgfsetdash{}{0pt}%
\pgfsys@defobject{currentmarker}{\pgfqpoint{-0.083333in}{0.000000in}}{\pgfqpoint{0.000000in}{0.000000in}}{%
\pgfpathmoveto{\pgfqpoint{0.000000in}{0.000000in}}%
\pgfpathlineto{\pgfqpoint{-0.083333in}{0.000000in}}%
\pgfusepath{stroke,fill}%
}%
\begin{pgfscope}%
\pgfsys@transformshift{0.631546in}{1.352247in}%
\pgfsys@useobject{currentmarker}{}%
\end{pgfscope}%
\end{pgfscope}%
\begin{pgfscope}%
\definecolor{textcolor}{rgb}{0.150000,0.150000,0.150000}%
\pgfsetstrokecolor{textcolor}%
\pgfsetfillcolor{textcolor}%
\pgftext[x=0.273521in,y=1.302105in,left,base]{\color{textcolor}\sffamily\fontsize{10.000000}{12.000000}\selectfont \(\displaystyle 0.4\)}%
\end{pgfscope}%
\begin{pgfscope}%
\pgfsetbuttcap%
\pgfsetroundjoin%
\definecolor{currentfill}{rgb}{0.150000,0.150000,0.150000}%
\pgfsetfillcolor{currentfill}%
\pgfsetlinewidth{1.003750pt}%
\definecolor{currentstroke}{rgb}{0.150000,0.150000,0.150000}%
\pgfsetstrokecolor{currentstroke}%
\pgfsetdash{}{0pt}%
\pgfsys@defobject{currentmarker}{\pgfqpoint{-0.083333in}{0.000000in}}{\pgfqpoint{0.000000in}{0.000000in}}{%
\pgfpathmoveto{\pgfqpoint{0.000000in}{0.000000in}}%
\pgfpathlineto{\pgfqpoint{-0.083333in}{0.000000in}}%
\pgfusepath{stroke,fill}%
}%
\begin{pgfscope}%
\pgfsys@transformshift{0.631546in}{1.731047in}%
\pgfsys@useobject{currentmarker}{}%
\end{pgfscope}%
\end{pgfscope}%
\begin{pgfscope}%
\definecolor{textcolor}{rgb}{0.150000,0.150000,0.150000}%
\pgfsetstrokecolor{textcolor}%
\pgfsetfillcolor{textcolor}%
\pgftext[x=0.273521in,y=1.680905in,left,base]{\color{textcolor}\sffamily\fontsize{10.000000}{12.000000}\selectfont \(\displaystyle 0.6\)}%
\end{pgfscope}%
\begin{pgfscope}%
\pgfsetbuttcap%
\pgfsetroundjoin%
\definecolor{currentfill}{rgb}{0.150000,0.150000,0.150000}%
\pgfsetfillcolor{currentfill}%
\pgfsetlinewidth{1.003750pt}%
\definecolor{currentstroke}{rgb}{0.150000,0.150000,0.150000}%
\pgfsetstrokecolor{currentstroke}%
\pgfsetdash{}{0pt}%
\pgfsys@defobject{currentmarker}{\pgfqpoint{-0.083333in}{0.000000in}}{\pgfqpoint{0.000000in}{0.000000in}}{%
\pgfpathmoveto{\pgfqpoint{0.000000in}{0.000000in}}%
\pgfpathlineto{\pgfqpoint{-0.083333in}{0.000000in}}%
\pgfusepath{stroke,fill}%
}%
\begin{pgfscope}%
\pgfsys@transformshift{0.631546in}{2.109847in}%
\pgfsys@useobject{currentmarker}{}%
\end{pgfscope}%
\end{pgfscope}%
\begin{pgfscope}%
\definecolor{textcolor}{rgb}{0.150000,0.150000,0.150000}%
\pgfsetstrokecolor{textcolor}%
\pgfsetfillcolor{textcolor}%
\pgftext[x=0.273521in,y=2.059705in,left,base]{\color{textcolor}\sffamily\fontsize{10.000000}{12.000000}\selectfont \(\displaystyle 0.8\)}%
\end{pgfscope}%
\begin{pgfscope}%
\pgfsetbuttcap%
\pgfsetroundjoin%
\definecolor{currentfill}{rgb}{0.150000,0.150000,0.150000}%
\pgfsetfillcolor{currentfill}%
\pgfsetlinewidth{1.003750pt}%
\definecolor{currentstroke}{rgb}{0.150000,0.150000,0.150000}%
\pgfsetstrokecolor{currentstroke}%
\pgfsetdash{}{0pt}%
\pgfsys@defobject{currentmarker}{\pgfqpoint{-0.083333in}{0.000000in}}{\pgfqpoint{0.000000in}{0.000000in}}{%
\pgfpathmoveto{\pgfqpoint{0.000000in}{0.000000in}}%
\pgfpathlineto{\pgfqpoint{-0.083333in}{0.000000in}}%
\pgfusepath{stroke,fill}%
}%
\begin{pgfscope}%
\pgfsys@transformshift{0.631546in}{2.488647in}%
\pgfsys@useobject{currentmarker}{}%
\end{pgfscope}%
\end{pgfscope}%
\begin{pgfscope}%
\definecolor{textcolor}{rgb}{0.150000,0.150000,0.150000}%
\pgfsetstrokecolor{textcolor}%
\pgfsetfillcolor{textcolor}%
\pgftext[x=0.273521in,y=2.438505in,left,base]{\color{textcolor}\sffamily\fontsize{10.000000}{12.000000}\selectfont \(\displaystyle 1.0\)}%
\end{pgfscope}%
\begin{pgfscope}%
\definecolor{textcolor}{rgb}{0.150000,0.150000,0.150000}%
\pgfsetstrokecolor{textcolor}%
\pgfsetfillcolor{textcolor}%
\pgftext[x=0.217965in,y=1.541647in,,bottom,rotate=90.000000]{\color{textcolor}\sffamily\fontsize{11.000000}{13.200000}\selectfont Val. Accuracy}%
\end{pgfscope}%
\begin{pgfscope}%
\pgfpathrectangle{\pgfqpoint{0.631546in}{0.594647in}}{\pgfqpoint{3.408546in}{1.894001in}} %
\pgfusepath{clip}%
\pgfsetroundcap%
\pgfsetroundjoin%
\pgfsetlinewidth{1.756562pt}%
\definecolor{currentstroke}{rgb}{0.400000,0.760784,0.647059}%
\pgfsetstrokecolor{currentstroke}%
\pgfsetdash{}{0pt}%
\pgfpathmoveto{\pgfqpoint{0.631887in}{0.682958in}}%
\pgfpathlineto{\pgfqpoint{0.632228in}{0.899040in}}%
\pgfpathlineto{\pgfqpoint{0.632569in}{1.458973in}}%
\pgfpathlineto{\pgfqpoint{0.632910in}{1.278592in}}%
\pgfpathlineto{\pgfqpoint{0.633250in}{1.387572in}}%
\pgfpathlineto{\pgfqpoint{0.633591in}{1.393209in}}%
\pgfpathlineto{\pgfqpoint{0.633932in}{1.449578in}}%
\pgfpathlineto{\pgfqpoint{0.634273in}{1.227859in}}%
\pgfpathlineto{\pgfqpoint{0.634614in}{1.184643in}}%
\pgfpathlineto{\pgfqpoint{0.634955in}{1.419514in}}%
\pgfpathlineto{\pgfqpoint{0.638363in}{1.318050in}}%
\pgfpathlineto{\pgfqpoint{0.641772in}{1.355629in}}%
\pgfpathlineto{\pgfqpoint{0.645180in}{1.481520in}}%
\pgfpathlineto{\pgfqpoint{0.648589in}{1.492794in}}%
\pgfpathlineto{\pgfqpoint{0.651997in}{1.436425in}}%
\pgfpathlineto{\pgfqpoint{0.655406in}{1.532252in}}%
\pgfpathlineto{\pgfqpoint{0.658814in}{1.618685in}}%
\pgfpathlineto{\pgfqpoint{0.662223in}{1.586743in}}%
\pgfpathlineto{\pgfqpoint{0.665632in}{1.571711in}}%
\pgfpathlineto{\pgfqpoint{0.699717in}{1.782155in}}%
\pgfpathlineto{\pgfqpoint{0.733802in}{1.990721in}}%
\pgfpathlineto{\pgfqpoint{0.767888in}{2.109096in}}%
\pgfpathlineto{\pgfqpoint{0.801973in}{2.203044in}}%
\pgfpathlineto{\pgfqpoint{0.836059in}{2.308266in}}%
\pgfpathlineto{\pgfqpoint{0.870144in}{2.327056in}}%
\pgfpathlineto{\pgfqpoint{0.904230in}{2.328935in}}%
\pgfpathlineto{\pgfqpoint{0.938315in}{2.364636in}}%
\pgfpathlineto{\pgfqpoint{0.972401in}{2.417247in}}%
\pgfpathlineto{\pgfqpoint{1.006486in}{2.389062in}}%
\pgfpathlineto{\pgfqpoint{1.040572in}{2.426642in}}%
\pgfpathlineto{\pgfqpoint{1.074657in}{2.441673in}}%
\pgfpathlineto{\pgfqpoint{1.108743in}{2.417247in}}%
\pgfpathlineto{\pgfqpoint{1.142828in}{2.398457in}}%
\pgfpathlineto{\pgfqpoint{1.176913in}{2.385304in}}%
\pgfpathlineto{\pgfqpoint{1.210999in}{2.419126in}}%
\pgfpathlineto{\pgfqpoint{1.245084in}{2.426642in}}%
\pgfpathlineto{\pgfqpoint{1.279170in}{2.439794in}}%
\pgfpathlineto{\pgfqpoint{1.313255in}{2.445431in}}%
\pgfpathlineto{\pgfqpoint{1.347341in}{2.424763in}}%
\pgfpathlineto{\pgfqpoint{1.381426in}{2.445431in}}%
\pgfpathlineto{\pgfqpoint{1.415512in}{2.434157in}}%
\pgfpathlineto{\pgfqpoint{1.449597in}{2.458584in}}%
\pgfpathlineto{\pgfqpoint{1.483683in}{2.441673in}}%
\pgfpathlineto{\pgfqpoint{1.517768in}{2.437915in}}%
\pgfpathlineto{\pgfqpoint{1.551854in}{2.452947in}}%
\pgfpathlineto{\pgfqpoint{1.585939in}{2.469858in}}%
\pgfpathlineto{\pgfqpoint{1.620024in}{2.452947in}}%
\pgfpathlineto{\pgfqpoint{1.654110in}{2.421005in}}%
\pgfpathlineto{\pgfqpoint{1.688195in}{2.426642in}}%
\pgfpathlineto{\pgfqpoint{1.722281in}{2.454826in}}%
\pgfpathlineto{\pgfqpoint{1.756366in}{2.447310in}}%
\pgfpathlineto{\pgfqpoint{1.790452in}{2.456705in}}%
\pgfpathlineto{\pgfqpoint{1.824537in}{2.452947in}}%
\pgfpathlineto{\pgfqpoint{1.858623in}{2.471737in}}%
\pgfpathlineto{\pgfqpoint{1.892708in}{2.432278in}}%
\pgfpathlineto{\pgfqpoint{1.926794in}{2.456705in}}%
\pgfpathlineto{\pgfqpoint{1.960879in}{2.460463in}}%
\pgfpathlineto{\pgfqpoint{1.994965in}{2.407852in}}%
\pgfpathlineto{\pgfqpoint{2.029050in}{2.456705in}}%
\pgfpathlineto{\pgfqpoint{2.063135in}{2.460463in}}%
\pgfpathlineto{\pgfqpoint{2.097221in}{2.483011in}}%
\pgfpathlineto{\pgfqpoint{2.131306in}{2.439794in}}%
\pgfpathlineto{\pgfqpoint{2.165392in}{2.460463in}}%
\pgfpathlineto{\pgfqpoint{2.199477in}{2.464221in}}%
\pgfpathlineto{\pgfqpoint{2.233563in}{2.445431in}}%
\pgfpathlineto{\pgfqpoint{2.267648in}{2.415368in}}%
\pgfpathlineto{\pgfqpoint{2.301734in}{2.464221in}}%
\pgfpathlineto{\pgfqpoint{2.335819in}{2.467979in}}%
\pgfpathlineto{\pgfqpoint{2.369905in}{2.430399in}}%
\pgfpathlineto{\pgfqpoint{2.403990in}{2.449189in}}%
\pgfpathlineto{\pgfqpoint{2.438076in}{2.467979in}}%
\pgfpathlineto{\pgfqpoint{2.472161in}{2.464221in}}%
\pgfpathlineto{\pgfqpoint{2.506246in}{2.447310in}}%
\pgfpathlineto{\pgfqpoint{2.540332in}{2.460463in}}%
\pgfpathlineto{\pgfqpoint{2.574417in}{2.466100in}}%
\pgfpathlineto{\pgfqpoint{2.608503in}{2.479253in}}%
\pgfpathlineto{\pgfqpoint{2.642588in}{2.454826in}}%
\pgfpathlineto{\pgfqpoint{2.676674in}{2.464221in}}%
\pgfpathlineto{\pgfqpoint{2.710759in}{2.469858in}}%
\pgfpathlineto{\pgfqpoint{2.744845in}{2.473616in}}%
\pgfpathlineto{\pgfqpoint{2.778930in}{2.447310in}}%
\pgfpathlineto{\pgfqpoint{2.813016in}{2.466100in}}%
\pgfpathlineto{\pgfqpoint{2.847101in}{2.469858in}}%
\pgfpathlineto{\pgfqpoint{2.881187in}{2.467979in}}%
\pgfpathlineto{\pgfqpoint{2.915272in}{2.473616in}}%
\pgfpathlineto{\pgfqpoint{2.949357in}{2.464221in}}%
\pgfpathlineto{\pgfqpoint{2.983443in}{2.479253in}}%
\pgfpathlineto{\pgfqpoint{3.017528in}{2.477374in}}%
\pgfpathlineto{\pgfqpoint{3.051614in}{2.466100in}}%
\pgfpathlineto{\pgfqpoint{3.085699in}{2.449189in}}%
\pgfpathlineto{\pgfqpoint{3.119785in}{2.466100in}}%
\pgfpathlineto{\pgfqpoint{3.153870in}{2.475495in}}%
\pgfpathlineto{\pgfqpoint{3.187956in}{2.473616in}}%
\pgfpathlineto{\pgfqpoint{3.222041in}{2.483011in}}%
\pgfpathlineto{\pgfqpoint{3.256127in}{2.464221in}}%
\pgfpathlineto{\pgfqpoint{3.290212in}{2.458584in}}%
\pgfpathlineto{\pgfqpoint{3.324298in}{2.451068in}}%
\pgfpathlineto{\pgfqpoint{3.358383in}{2.475495in}}%
\pgfpathlineto{\pgfqpoint{3.392468in}{2.471737in}}%
\pgfpathlineto{\pgfqpoint{3.426554in}{2.460463in}}%
\pgfpathlineto{\pgfqpoint{3.460639in}{2.456705in}}%
\pgfpathlineto{\pgfqpoint{3.494725in}{2.471737in}}%
\pgfpathlineto{\pgfqpoint{3.528810in}{2.475495in}}%
\pgfpathlineto{\pgfqpoint{3.562896in}{2.454826in}}%
\pgfpathlineto{\pgfqpoint{3.596981in}{2.466100in}}%
\pgfpathlineto{\pgfqpoint{3.631067in}{2.479253in}}%
\pgfpathlineto{\pgfqpoint{3.665152in}{2.460463in}}%
\pgfpathlineto{\pgfqpoint{3.699238in}{2.467979in}}%
\pgfusepath{stroke}%
\end{pgfscope}%
\begin{pgfscope}%
\pgfpathrectangle{\pgfqpoint{0.631546in}{0.594647in}}{\pgfqpoint{3.408546in}{1.894001in}} %
\pgfusepath{clip}%
\pgfsetroundcap%
\pgfsetroundjoin%
\pgfsetlinewidth{1.756562pt}%
\definecolor{currentstroke}{rgb}{0.988235,0.552941,0.384314}%
\pgfsetstrokecolor{currentstroke}%
\pgfsetdash{}{0pt}%
\pgfpathmoveto{\pgfqpoint{0.631887in}{0.660411in}}%
\pgfpathlineto{\pgfqpoint{0.632228in}{0.761875in}}%
\pgfpathlineto{\pgfqpoint{0.632569in}{1.049357in}}%
\pgfpathlineto{\pgfqpoint{0.632910in}{1.462731in}}%
\pgfpathlineto{\pgfqpoint{0.633250in}{1.348113in}}%
\pgfpathlineto{\pgfqpoint{0.633591in}{1.045599in}}%
\pgfpathlineto{\pgfqpoint{0.633932in}{1.502189in}}%
\pgfpathlineto{\pgfqpoint{0.634273in}{1.346234in}}%
\pgfpathlineto{\pgfqpoint{0.634614in}{1.408240in}}%
\pgfpathlineto{\pgfqpoint{0.634955in}{1.492794in}}%
\pgfpathlineto{\pgfqpoint{0.638363in}{1.250407in}}%
\pgfpathlineto{\pgfqpoint{0.641772in}{1.436425in}}%
\pgfpathlineto{\pgfqpoint{0.645180in}{1.564195in}}%
\pgfpathlineto{\pgfqpoint{0.648589in}{1.487157in}}%
\pgfpathlineto{\pgfqpoint{0.651997in}{1.620564in}}%
\pgfpathlineto{\pgfqpoint{0.655406in}{1.464610in}}%
\pgfpathlineto{\pgfqpoint{0.658814in}{1.699481in}}%
\pgfpathlineto{\pgfqpoint{0.662223in}{1.939989in}}%
\pgfpathlineto{\pgfqpoint{0.665632in}{2.218076in}}%
\pgfpathlineto{\pgfqpoint{0.699717in}{2.099701in}}%
\pgfpathlineto{\pgfqpoint{0.733802in}{2.073395in}}%
\pgfpathlineto{\pgfqpoint{0.767888in}{2.447310in}}%
\pgfpathlineto{\pgfqpoint{0.801973in}{2.460463in}}%
\pgfpathlineto{\pgfqpoint{0.836059in}{2.475495in}}%
\pgfpathlineto{\pgfqpoint{0.870144in}{2.483011in}}%
\pgfpathlineto{\pgfqpoint{0.904230in}{2.481132in}}%
\pgfpathlineto{\pgfqpoint{0.938315in}{2.473616in}}%
\pgfpathlineto{\pgfqpoint{0.972401in}{2.486769in}}%
\pgfpathlineto{\pgfqpoint{1.006486in}{2.484890in}}%
\pgfpathlineto{\pgfqpoint{1.040572in}{2.413489in}}%
\pgfpathlineto{\pgfqpoint{1.074657in}{2.486769in}}%
\pgfpathlineto{\pgfqpoint{1.108743in}{2.484890in}}%
\pgfpathlineto{\pgfqpoint{1.142828in}{2.484890in}}%
\pgfpathlineto{\pgfqpoint{1.176913in}{2.477374in}}%
\pgfpathlineto{\pgfqpoint{1.210999in}{2.486769in}}%
\pgfpathlineto{\pgfqpoint{1.245084in}{2.484890in}}%
\pgfpathlineto{\pgfqpoint{1.279170in}{2.484890in}}%
\pgfpathlineto{\pgfqpoint{1.313255in}{2.370272in}}%
\pgfpathlineto{\pgfqpoint{1.347341in}{2.434157in}}%
\pgfpathlineto{\pgfqpoint{1.381426in}{2.484890in}}%
\pgfpathlineto{\pgfqpoint{1.415512in}{2.484890in}}%
\pgfpathlineto{\pgfqpoint{1.449597in}{2.481132in}}%
\pgfpathlineto{\pgfqpoint{1.483683in}{2.484890in}}%
\pgfpathlineto{\pgfqpoint{1.517768in}{2.481132in}}%
\pgfpathlineto{\pgfqpoint{1.551854in}{2.486769in}}%
\pgfpathlineto{\pgfqpoint{1.585939in}{2.304509in}}%
\pgfpathlineto{\pgfqpoint{1.620024in}{2.343967in}}%
\pgfpathlineto{\pgfqpoint{1.654110in}{2.484890in}}%
\pgfpathlineto{\pgfqpoint{1.688195in}{2.477374in}}%
\pgfpathlineto{\pgfqpoint{1.722281in}{2.481132in}}%
\pgfpathlineto{\pgfqpoint{1.756366in}{2.479253in}}%
\pgfpathlineto{\pgfqpoint{1.790452in}{2.486769in}}%
\pgfpathlineto{\pgfqpoint{1.824537in}{2.488647in}}%
\pgfpathlineto{\pgfqpoint{1.858623in}{2.486769in}}%
\pgfpathlineto{\pgfqpoint{1.892708in}{2.340209in}}%
\pgfpathlineto{\pgfqpoint{1.926794in}{2.488647in}}%
\pgfpathlineto{\pgfqpoint{1.960879in}{2.486769in}}%
\pgfpathlineto{\pgfqpoint{1.994965in}{2.477374in}}%
\pgfpathlineto{\pgfqpoint{2.029050in}{2.486769in}}%
\pgfpathlineto{\pgfqpoint{2.063135in}{2.481132in}}%
\pgfpathlineto{\pgfqpoint{2.097221in}{2.466100in}}%
\pgfpathlineto{\pgfqpoint{2.131306in}{2.488647in}}%
\pgfpathlineto{\pgfqpoint{2.165392in}{2.488647in}}%
\pgfpathlineto{\pgfqpoint{2.199477in}{2.488647in}}%
\pgfpathlineto{\pgfqpoint{2.233563in}{2.488647in}}%
\pgfpathlineto{\pgfqpoint{2.267648in}{2.486769in}}%
\pgfpathlineto{\pgfqpoint{2.301734in}{2.454826in}}%
\pgfpathlineto{\pgfqpoint{2.335819in}{2.486769in}}%
\pgfpathlineto{\pgfqpoint{2.369905in}{2.488647in}}%
\pgfpathlineto{\pgfqpoint{2.403990in}{2.467979in}}%
\pgfpathlineto{\pgfqpoint{2.438076in}{2.231229in}}%
\pgfpathlineto{\pgfqpoint{2.472161in}{2.458584in}}%
\pgfpathlineto{\pgfqpoint{2.506246in}{2.488647in}}%
\pgfpathlineto{\pgfqpoint{2.540332in}{2.486769in}}%
\pgfpathlineto{\pgfqpoint{2.574417in}{2.486769in}}%
\pgfpathlineto{\pgfqpoint{2.608503in}{2.488647in}}%
\pgfpathlineto{\pgfqpoint{2.642588in}{2.486769in}}%
\pgfpathlineto{\pgfqpoint{2.676674in}{2.456705in}}%
\pgfpathlineto{\pgfqpoint{2.710759in}{2.488647in}}%
\pgfpathlineto{\pgfqpoint{2.744845in}{2.488647in}}%
\pgfpathlineto{\pgfqpoint{2.778930in}{2.488647in}}%
\pgfpathlineto{\pgfqpoint{2.813016in}{2.488647in}}%
\pgfpathlineto{\pgfqpoint{2.847101in}{2.486769in}}%
\pgfpathlineto{\pgfqpoint{2.881187in}{2.486769in}}%
\pgfpathlineto{\pgfqpoint{2.915272in}{2.488647in}}%
\pgfpathlineto{\pgfqpoint{2.949357in}{2.473616in}}%
\pgfpathlineto{\pgfqpoint{2.983443in}{2.486769in}}%
\pgfpathlineto{\pgfqpoint{3.017528in}{2.486769in}}%
\pgfpathlineto{\pgfqpoint{3.051614in}{2.475495in}}%
\pgfpathlineto{\pgfqpoint{3.085699in}{2.471737in}}%
\pgfpathlineto{\pgfqpoint{3.119785in}{2.488647in}}%
\pgfpathlineto{\pgfqpoint{3.153870in}{2.458584in}}%
\pgfpathlineto{\pgfqpoint{3.187956in}{2.488647in}}%
\pgfpathlineto{\pgfqpoint{3.222041in}{2.488647in}}%
\pgfpathlineto{\pgfqpoint{3.256127in}{2.486769in}}%
\pgfpathlineto{\pgfqpoint{3.290212in}{2.486769in}}%
\pgfpathlineto{\pgfqpoint{3.324298in}{2.469858in}}%
\pgfpathlineto{\pgfqpoint{3.358383in}{2.458584in}}%
\pgfpathlineto{\pgfqpoint{3.392468in}{2.488647in}}%
\pgfpathlineto{\pgfqpoint{3.426554in}{2.469858in}}%
\pgfpathlineto{\pgfqpoint{3.460639in}{2.475495in}}%
\pgfpathlineto{\pgfqpoint{3.494725in}{2.488647in}}%
\pgfpathlineto{\pgfqpoint{3.528810in}{2.488647in}}%
\pgfpathlineto{\pgfqpoint{3.562896in}{2.454826in}}%
\pgfpathlineto{\pgfqpoint{3.596981in}{2.488647in}}%
\pgfpathlineto{\pgfqpoint{3.631067in}{2.486769in}}%
\pgfpathlineto{\pgfqpoint{3.665152in}{2.469858in}}%
\pgfpathlineto{\pgfqpoint{3.699238in}{2.178618in}}%
\pgfusepath{stroke}%
\end{pgfscope}%
\begin{pgfscope}%
\pgfsetrectcap%
\pgfsetmiterjoin%
\pgfsetlinewidth{1.254687pt}%
\definecolor{currentstroke}{rgb}{0.150000,0.150000,0.150000}%
\pgfsetstrokecolor{currentstroke}%
\pgfsetdash{}{0pt}%
\pgfpathmoveto{\pgfqpoint{0.631546in}{0.594647in}}%
\pgfpathlineto{\pgfqpoint{0.631546in}{2.488647in}}%
\pgfusepath{stroke}%
\end{pgfscope}%
\begin{pgfscope}%
\pgfsetrectcap%
\pgfsetmiterjoin%
\pgfsetlinewidth{1.254687pt}%
\definecolor{currentstroke}{rgb}{0.150000,0.150000,0.150000}%
\pgfsetstrokecolor{currentstroke}%
\pgfsetdash{}{0pt}%
\pgfpathmoveto{\pgfqpoint{0.631546in}{0.594647in}}%
\pgfpathlineto{\pgfqpoint{4.040092in}{0.594647in}}%
\pgfusepath{stroke}%
\end{pgfscope}%
\begin{pgfscope}%
\pgfsetroundcap%
\pgfsetroundjoin%
\pgfsetlinewidth{1.756562pt}%
\definecolor{currentstroke}{rgb}{0.400000,0.760784,0.647059}%
\pgfsetstrokecolor{currentstroke}%
\pgfsetdash{}{0pt}%
\pgfpathmoveto{\pgfqpoint{0.756546in}{0.991986in}}%
\pgfpathlineto{\pgfqpoint{1.034324in}{0.991986in}}%
\pgfusepath{stroke}%
\end{pgfscope}%
\begin{pgfscope}%
\definecolor{textcolor}{rgb}{0.150000,0.150000,0.150000}%
\pgfsetstrokecolor{textcolor}%
\pgfsetfillcolor{textcolor}%
\pgftext[x=1.145435in,y=0.943375in,left,base]{\color{textcolor}\sffamily\fontsize{10.000000}{12.000000}\selectfont fc}%
\end{pgfscope}%
\begin{pgfscope}%
\pgfsetroundcap%
\pgfsetroundjoin%
\pgfsetlinewidth{1.756562pt}%
\definecolor{currentstroke}{rgb}{0.988235,0.552941,0.384314}%
\pgfsetstrokecolor{currentstroke}%
\pgfsetdash{}{0pt}%
\pgfpathmoveto{\pgfqpoint{0.756546in}{0.795258in}}%
\pgfpathlineto{\pgfqpoint{1.034324in}{0.795258in}}%
\pgfusepath{stroke}%
\end{pgfscope}%
\begin{pgfscope}%
\definecolor{textcolor}{rgb}{0.150000,0.150000,0.150000}%
\pgfsetstrokecolor{textcolor}%
\pgfsetfillcolor{textcolor}%
\pgftext[x=1.145435in,y=0.746647in,left,base]{\color{textcolor}\sffamily\fontsize{10.000000}{12.000000}\selectfont conv}%
\end{pgfscope}%
\end{pgfpicture}%
\makeatother%
\endgroup%
}
			\caption{Validation Accuracy}
			\label{fig:tomplotmaxvalacc}
		\end{subfigure}
		\caption[Two-or-more Clumps Problem and Structural Priors]{TODO. Class imbalance responsible for cycle in loss?}
		\label{fig:tomplot}
	\end{figure}
	
\section{The Neocognitron}
	\begin{chapquote}{Kunihiko Fukushima, \textit{Neocognitron, 
				%A Self-organizing Neural Network Model
				%for a Mechanism of Pattern Recognition
				%Unaffected by Shift in Position , 
				Biol.\ Cybernetics 1980}}
		`` One of the largest and long-standing difficulties in designing a pattern-recognizing machine has been the problem how to cope with the shift in position and the distortion in shape of the input patterns. The neocognitron proposed in this paper gives a drastic solution to this difficulty.
		''
	\end{chapquote}
	
	In an era in which fully connected networks were used to learn any input type, \citet{Fuk80} showed that for structured inputs a drastically different architecture could make a big difference in generalization. The \emph{Neocognitron}~\citep{Fuk80, fukushima2013artificial} was a biologically motivated architecture, motivated by what are typically called simple and complex cells in the primary visual cortex (V1), as found by \citet{Hubel1959a}. To model simple cells; cells whose response correlated with simple oriented edges in a translation invariant manner, the neocognitron used shared weights which were connected to local image patches of the input image (and were not simply described as convolution of a filter). Complex cells were modelled by a ``blurring'' operation, what we now term more generally as \emph{pooling}. The neocognitron network consisted of alternating layers of simple and complex cells, \ie alternating convolution and pooling layers, much as seen in state of the art convolutional networks.

	While the Neocognitron was ahead of its time, and is now recognized as the first iteration of what were to become convolutional neural networks, the article's neurological focus, timing, and to some degree country of origin, meant that it was somewhat unnoticed in the mainstream field of connectionist research. In fact Yann LeCun recounts specifically his interest in Japan, rather than any particular citation, having lead to his discovery of the work\footnote{As related in a Q\&A session at the 2016 International Computer Vision Summer School (ICVSS)}.
	
	\section{Convolutional Neural Networks}
	\begin{chapquote}{Yann LeCun, \textit{Backpropagation Applied to Handwritten Zip Code Recognition, 1989}}
		``Classical work in visual pattern recognition has demonstrated the advantage of extracting local features and combining them to form higher order features. Such knowledge can be easily built into the network by forcing the hidden units to combine only local sources of information. Distinctive features of an object can appear at various location on the input image. Therefore it seems judicious to have a set of feature detectors that can detect a particular instance of a feature anywhere on the input place. Since the precise location of a feature is not relevant to the classification, we can afford to lose some position information in the process. Nevertheless, approximate position information must be preserved, to allow the next levels to detect higher order, more complex features.''
	\end{chapquote}
	
	Despite the pioneering novelty of the work on neocognitrons, it was only following the simplification and improvement of~\citet{lecun1989backpropagation,Lecun1998} in both the description of the network and its operation that it gained wider acknowledgement as a breakthrough for image recognition. In their work the local shared weights of the neocognitron are put in the context of convolution, and the averaging operation replaces with max-pooling. The application to handwritten digit recognition gave state of the art results, and would result in the \emph{LeNet5} network, still used today in commercial applications.
	
	The application of the LeNet-style CNN architecture to more complex problems, however, proved infeasible. These problems required a deeper hierarchy of representation, which implied a large number of layers. Networks with a large number of layers proved to be un-trainable due to numerous issues with the model itself, notably vanishing gradients~\citep{hochreiter1991untersuchungen}, and the lack of large datasets and computational power at the time. Convolutional neural networks fell out of favour, and were passed over in favour of the more successful paradigm of using hand-crafted local features for many tasks, and in particular the problem of object instance recognition was well addressed by such solutions. Meanwhile object class recognition remained a difficult problem, for which the best solutions were deformable parts models, also based on local features.
	
	
	\section{Learning Neural Network Architectures}
	\begin{chapquote}{David MacKay, \textit{A Practical Bayesian Framework for Backprop Networks, Neural Computation 1991}}
		``There are many knobs on the black box of 'backprop' (learning by back-propagation of
		errors). Generally these knobs are set by rules of thumb, trial and error, and the use of reserved test data to assess generalization ability (or more sophisticated cross-validation).''
	\end{chapquote}
	
	\mynote{todo talk about tiling, and MaKay's work}
	\citet{mezard1989learning,MacKay91,mackay1992practical}
	
	\section{Network Pruning, Compression and Quantization}
	\mynote{todo}
	Another approach to learning efficient networks is pruning
	\citet{lecun1989optimal,sietsma1988neural,Xing2009,journals/corr/HanMD15,ullrich2017soft,}
	
	
	\subsection{Structural Priors}
	In practice when fitting a curve, we have little idea of what order polynomial would best fit the data. Necessarily, we must use a relatively higher order curve to fit the data. Similarly, with a neural network we rarely have knowledge of the underlying structure of the solution (but when we do, we should use it to parameterize our models appropriately~\citep{jain2016structural}), instead we must use networks with more parameters than necessary to ensure that there is enough capacity to learn the underlying, but potentially sparse solution. Over-parameterization of a model generally leads to poor generalization however, due to overfitting. To prevent this, there are a number of methods in which we can relate our prior knowledge that the model is over-parameterized to the optimization:
	
	\begin{description}
	\item[Weak Priors]
	Knowing only that our model is over-parameterized is a relatively weak prior, however we can encode this into the fit by using a regularization penalty. This restricts the model to effectively use only a small number of the parameters by adding a penalty, for example on the L2 norm of the model weights. For polynomial regression, this is called \emph{ridge regression}, while for neural networks it is called \emph{weight decay}.
	
	\item[Strong Priors]
	With more prior information on the task, \eg from the convexity of the polynomial, we may imply that a certain order polynomial is more appropriate, and restrict learning to that order. For example, given samples from a polynomial appear to be convex, we can surmise that the polynomial is likely to be of an even or \engordnumber{2}-order, and restrict our fit to be of that order. 
    \end{description}
    
	Although neural networks are usually posed as general learning machines, time and again it has been demonstrated that neural networks only truly stand out as a learning method when we use strong structural priors, encoding our prior knowledge of the task in the architecture itself. As observed by \citet{denker1987large} this may be considered closer to modifying the problem to be solved itself, rather than changing the learning method alone. For example, by asking a CNN to learn to classify a dataset, we are asking the network to ``Classify these types of images'', whereas by asking a fully-connected network to classify the same dataset, we are asking the network ``Classify these types of data''. The first task is inherently easier.

    Structural priors may be considered one of the greatest contributions to the success of deep learning, but arguably can also be considered the cause of its greatest failure. Structural priors allow deep models to more easily learn specific tasks, \eg image classification or speech recognition. At the same time, by specializing our network models, we are moving further away from the goal of general artificial intelligence, \ie learning models that can do both tasks.
\end{document}
